\providecommand {\OPTIONS}{screen,review}

\documentclass[sigplan,nonacm,\OPTIONS]{acmart}
\providecommand \Draft {\True}
\providecommand \Final {\False}
\providecommand \Long {\True}
\providecommand \Anonymous {\False}

\message {*** DRAFT ***}


\usepackage{suspended}
\setlist[enumerate]{label*=\textbf{(\arabic*)}}

\usepackage{balance} % For balanced columns
\usepackage{marginnote}

\citestyle{acmnumeric}

\author{Alistair O'Brien}
\email{ajo41@cam.ac.uk}
\orcid{0009-0009-0055-7793}
\affiliation{
  \institution{University of Cambridge}
  \city{Cambridge}
  \country{United Kingdom}
}

\author{Didier R\'emy}
\email{didier.remy@inria.fr}
\affiliation{
  \institution{Inria}
  \country{France}
}

\author{Gabriel Scherer}
\email{gabriel.scherer@inria.fr} \affiliation{
  \institution{Inria \& Irif, Universit\'e Paris Cit\'e}
  \country{France}
}


\title{Omnidirectional type inference for \ML}

\begin{document}

\maketitle

At the heart of the Damas-Hindley-Milner (\HM) type system lies the ability to
\emph{guess} types. For instance, in $\efun \x \e$, the parameter type for $\x$
is \emph{guessed} and subsequently constrained by the body $\e$. The elegance
of the \ML discipline is that such guesses are never arbitrary: there always
exists a ``most general guess''---a \emph{principal type}. Every well-typed
expression $\e$ admits a principal type $\ts$ from which all other valid types
for $\e$ are instances of $\ts$. This guarantees both predictability and
efficiency of inference: local typing decisions are always optimal without
resorting to backtracking.

\parcomment {Extensions often break pincipality}

Principality, however, is \emph{fragile}. Many extensions of \ML---GADTs
\citep*{conf/icfp/SchrijversJSV09, conf/aplas/GarrigueR13}, higher-rank
polymorphism \citep*{conf/popl/OderskyL96, journals/pacmpl/SerranoHJV20}, and
static overloading
\citep*{Chargueraud-Bodin-Dunfield-Riboulet/jfla2025}---break it.
%
Consider the following example in \OCaml
%% Useless footnote
%% \footnote{
%% %
%%   All \OCaml examples should be checked with the options
%%   \code{-principal -w +41+18 -warn-error +41+18}, which enables
%%   principal type inference and escaltes the associated warnings to errors.
%% }
using overloaded record field names:
\begin{program}[input,escapechar=@,belowskip=0em]
type point    = {x : int; y : int}
type gs_point = {x : int; y : int; color : int}

let getx p  = r.x
\end{program}
\begin{program}[error,aboveskip=0em]
             ^^^
Error (warning 41 [ambiguous-name]):
x belongs to several types: gs_point point.
\end{program}
When typing \code{getx}, one could \emph{guess} the type of \code{r} to be
either \code{point} or \code{gs_point}---neither of which is strictly more
general than the other, violating principality!

\parcomment{Known information fixes principality / Robust forms}

Principality can always be recovered through explicit type annotations---for
instance, annotating the projection \code{r.x} as \code{(r : point).x}.  Each
\emph{fragile} construct (\eg an overloaded field projection $\efield \e
\elab$) therefore admits a \emph{robust} counterpart (\eg a qualified
projection $\exfield \e \T \elab$), whose \emph{natural} typing rules preserve
principality.
%
However, robust forms are cumbersome to use, as they always require explicit
annotations. Inference algorithms thus rely on a minimal amount of \emph{known}
type information to disambiguate fragile terms and elaborate them into their
robust counterparts \citep*{Pottier-Regis-Gianas/stratified@popl06}.

\parcomment{The problem}

In practice, specifying what counts as \emph{known type information}
declaratively is difficult.
%
Existing approaches typically rely on one of two frameworks:
\emph{bidirectional} type inference \citep*{conf/popl/PierceT98} and
\emph{\geninst-directional} inference~\citep*{Garrigue-Remy/poly-ml}. Both
impose some \emph{static} ordering of inference, using it to propagate inferred
types and user-provided annotations as known information.
%
For instance, in a bidirectional system, a type $\t$ is \emph{known}
when it is:
%
\begin{enumerate*}
  \item part of an annotation;
  \item supplied as input to a checking conclusion ($\G \th e \Leftarrow \t$); or
  \item an output of a synthesizing premise ($\G \th \e \Rightarrow \t$).
\end{enumerate*}
\XDR{We still do not see the difference between being a premise or a
conclusion}\relax
While effective in many settings, the rigidity of a static ordering causes even
simple examples---whose types could easily be guessed---to be rejected; for
instance, \code{fun r -> (r.x, r.color)} is ill-typed in \OCaml.


\parcomment {Our solution}

We propose \emph{omnidirectional} type inference, which relies on a
\emph{dynamic} order of inference: constraints may be solved in any order,
suspending whenever progress requires \emph{known} type information. Other
constraints may continue to be sol\-ved; once the missing information becomes
available (typically via unification), the suspended typing constraints are
resumed.

\begin{local}
This approach comes naturally for simply typed systems (when inference is
purely unification-based), but becomes challenging in the presence of \ML-style
\emph{let-generalization}. In existing \ML inference algorithms, typing a
let-binding $\elet \x \ea \eb$ follows a fixed order: first type $\ea$, then
generalize its type, and finally type the body $\eb$ under the extended
environment.
%
We solve this by introducing two novel devices:
\begin{enumerate*}

  \item \emph{suspended match constraints}, which defer disambiguation until
    sufficient contextual information is known, and

  \item \emph{incrimental instantiation} \ie the ability to instantiate type
    schemes that are not yet fully determined and consequently revisit their
    instances when they are being refined, \emph{incrementally}.

\end{enumerate*}
\sloppy

\end{local}

Thus far, we have developed:
\begin{enumerate}

  \item A novel constraint language for omnidirectional inference, equipped
    with a declarative semantics for suspended constraints and a new
    characterization of \emph{known} type information.

  \item The \OML calculus, an extension of \ML featuring advanced extensions of
    \OCaml such as static overloading of record labels, and an
    \emph{omnidirectional recipe} to derive its typing rules.

  \item A sound, complete, and \emph{principal} constraint-based inference
    algorithm for \OML.

  \item
    A formalized constraint solver as a series of non-deter\-ministic rewriting
    rules and an efficient implementation
    % De-anonymizing URL (available at https://github.com/johnyob/mlsus)
    using \emph{incremental instantiation}.

\end{enumerate}

Suspending unsolved constraints is not new---it appears in dependently typed
systems \citep*{huet-unif, Miller/pattern-unif@iclp91, allais2022typos} and in
\OutsideIn \citep*{conf/icfp/SchrijversJSV09} in the form of \emph{delayed
implication} constraints---but combining this technique with \ML-style
\emph{local let-generalization}, and giving it a declarative semantics, is
novel.
%
Our ultimate aim is to apply this to \OCaml, where \emph{local
let-generalization} is indepensible (unlike in \Haskell, which only generalizes
at the top level), yet this makes omnidirectional inference especially
challenging to specify and implement.

\clearpage

\bibliography{suspended}

\end{document}
