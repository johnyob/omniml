\documentclass[10pt]{cam-letter}

\camname{Alistair O'Brien\\(on behalf of all authors)}
\camdepartment{Department of Computer\\Science and Technology}
\camaddress{%
  William Gates Building,\\
  15 JJ Thomson Ave,\\
  Cambridge CB3 0FD
}
\camphone{01223 763500}
\camwebsite{cst.cam.ac.uk}

\camsubject{Omnidirectional type inference for \ML}
\address{%
  \textbf{PhD Student}\\
  Department of Computer Science and Technology, \\
  University of Cambridge\\
  Email: \href{mailto:ajo41@cam.ac.uk}{ajo41@cam.ac.uk}\\
  Website: \href{https://ajo41.dev}{ajo41.dev}
}

\usepackage{letterbib}
\usepackage{suspended}

\begin{document}

\begin{letter}{
  Prof. Alastair Donaldson,\\
  Editor-in-Chief,\\
  ACM TOPLAS
}

\opening{Dear Prof. Donaldson,}

% Point: introduce work and us

Please consider for publication in TOPLAS our paper entitled
``\emph{Omnidirectional type inference for \ML: principality any way}'',
authored by myself, Didier R\'emy, and Gabriel Scherer.

% Point: confirming that the work hasn't been submitted elsewhere
% Aside:
% - Frozen constraints (Marinot)
% - WITS 2026

We confirm that this manuscript has not been published elsewhere and is not
under consideration by any other journal or conference. Two short, preliminary
abstracts related to this line of work exist:
%
\begin{enumerate*}

  \item

  \citet*{frozen-constraints-2021}, and

  \item \citet*{omniml-wits-2026}, submitted to WITS 2026 and
  presently under review.

\end{enumerate*}
%
Both abstracts are brief summaries that omit the technical developments,
metatheory, and proofs contained in this manuscript. The TOPLAS submission
constitutes the full and definitive version of the work.

% Point: Why this work belongs in TOPLAS

This work aligns well with TOPLAS's focus on programming language theory. It
contributes the first concrete step in a broader program of
\emph{omnidirectional} type inference, whose long-term aim is to support
advanced \ML extensions such as static overloading, GADTs, and first-class
polymorphism.
%
A key difficulty in this space is preserving \emph{principality}: the
existence of most general types that characterise all valid typings for a
program.  Principality is crucial for predictable and efficient inference,
yet is easily lost in the presence of such features.

Existing approaches preserve principality by propagating \emph{known} type
information in a rigid syntactic order (\eg bidirectional typing), often
forcing premature decisions and rejecting otherwise valid programs.
Omnidirectionaly relaxes this by allowing information to flow in a flexible,
demand-driven order.
%
However, achieving this in the presence of \ML-style
local let-generalization---as in \OCaml---is particularly difficult.

In this manuscript, we introduce the meta-theoretical machinery needed to
realise this idea and provide an \emph{omnidirectional recipe} for
instantiating the framework to specific language features.
%
We demonstrate the approach on two \OCaml features: static overloading of
record labels and datatype constructors, and semi-explicit first-class
polymorphism. Both instantiations yield a principal type inference algorithm
that is more expressive than \OCaml's current typechecker.

% No conflicts of interest
We have no conflicts of interest to disclose.

% Thanks :)
All authors have approved the manuscript and consent to its submission to
TOPLAS. Thank you for considering our work.

\closing{Sincerly,}

\end{letter}

\bibliography{suspended}

\end{document}
