%; whizzy section

%% Leave the above line for didier
%% No macros before \documentclass

\RequirePackage{boolean}
\input{\jobname.cfg}

\documentclass[acmsmall,screen,nonacm%
\Anonymous{,anonymous}{}%
\Final{}{,review}%
%% \Draft{,draft}{}% Intendedly hidden, as it destroys hyperrefs
]{acmart}

\newcommand{\acmart}{\True}
\usepackage{suspended}

%% \Xfirstname defined in {mycomments}
%% Use either
%%   \Xfistname[text to comment]{your comment on the text}
%% or
%%   \Xfirstname{free comment}
%% Uncomment this line to hide all comments.
% \UNXXX{}

\author{Alistair O'Brien}
\orcid{0009-0009-0055-7793}
\affiliation{
  \institution{University of Cambridge}
  \city{Cambridge}
  \country{United Kingdom}
}
\email{ajo41@cam.ac.uk}

\author{Didier R\'emy}
\orcid{0000-0002-0693-6278}
\email{didier.remy@inria.fr}
\affiliation{
  \city{Paris}
  \institution{INRIA}
  \country{France}
}

\author{Gabriel Scherer}
\orcid{0000-0003-1758-3938}
\email{gabriel.scherer@inria.fr}
\affiliation{
  \institution{INRIA \& IRIF, Universit\'e Paris Cit\'e}
  \city{Paris}
  \country{France}
}

\title{Omnidirectional type inference for \ML: principality any way}

\begin{document}

\begin{abstract}

% Background + Gap
  The Damas-Hindley-Milner (\ML) type system owes its success to
  \emph{principality}, the property that every well-typed expression has a
  unique most general type. This makes inference predictable and efficient.
  %
  Yet, principality is \emph{fragile}: many extensions of \ML---GADTs,
  higher-rank polymorphism, and static overloading---break it by introducing
  \emph{fragile} constructs that resist principal inference. Existing
  approaches recover principality through \emph{directional} inference
  algorithms, which propagate \emph{known} type information in a fixed (or
  \emph{static}) order (\eg as in bidirectional typing) to disambiguate such
  constructs. However, the rigidity of a \emph{static} inference order often
  causes otherwise well-typed programs to be rejected.

% Innovation

  We propose \emph{omnidirectional} type inference, where type information
  flows in a \emph{dynamic} order. Typing constraints may be solved in any
  order, suspending when progress requires known type information and resuming
  once it becomes available, using \emph{suspended match constraints}. This
  approach is straightforward for simply typed systems, but extending it to \ML
  is challenging due to \emph{let-generalization}. Existing \ML inference
  algorithms type \Let-bindings $\elet \x \ea \eb$ in a fixed order---type
  $\ea$, generalize its type, and then type $\eb$. To overcome this, we
  introduce \emph{incremental instantiation}, allowing partially solved type
  schemes containing suspended constraints to be instantiated, with a mechanism
  to incrementally update instances as the scheme is refined.
  %
  % Benefit
  %
  Omnidirectionality provides a \emph{general framework} for restoring
  principality in
  the presence of fragile features. We demonstrate its versatility on two
  fundamentally different features of \OCaml: static overloading of record
  labels and datatype constructors and semi-explicit first-class polymorphism.
  In both cases, we obtain a \emph{principal} type inference algorithm that is
  more expressive than \OCaml's current typechecker.

\end{abstract}
\maketitle

\section{Introduction}
\label{sec/introduction}

\parcomment {Introduction. What is \ML, what is principality?}

The Damas-Hindley-Milner (\ML) \citep*{Damas-Milner/W@popl82,
hindley1969principal} type system has long occupied a sweet spot in the design
space of strongly typed programming languages, as it enjoys the \emph{principal
types property}: every well-typed expression $\e$ has a most general type $\ts$
from which all other valid types for $\e$ are instances of $\ts$. For example,
the identity function $\efun \x \x$ has the principal type $\tfor \tv \tv \to
\tv$, generalizing types like $\tint \to \tint$ and $\tbool \to \tbool$.

\parcomment {Benefits of principality: practical implications}

The existence of principal types in \ML has important practical benefits. It
makes inference predictable, compositional, and efficient: since every
expression has a most general type, local typing decisions are always optimal,
with no need for guessing or backtracking. Beyond inference, principality
ensures that well-typedness is stable under common program transformations such
as let-contraction (or inlining) and argument reordering.

\parcomment {Principality is fragile. Extensions often break it}

Principality, however, is fragile. Many extensions of \ML---such as extensible
records with row-polymorphism \citep*{Remy/popl89, conf/popl/RemyV97,
conf/lics/Wand89, journals/toplas/Ohori95, garrigue1998programming} and
higher-kinded types \citep*{journals/jfp/Jones95}---are \emph{robust}: they
preserve principality. Others, including GADTs
\citep*{conf/icfp/SchrijversJSV09, conf/aplas/GarrigueR13}, higher-rank and
first-class polymorphism \citep*{conf/popl/OderskyL96,
Garrigue-Remy/poly-ml, journals/pacmpl/SerranoHJV20}, and static overloading
\citep*{Chargueraud-Bodin-Dunfield-Riboulet/jfla2025}, are \emph{fragile}: they
break principality under their \emph{natural} typing rules.

\parcomment {Example}

This fragility can already be observed in \OCaml through impredicative
higher-rank (\ie first-class) polymorphism, exposed by \emph{polymorphic
methods}~\citep{Garrigue-Remy/poly-ml} (\smashcolorbox{welltyped}{\strut green} indicates
typechecking success and \smashcolorbox{illtyped}{\strut red} indicates
failure):
%
\begin{program}[input]
let self x = x#f x °\Ocamlcomment{\ocamlFlag {\OCaml}1}°
\end{program}
%
In \OCaml, objects are defined as a collection of methods within
\code{object ... end}, and accessed using $\e \esend m$. Unlike Java or C++,
\OCaml uses \emph{structural typing} for objects: object types are a list of
method types between two chevrons \eg \ocaml[angles]{<f : 'a. 'a -> 'a>},
where the method \code{f} has the polymorphic identity function type
$\all \tv {\tv \to \tv}$ (the $\forall$ being omitted in \OCaml syntax).
%
When typing \code{self} in the example above, one could \emph{guess} the type
of \code{x} to be either $\epolymeth f \tv {\tv \to \tv}$ or $\epolymeth f \tv
{\tv \to \tv \to \tv}$---neither of which is strictly more general than the
other, violating principality.

\parcomment {Recovering principality}

Principality can be recovered through explicit type annotations. The return
type of overloaded datatype constructors may be annotated; polymorphic
expressions can be annotated with a type scheme; and for GADTs, both the type
of the \texttt{match} scrutinee and return type can be annotated with a rigid
type, which is refined by type equalities introduced in each branch.
In this example, the binding of \code{x} should be annotated with the
higher-rank type $\epolymeth f \tv {\tv \to \tv}$.

\parcomment {Fragile constructs can be viewed as an elaboration into robust ones}

Every fragile construct has a corresponding \emph{robust} form where the type
annotation is mandatory---for instance, $\eannot {\e \esend m} {} \ts$ is the
robust form of $\e \esend  m$ for polymorphic method invocation.  Robust forms
preserve principality, but at the cost of being significantly more cumbersome
to use. Fragile forms relieve this burden, but can only be elaborated into
their robust counterpart if sufficient type information is already available
from the context. Thus, typing fragile constructs ultimately amounts to
identifying when type information is \emph{known}.

\parcomment {The hard part of the problem -- defining known type information.}


Intuitively, by known information, we mean typing constraints that
must hold---either from typing rules (\eg application requires the
function to have an arrow type) or programmer supplied type annotations. Yet,
formulating a declarative specification for \emph{when} type information is
known is difficult: most specifications are often twisted with some direct or
indirect algorithmic flavor in order to preserve principality and completeness.

\parcomment {Existing approaches: all rely on a static order of inference}

The two dominant approaches thus far are \emph{bidirectional} type inference
\citep*{conf/popl/PierceT98} and \emph{\geninst-directional}
inference~\citep*{Garrigue-Remy/poly-ml}. Each impose some \emph{static}
ordering of inference, using it to propagate inferred types and user-provided
annotations as \emph{known} information.

\parcomment {Static orders have limitations}

While effective in many settings, the rigidity of a static ordering causes even
simple examples whose type could easily be inferred to be rejected. For
instance, \OCaml accepts or rejects the following expression, depending on the
position of the annotation:

\begin{program}[input,angles]
let self_1_1 (x : <f : 'a. 'a -> 'a>) = if true then x#f x else x °
\Ocamlcomment{\ocamlFlag {\OCaml}0}°
let self_1_2 x = if true then x#f x else (x : <f : 'a. 'a -> 'a>) °
\Ocamlcomment{\ocamlFlag {\OCaml}1}°
\end{program}


\parcomment {Our solution: a dynamic order}

We propose \emph{omnidirectional} type inference, which relies on a
\emph{dynamic} order of inference. The solving of inference constraints
may proceed in any order, suspending whenever progress requires \emph{known}
type information. Other constraints may continue to be solved; once the
missing information becomes available (typically via unification), the
suspended typing constraints are resumed.

\parcomment {Omnidirectionality is a framework}

We present omnidirectionality as a general framework for inference in the
presence of fragile features. While this paper instantiates our framework for
two concrete features in \OCaml---static overloading and polymorphic object
methods---its scope is broader: we expect it to extend to more richer features
such as GADTs, polymorphic parameters \citep{White/polyparams@ml2023}, and
more generalized static overloading \ala Swift.

\parcomment {Scaling to ML / let-generalization}

Because our long-term goal is to integrate omnidirectional inference into
\OCaml's typechecker, it must scale to \ML-style polymorphism. While the idea
of suspending constraints is not new (see \cref{sec:related-work}), we show how
suspended constraints can coexist with \ML \emph{local let-generalization}---an
indispensable feature of \OCaml\footnote{In contrast, \Haskell only supports
top-level implicit let-generalization.}---but one that makes suspended
constraints uniquely difficult to implement and specify declaratively.

\subsection* {Contributions}

Section \cref{sec/overview} introduces our setting: \OCaml's static overloading
of datatype constructors and record labels, and polymorphic methods. We review
directional inference, its limitations, and motivate omnidirectional inference.
To this end, we introduce our three key ideas: a new characterization of
\emph{known} type information, \emph{suspended match constraints}, and
\emph{incremental instantiation}, and demonstrate how together they enable
principal type inference for these fragile features. Before turning to
technical developments, we also discuss the \emph{limitations and trade-offs}
of omnidirectionality.

The subsequent sections present our main contributions:
\begin{enumerate}

  \item[(\cref{sec/oml})]

    The \OML calculus, an extension of \ML featuring \OCaml's static
    overloading of record labels and semi-explicit first-class polymorphism
    (\cref{sec/overview/polytypes}).
    We give typing rules with a new declarative
    characterization of \emph{known} type information.

  \item[(\cref{sec:constraints})]

    A novel constraint language for omnidirectional inference, equipped with a
    semantics for suspended constraints.
    %
    We describe the translation of \OML programs to constraints
    representing typing problems, and establish the expected metatheoretic
    properties: soundness, completeness, and principality of inference.

  \item[(\cref{sec:solving})]

    A formal definition of our constraint solver as a series of
    non-deterministic rewriting rules, proved correct with respect to
    the constraint semantics. The rewriting rules detail our treatment
    for the interaction of let-generalization with suspended
    constraints via \emph{incremental instantiation}, and the formal
    description can be directly related to an efficient
    implementation.

  \item[(\cref{sec:implementation})]

    A description of an efficient implementation of our solver, including our
    treatment of suspended constraints and partial type schemes. Validating
    that omnidirectional inference for \ML is practical.

\end{enumerate}
Finally, \cref{sec:related-work} compares related work. Section
\cref{sec:future-work} concludes with future work. Appendix
\cref{app:full-reference} contains a complete technical reference, collecting
key definitions and figures for convenient lookup. All proofs are deferred to
the appendices.

\section{Overview}
\label{sec/overview}

We ground our work in two fragile features of \OCaml: \emph{static overloading}
of record labels and constructors, and \emph{polymorphic object methods}. Both
are useful in practice: static overloading is widely relied upon in large
programs, and polymorphic methods make first-class polymorphism available
within \OCaml.

\subsection{Static overloading of constructors and record labels}
\label{sec/overview/overloading}

\XDR{We never say nor show that we do not use field types for
disambiguation, which the reader could expect}
%
\Xalistair{Here probably isn't the place to mention that. We could mention this
in limitations. It, however, isn't much of a limitation since all other
implementations of record disambiguation (\eg OCaml, Haskell) do not
disambiguate on the field type.}

\parcomment{What do we mean by static overloading?}

\emph{Static overloading} denotes a form of overloading in which resolution is
performed entirely at compile time, enabling the compiler to select a unique
implementation without relying on runtime information---in contrast to
\emph{dynamic overloading}, which defers resolution to runtime via mechanisms
such as dictionary-passing or dynamic dispatch. Many mainstream languages, such
as C++, Rust, and Java, use static overloading; its appeal is that it provides
a \emph{zero-cost} abstraction.

\parcomment{\OCaml's static overloading}

\OCaml supports a limited yet useful form of static overloading for record
labels and datatype constructors. Ambiguity is resolved using \emph{known
type information} under its directional inference algorithm (discussed in
\cref{sec/overview/directional}).
%
To illustrate static overloading in \OCaml, consider two nominal
record types with overlapping field names:
%% Fixing the wierd page break
%% Option 1: insert a page break, manually.
%% Option 2: just add a vertical penalty, manually.
%%
%% (1) looks nicer when there is a page break, but is wrong otherwise.
%% (2) is always correct but looks not so nice when there is a break.
%%
%% \begin{program}[input]
%% type point      = { x : int; y : int }
%% \end{program}
%% \pagebreak
%% \begin{program}[input=0em]
%% type gray_point = { x : int; y : int; color : int }
%% \end{program}
%%
\begin{program}[input]
type point      = { x : int; y : int }
type gray_point = { x : int; y : int; color : int } °
\vadjust {\penalty 10000}°
\end{program}
With both definitions in scope, \OCaml must statically disambiguate each
field usage:
\begin{program}[input,checkocaml]
let one = { x = 42; y = 1337 }                           °\ocamlflags 00°
let ex_1 r = r.x                                         °\ocamlflags 21°
let ex_2 (r : point) = r.x + r.y                         °\ocamlflags 10°
let ex_3 r = (r.x, (r : point).y)                        °\ocamlflags 10°
\end{program}
%
The type of expression \ocaml!one! has the unambiguous type \code{point},
even though both \code{point} and \code{gray_point} define the fields
\code{x} and \code{y}.
This is because \OCaml performs \emph{closed-world reasoning}: the
typechecker is able to unambiguously
infer the type of \ocaml{one} as \code{point}, since it is the only record type
whose domain is \ocaml!$\{$ x, y $\}$!. Similarly,
\code{r.color} necessarily infers \code{gray_point} for the type of \code{r}.

\parcomment{Examples}

By contrast, \code{r.x} is ambiguous unless the type of \code{r} is
\emph{known}.
%
In \ocaml{ex_1}, the type of \code{r} is unconstrained, so disambiguation
fails.\footnote{\let \code \code
  In fact, \OCaml does not fail on ambiguous types, but instead applies a
  default resolution strategy: it emits a warning and selects the last
  matching type definition in scope. Here, this will amount to choosing the
  type \code{gray_point} for \code{r}. To check all our examples, use the options
  \code{-principal -w +41+18 -warn-error +41+18}, which enables principal
  type inference and escalates the associated warnings to errors.
  \label{fn/principal}
}
%
%% Extraxted comment:
% Keep \code over \texttt (TeXpresso dies on \texttt in footnotes)
%% Avoid comments above that do not start at the beginning of a line as thee
%% a real risk that fill-paragraph will mess up....
%% The problem was that \string\texttt and \string\code did not have the
%% same semantics: \string\code calls ocamlsettings that selects the \small
%% font, as the program mode, and the font was oversized in footnotes.
%% I fixed this, so \code can now be freely used in footnotes.
In \code{ex_2}, the annotation fixes the type of \code{r}, thus \code{r}'s type
is \emph{known} and resolves \code{r.x} and \code{r.y} unambiguously. In
\code{ex_3}, the type of \code{r} can only be \code{point}: considering the
second projection first, we learn that \code{r} must have the type
\code{point}, and since it is $\lambda$-bound, this should make the first
projection unambiguous. However, \OCaml still rejects this example due to its
\emph{static order} of inference (\cref{sec/overview/directional}).

\paragraph{Default rules}
\label{sec/default-rules}

If local type information and closed-world reasoning are insufficient, \OCaml
falls back to a syntactic default: it selects the most recently defined
compatible type. For example, \OCaml accepts the following expression, when
\smashcolorbox{warning}{\strut warnings} are not turned into
\smashcolorbox{illtyped}{\strut errors}%
% This prevents the color boxes of `warnings` and `errors` from touching
% the code block by increasing the depth of the current line.
\lower 4pt \hbox{\strut}:%
% A manual link to the fn/principal footnote. \cref prints something that
% looks like a section reference (not a footnote).
\hyperref[fn/principal]{\textsuperscript{\ref*{fn/principal}}}
\begin{program}[input,checkocaml=true]
let getx r = r.x                                      °\ocamlflags 21°
\end{program}
%
The expression is compatible with both \code{point} and \code{gray_point},
since each defines a field \code{x}. But \code{gray_point} is chosen simply
because it appears later in the source.

\parcomment {Default rules are inherently non-principal}

Such fallback behaviour is inherently \emph{non-principal}: it reflects the
typechecker's decision to abandon principal inference and arbitrarily select a
syntactic default when no unique type can be inferred. We therefore give no
formal account of such ``default rules''.

\parcomment{Default rules interact poorly with directional inference}

Defaulting also interacts poorly with \OCaml's directional inference. Once
the compiler selects a type, it commits to it---even if that choice causes
errors downstream. Consider:
\begin{program}[input,checkocaml=true]
let ex_4 r = let x = r.x in x + (r : point).y    °\ocamlflags 10°
\end{program}
Here, \OCaml defaults to \code{gray_point} for \code{r} when typing \code{r.x},
and subsequently fails on \code{(r : point).y}. \OML succeeds by suspending the
resolution of \code{r.x} until it learns from \code{(r : point).y} that \code{r}
has type \code{point}.

\parcomment{Variants are supported, but not discussed}

Since overloaded datatype constructors are analogous to record fields, we focus
only on record fields in this work. Our prototype implementation
(\cref{sec:implementation}), however, supports both.


\subsection{Polymorphic methods}

\parcomment{Why polymorphic methods?}

Polymorphic methods \citep*{Garrigue-Remy/poly-ml} bring some System-$\F$-like
expressiveness to \OCaml by allowing first-class polymorphism (impredicative
higher-rank polymorphism) while preserving principal type inference.

%% \begin{mathparfig}[htpb!]%
%%   {fig:polyml}%
%%   {Syntax and typing rules for polytypes from
%%   \citet*{Garrigue-Remy/poly-ml}.}

%% \par
%% \end{mathparfig}

\paragraph{From polymorphic methods to polytypes}
\label{sec/overview/polymethods-reduction}

\parcomment{Idea and syntax}

Polymorphic methods can be translated into ordinary methods that carry a
\emph{polytype}: a boxed type scheme $\tapoly \ts \av$ that
can be explicitly unboxed at use sites.
%
The purpose of the annotation variable $\av$ will be explained
shortly. Boxed polytypes are considered to be (mono)types, enabling
impredicativity. We write $\epoly[\ts] \e$ to box a term $\e$ with the
scheme $\ts$, and $\einst \e$ to unbox a polytype, instantiating
it.

\begin{mathparfig}[tb]
  {fig:polyml}
  {Syntax and typing rules for polytypes from
   \PolyML~\citep*{Garrigue-Remy/poly-ml}.}
%% No \par here
\begin{bnfgrammar}
     \entry[Terms]{\e}{
       \epoly[\ts] \e
       \and \einst \e
       \and \eannot \e {} \t
       \and \dots
     }\\
     \entry[Types]{\t}{
        \tapoly \ts \av
        \and \dots
     }\\
     \entry[Type schemes]{\ts}{
       \t \and \tfor \tv \ts \and \tfor \av \ts
     }\\
    \entry[Annotation variables]{\av}{
     }\\
\end{bnfgrammar}

  \infer[PolyML-Poly]
    { \G \th \e : \tsa \\ (\tsa : \ts : \tsb) }
    {\G \th \epoly[\ts] \e : \tapoly {\tsb} \av}

  \infer[PolyML-Inst]
    {\G \th \e : \tfor \av {\tapoly \ts \av}}
    {\G \th \einst \e : \ts}

  \infer[PolyML-Annot]
    {\G \th \e : \ta \\ (\ta : \t : \tb)}
    {\G \th \eannot \e {} \t : \tb}
%% No \par here
\end{mathparfig}

\parcomment{Example}

Concretely, the polymorphic method of
%
\ocaml{object method id : 'a. 'a -> 'a = fun x -> x end}
%
is translated to \ocaml{object method id = [ fun x -> x : 'a. 'a -> 'a ] end}.
Method invocation implicitly unboxes the polytype \eg $\ttlab \x \esend
\ttlab {id}$ becomesn $\einst {\ttlab \x \esend \ttlab {id}}$.

\parcomment{Why bother?}

This reduction is useful for two reasons:
\begin{enumerate*}
  \item Inference for \OCaml's object layer is largely governed by row-polymorphism,
    which is \emph{robust} and does not threaten principality; it is therefore
    orthogonal to our concerns. In contrast, polytypes are \emph{fragile}.


  \item Polytypes underpin other features in \OCaml, notably the recent
    addition of polymorphic function parameters \citep*{polyparams}.

\end{enumerate*}

\paragraph{Semi-explicit first-class polymorphism}
\label{sec/overview/polytypes}

Polytypes expose the tricky interaction with principality of interest. For the
remainder of this work, we therefore focus on polytypes\penalty 50---also
called \emph{semi-explicit first-class polymorphism}
\citep*{Garrigue-Remy/poly-ml}---as originally formulated in the \PolyML
calculus, whose typing rules are collected in \cref{fig:polyml}.

\parcomment {Annotation variables. Why?}

Annotation variables record the origins of polytypes and may themselves be
generalized, yielding type schemes such as $\tfor \av {\tapoly \ts \av}$. When
$\av$ is generalized, the polytype is considered \emph{known}, rather than
still being inferred---this distinction is precisely the purpose of annotation
variables.

\parcomment {Boxing}

The introduction form (\Rule{PolyML-Poly}) for polytypes is a boxing operator
$\epoly[\ts] \e$ with an explicit polytype annotation $\ts$.
%
The resulting expression has type $\tapoly {\tsb} \av$ where $\av$ is an
arbitrary (typically fresh) annotation variable and $\tsb$ is a \emph{freshened
copy} of $\tsa$, \ie a variant of $\tsa$ with only the annotation variables
of $\ts$ renamed (see \Rule{PolyML-Annot} below for details). Because
$\ts$ is supplied by the programmer, the polytype is treated as known:
$\epoly[\ts] \e$ also has the generalized type scheme $\all \av {\tapoly {\tsb}
\av}$. This is by design---the explicit annotation in $\epoly[\ts] \e$ records
that the polytype is known.


\parcomment {Unboxing (principality restriction and \geninst-directionality)}

Conversely, to instantiate a polytype expression (\Rule{PolyML-Inst}), one must
use an explicit unboxing operator $\einst \e$, which requires no accompanying
type annotation. However, the operator requires $\e$ to have a \emph{known} polytype scheme
of the form $\tfor \av {\tapoly \ts \av}$ and then assigns $\einst \e$ the type
$\ts$. If, by contrast, $\e$ has the type $\tapoly \ts \av$ for some
non-generalizable annotation variable $\av$, then $\e$ is considered of a
\emph{not-yet-known} polytype, and therefore $\einst \e$ is ill-typed. This
restriction enforces principality, preventing instantiation on \emph{guessed}
polytypes.

\parcomment {Example ill-typed term}

For example, the expression $\efun \x {\einst \x}$ is not typable. Indeed, the
$\lambda$-bound variable $\x$ is assigned a monotype. The only admissible type
for $\x$ is $x : \tapoly \ts \av$ for some $\ts$ and $\av$.  Since $\av$
appears in the type of $\x$ in the typing context at the point of typing
$\einst \x$, it cannot be generalized prior to unboxing, rendering the term
ill-typed.

\parcomment {Annotations}

\Rule{PolyML-Annot} can be used to freshen annotation variables. The auxiliary
relation $(\tsa : \ts : \tsb)$ (also used in \Rule{PolyML-Poly}) holds if there
exists renamings $\eta_1, \eta_2$ on annotation variables (leaving ordinary
type variables unchanged) and a type substitution $\theta$ such that $\tsa =
\theta(\eta_1(\ts))$ and $\tsb = \theta(\eta_2(\ts))$.
%
Intuitively, $(\tsa : \ts : \tsb)$ first produces two \emph{fresh copies} of
$\ts$, preventing unwanted sharing of annotation variables that could otherwise
block generalization. It then instantiates the corresponding free type
variables between these two copies using $\theta$.  We usually omit annotation
variables in source type annotations, since we can implicitly introduce fresh
ones in their place.

For example, $\efun {\x : \tapoly \ts {}}
{\einst \x}$, which is syntactic sugar for $\efun \x {\elet \x {(\x : \tapoly
\ts {})} {\einst \x}}$, is well-typed because the explicit annotation
introduces a fresh variable annotation $\ava$, which can then be generalized,
yielding $\tfor \ava {\tapoly \ts \ava}$ for the type of the let-bound
variable $\x$.

\subsection{Directional type inference}
\label{sec/overview/directional}

We now discuss the two main directional inference approaches:
\geninst-directional and bidirectional, illustrated using polytypes as a
running example. We then discuss limitations of both approaches, providing us
with the motivation for omnidirectional type inference.

\paragraph{\Geninst-directional type inference}

\parcomment{\ML (and \geninst) have a fixed order}

Most \ML type inference algorithms proceed in a fixed order when typechecking
let-bindings $\elet \x \ea \eb$: first typecheck the definition $\ea$,
generalize its type, and then typecheck the body $\eb$ under the extended
environment. \Geninst-directionality, originating from \PolyML
(\cref{sec/overview/polytypes}), leverages this ordering to resolve overloaded
or ambiguous constructs in a \emph{principal} way. The key point is that this
ordering reveals which types are \emph{known}, namely types that are fixed by
generalization---and therefore stable enough to guide disambiguation.

\parcomment{What is a known type?}

A type $\t$ (\eg $\tapoly \ts \av$) is considered \emph{known} if its
annotation variables are eligible for generalization (\eg $\tfor \av {\tapoly \ts
\av}$). Conversely, types with monomorphic annotation variables are
considered \emph{not-yet-known} and cannot be relied on for
disambiguation.

\parcomment{Naming}

We call this \geninst-directional (read as ``\textbf{pi}-directional'') type
inference, to mean that \textbf{p}olymorphic expressions must be typed before
their \textbf{i}nstances. \Geninst-directionality is subtle, but it aligns with
the implicit inference order already present in most \ML-like typecheckers,
making it straightforward to retrofit into existing implementations. For
\OCaml, the mechanism for annotation variables even comes \emph{for free}, as
a byproduct of the extensive optimizations in its inference algorithm.

\parcomment{Who uses this?}

Building on its introduction in \PolyML, \geninst-directionality was later used
by~\citet*{LeBotlan-Remy/recasting-mlf} for empowering \penalty1 \MLF.
%
It has since been adopted in \OCaml for features such as polymorphic object
methods and the overloading of record fields and variant constructors. More
generally, \OCaml uses \geninst-directionality whenever the typechecker employs
type-based disambiguation.

\parcomment{Good example}

To illustrate \geninst-directionality, consider:%
%
\footnote{
% Some space hacking since it appears spaces aren't respected in lstinline w/
% mathescape
In \OCaml, these examples can be typechecked by translating
$\epoly[\ts] \e$ to
\Code{object method f} $: \ts = \e$ \Code{end}
and $\einst \e$ to $\e \esend \texttt{f}$. This translation is the inverse of
that discussed in \cref{sec/overview/polymethods-reduction}.
}
\begin{program}[input,angles,checkpolyml]
let pid = [ fun x -> x : 'a. 'a -> 'a ] °\ocamlflags 00°
let ex_5 = let p = pid in <p> °\ocamlflags 00°
let ex_6 = (fun p -> <p>) pid °\ocamlflags 10°
\end{program}
At first glance, \code{ex_5} and \code{ex_6} appear equivalent: both simply
instantiate the polytype bound to \code{p}. Yet, \OCaml accepts \code{ex_5} and
rejects \code{ex_6}. This is because the \Let-binding in \code{ex_5}
allows \code{p} to have type scheme $\tfor \av {\tapoly {\tfor \tv \tv \to
\tv} \av}$, and thus its type is considered \emph{known}---permitting unboxing
(\Rule{PolyML-Inst}).
In \code{ex_6}, by contrast, \code{p} is monomorphic at the point of
instantiation as it is $\lambda$-bound, and unboxing is therefore
forbidden.

To emphasize that this behavior is specification-driven and not an artifact of
\OCaml's inference algorithm, consider two equivalent versions of
\ocaml{ex_6}:\footnote {\let \code \Code
\code{app} and \code{rev_app} are the application
function \Code{fun f x -> f x} and the reverse application function
\Code{fun x f -> f x}, respectively.}
\begin{program}[input,angles,checkpolyml]
let ex_6_2 = app (fun p -> <p>) pid                        °\ocamlflags 10°
let ex_6_3 = rev_app pid (fun p -> <p>)                    °\ocamlflags 20°
\end{program}
While these terms are semantically equivalent, they highlight a potential
hazard: their typability may vary under a directionally biased inference
algorithm, depending on whether the function or argument is typed first.  To
limit such implementation-dependent behavior, \OCaml infers all
subexpressions in an \emph{order-independent} manner until they are \Let-bound.
Consequently, \OCaml does not make any distinction between
\ocaml[indices]{ex_6}, \ocaml[indices]{ex_6_2}, and \ocaml[indices]{ex_6_3}.

\parcomment{Short comings of pi-directionality}

Treating both examples uniformly is in one sense a strength of
\geninst-directionality, but it also reveals a limitation:
annotability is fragile, in that well-typedness depends on the
\emph{precise} placement of annotations, often forcing the programmer
to introduce annotations that would otherwise be unnecessary.
For instance, the following two terms differ only in the position of
the annotation, yet only the one on the left-hand side is well-typed in
\OCaml---while they are both well-typed in \OML:
\begin{program}[input,angles,checkpolyml]
let self_2_1 x = <(x : [ 'a. 'a -> 'a ])> x °\ocamlflags 00°
let self_2_2 x = <x> (x : [ 'a. 'a -> 'a ]) °\ocamlflags 10°
\end{program}

\paragraph{Bidirectional type inference}

Bidirectional type inference is a standard alternative to unification for
propagating type information. It is typically formulated by splitting typing
rules into two modes: \emph{checking mode} ($\G \th \e \Leftarrow \t$), which
typechecks a term $\e$ against a type $\t$ in a given context, and
\emph{inference mode} which infers $\e$'s type from the context alone ($\G \th \e
\Rightarrow \t$).

\parcomment{Assignment of modes}

The type system designer assigns modes---checking or inference---to each
language construct. For instance, one can decide to typecheck function
applications $\eapp \ea \eb$ by first \emph{inferring} that $\ea$ has some
function type $\ta \tarrow \tb$, and then \emph{checking} $\eb$ against $\ta$ (\Rule{Syn-App});
but the opposite, mode-correct choice (\Rule{Chk-App}) is also possible:
\begin{mathpar}
  \infer[Syn-App]
    {\G \th \ea \Rightarrow \ta \to \tb \\ \G \th \eb \Leftarrow \ta}
    {\G \th \eapp \ea \eb \Rightarrow \tb}

  \infer[Chk-App]
    {\G \th \ea \Leftarrow \ta \to \tb \\ \G \th \eb \Rightarrow \ta}
    {\G \th \eapp \ea \eb \Leftarrow \tb}
\end{mathpar}

\parcomment{Known type information}

Within the bidirectional framework, a type $\t$ is \emph{known} when it is
either:
\begin{enumerate*}
  \item part of an annotation,
  \item supplied as input to a checking judgment in the conclusion
  ($\G \th \e \Leftarrow \t$), or
  \item produced by a synthesizing premise ($\G \th \e \Rightarrow \t$)
\end{enumerate*}.

Using this discipline, we can recast polytypes and eliminate two artifacts
needed in our \geninst-directional presentation: explicit annotations on boxing
and annotation variables. Since $\eannot \e {} \t$ already propagates known
information, $\epoly \e$ requires no attached annotation; and because
``knownness'' now follows from inference modes rather than polymorphism,
annotation variables are unnecessary. The resulting syntax and typing rules are
as follows:
\begin{mathpar}
  \begin{minipage}[l]{15em}
  %% \scalebox{0.9}{
  \begin{bnfgrammar}
    \noalign{\vskip -1em}
    \entry[Terms]{\e}{
       \epoly \e
       \and \einst \e
       \and \eannot \e {} \t
       \and \dots
     }\\
     \entry[Types]{\t}{
        \tpoly \ts
        \and \dots
     }\\
     \entry[\llap{Type schemes}]{\ts}{
       \t \and \tfor \tv \ts
     }\\
  \end{bnfgrammar}
  %% }
  \end{minipage}
  \hfill
  \infer[Chk-Poly]
    {\G \th \e \Leftarrow \ts}
    {\G \th \epoly \e \Leftarrow \tpoly \ts}

  \infer[Syn-Inst]
    {\G \th \e \Rightarrow \tpoly \ts}
    {\G \th \einst \e \Rightarrow \ts}

  \infer[Syn-Annot]
    {\G \th \e \Leftarrow \t}
    {\G \th \eannot \e {} \t \Rightarrow \t}
\end{mathpar}

\parcomment{Limitations}

However, there is usually no optimal assignment of modes: for any choice of
modes, some programs will typecheck successfully, while others will fail
unnecessarily. Yet, the typing rules must irrevocably commit to a fixed set of
modes, after which, principal types often exist, but only with respect to a
specification that made non-principal choices to begin with. For instance,
\code{ex_6} would be ill-typed using the above rules for polytypes (with
\Rule{Syn-App}).

\parcomment{Contextual typing}

Recent work on \emph{contextual typing} \citep{journals/pacmpl/XueO24}
addresses this difficulty by deferring the commitment between \Rule{Syn-App}
and \Rule{Chk-App}. Typing multiple arguments from right to left
(\Rule{Chk-App}) when sufficient contextual information is available, and thus
successfully typechecks \code{ex_6}. Nevertheless, it still enforces a fixed
order of propagation, so some well-typed programs are rejected as ill-typed
(\eg \code{ex_6_2, ex_6_3}).

% Contextual typing stuff
%
% See typing rules in Figure 5 on pg 11 of
% https://dl.acm.org/doi/pdf/10.1145/3674655
%
% Here is a brief argument of why contextual typing fails here: Take ex_62 for
% instance (everything inlined to avoid talking about implicit polymorphism):
%
% ```ocaml
%  (fun f x -> f x) (fun p -> < p >) [ fun x -> x : 'a. 'a -> 'a ]
% ```
%
% In order to type check `fun f x -> f x`, we must type check it with quantity
% 2 (S S 0) since it has two unknown parameters (according to DLam on pg 11).
% Thus to typecheck the application `(fun f x -> f x) (fun p -> < p >)` we need
% to use DApp2 with quantity 1 (S 0) for the conclusion. But this requires us
% to type `fun p -> < p >` with quantity 0! This is not possible (n- is
% undefined for n = 0).
%
% Similar argument for ex_63: `(fun x f -> f x) [ fun x -> x : 'a. 'a -> 'a ]`
% is typable with quantity 1. But the application of `(fun p -> < p >)` would
% (yet again) need to be typable with quantity 0 (in order to apply DApp2).
% This is not possible.

\parcomment{QuickLook and Freezing Bidirectional}

In parallel, \QuickLook \citep{journals/pacmpl/SerranoHJV20} and \Frost
\citep{frost} take a pragmatic approach to mitigating order-dependence in
applications. Both target impredicative higher-rank polymorphism, using partial
approximations of function and argument types to guide type inference at the
point of application. This enables a limited form of \emph{argument
reordering}, where information from later arguments can influence the typing of
earlier ones within an application spine. In practice, these mechanisms work
well, but they still enforce a fixed overall direction of propagation between
surrounding constructs (\eg let-bindings).

\paragraph{Limitations of directional type inference}

Bidirectional type inference is lightweight, practical, and well-suited for
complex language features such as higher-rank polymorphism, dependent types, or
subtyping. It supports the propagation of type information with minimal
annotations. Its main downside lies in the need to fix an often arbitrary flow
of type information---as in the case of function applications discussed above.

On the other hand, \emph{\geninst-directional} type inference appears better
suited for \ML: %
\begin{enumerate*}
    \item thanks to its use of polymorphism\penalty 50---the essence of \ML; and
    \item its ability to be retrofitted easily onto existing typecheckers.
\end{enumerate*}
But it remains surprisingly weak in some cases: it does not even allow the
propagation of user-provided type annotations from a function to its argument!
This weakness is sometimes counter-intuitive to the user. For example, the
following would be rejected as ambiguous using \geninst-directional type
inference alone:
\begin{program}[input,angles,checkpolyml]
let ex_7 = °\ocamlflags 00°
  let g (f : [ 'a. 'a -> 'a ] -> int) = f pid in
  g (fun p -> <p> 42)
\end{program}
Here, \code{p} is $\lambda$-bound and therefore monomorphic. Without further
propagation, the term \ocaml[angles]{<p>} would be ambiguous, as no polymorphic (and
thus \emph{known}) type can be ascribed to \code{p}. \OCaml resolves this by
supplementing \geninst-directional inference with a form of bidirectional
propagation: the expected type of \code{g}'s parameter (\code{[ 'a. 'a -> 'a ] -> int}) is
bidirectionally propagated to the application of \code{g}, assigning
\code{p} to have the \emph{known} type \code{[ 'a. 'a -> 'a ]} and thereby
disambiguating \ocaml[angles]{<p>}.

\subsection{Omnidirectional type inference}
\label{sec/overview/omni}

% There are really 3 core ideas of our paper that we try to get across in this
% section:
%
%  1. A characterization of 'known' type information that relies on a _dynamic_
%  order
%
%  2. Suspended match constraints -- suspension is the primitive used to wait
%  for known type information when we cannot progres
%
%  3. ML generalization previously had a fixed order of solving, this doesn't
%  work for omnidirectionality. The fix: incremental instantiation / partial
%  generalization.


\parcomment{Idea: inference has a dynamic order}

Omnidirectional inference infers typing constraints in any order. Constraints
advance \emph{dynamically}; those that require \emph{known} type information
suspend, and resume when other constraints supply it. This stands in contrast
to the fixed \emph{static} order of bidirectional and \geninst-directional
inference.

\parcomment{Example: dynamic order types more programs}
%
Consider again \code{ex_6_2, ex_6_3} from \cref{sec/overview/directional}:
\begin{program}[input,angles]
let ex_6_2 = app (fun p -> <p>) pid  °\Ocamlcomment{\ocamlFlag {\OML}0}°
let ex_6_3 = rev_app pid (fun p -> <p>) °\Ocamlcomment{\ocamlFlag {\OML}0}°
\end{program}
%
Under \geninst-directionality, both terms are ill-typed; under a bidirectional
approach (using \Rule{Syn-App}), only \code{ex_6_2} is rejected. Yet both terms
have a principal type---it is merely a question of propagating type information
in the right order. Omnidirectional type inference typechecks both: since it
allows the typing of either side to proceed first, suspending the other until
the relevant type information is \emph{known}.

\paragraph{To be or not to be known}
\label{sec/overview/omni/unicity}

\parcomment{Idea: known = contextual uniqueness}

That is the question, indeed. To specify omnidirectionality declaratively, we
must say when a type is considered \emph{known} without relying on any fixed
directional order. Our key idea is that a type is \emph{known} when it is the
\emph{unique} type that can be inferred within some surrounding term context
$\E$.

\parcomment{Example: ex_62}

Take \code{ex_6_2} as an example, using the following term context where
$\hole$ denotes the hole:
%
\begin{program}[input, mathescape=true]
let pid = [ fun x -> x : 'a. 'a -> 'a ]
let ex_6_2 = app (fun p -> $\hole$) pid
\end{program}
%
Here, \code{p} has a uniquely inferrable type $\tpoly {\tfor \tv
{\tv \to \tv}}$, no other type can be inferred for \code{p}. As a result, we
can consider \code{p}'s type \emph{known}.

\parcomment{Elaboration}

\begin{wraphbox}{}{}
\begin{mathpar}[inline]
  \infer[Use-I]
    {\eshape \E \e {\tpoly \ts} \\ \G \th \E\where{\exinst \e {} \ts} : \t}
    {\G \th \E\where{\einst \e} : \t}
\end{mathpar}
\end{wraphbox}
Once the type is known, the fragile implicit term can be elaborated to a
robust explicit counterpart. For example, the unboxing \ocaml[angles]{<p>} can
be elaborated into the explicitly annotated form \ocaml[angles]{<p : 'a. 'a -> 'a>}.
%
Consequently, to typecheck $\einst \e$ under the context $\E$, it suffices to
assert that $\e$ has the known polytype $\tpoly \ts$ and elaborate
$\einst \e$ into an annotated unboxing $\exinst \e {} \ts$, as captured by
\Rule{Use-I}.
%
The predicate $\eshape \E \e {\tpoly \ts}$---our \emph{unicity
condition}---formalizes precisely what it means for a type to be \emph{known}.
We defer its technical definition, which is rather subtle, to \cref{sec/oml/typing/I}.

\paragraph{Suspension in action}
\label{sec/overview/omni/match}

\parcomment{Suspended constraints}

Suspension is the mechanism that allows inference to proceed in any order,
in spite of constructs that require \emph{known} type information.  In our
framework, we realize this through our novel primitive: \emph{suspended
match constraints}.

\parcomment{Syntax + informal semantics}

A match constraint $(\cmatch \t {{\overline{\cbranch \cpat \c}}})$ pairs a
(typically unknown) matchee type $\t$ with a finite series of shape-pattern
branches $\overline{\cbranch \cpat \c}$. Such constraints remain
\emph{suspended} until the \textit{shape} of $\t$ (\ie its top-level
constructor) is known. Then, they are \emph{discharged}: a unique branch is
selected and its associated constraint has to be solved. A match constraint
that is never discharged is considered unsatisfiable.

\parcomment{A note on shape patterns}

For now, it suffices to think of shapes as parts of types (\eg the type
constructor $\T$ in $\Fapp[\T] \tys$%
\footnote{
  Type constructors are prefixed, except in \OCaml code, where they are postfixed.
})
while shape patterns $\cpat$ act as type `destructors', binding parts of the type
(\eg the constructor name $\T$) to meta-variables (\eg the pattern variable
$\ct$) used in $\cs$. This will be made precise in
\cref{sec/constraints/shapes}.

\parcomment{Why our solution makes things easier?}

We now illustrate the role of suspended constraints on our running
\emph{fragile} features: static overloading of records (and variants) and
semi-explicit first-class polymorphism.
%
Each feature translates the typability of the term into constraints, formalized
using a constraint generation function of the form
$\cinfer \e \t$, which,
given a term $\e$ and expected type $\t$, produces a constraint $\c$ which is
satisfiable if and only if $\e$ has the type $\t$.
%
As we will see, once we adopt the suspended constraint machinery developed in
this paper, much of the complexity of these typing fragile constructs
vanishes---suspended constraints do most of the heavy lifting.

\parcomment{Records}

For an ambiguous record projection $\efield \e \elab$,
we generate the typing constraint:
\begin{mathpar}
\cinfer {\efield \e \elab} \t \wide\eqdef
  \cexists \tv \cinfer \e \tv
  \cand
  \cmatch \tv
      {{\cbranch {\cpatrcd \ct}
  {{\labfrom \elab \ct \leq (\tlab \tv \t)}}}
      }
\end{mathpar}

This constraint introduces the unification variable $\tv$, unifying it
with the type of $\e$ (via $\cinfer \e \tv$), and suspends resolution
of the return type $\t$ until the type $\tv$ of $\e$ becomes
\emph{known} to be some non-variable type $\tz$. The branch then
matches $\tz$ against the record type pattern $(\cpatrcd \ct)$. If $\tz$
is a record type $(\trcd {\T} \tyas)$ for some ground record name ${\T}$,
then the record name variable $\ct$ in the pattern is bound to $\T$,
otherwise the whole constraint fails. When $\ct$ is bound to $\T$, the
right-hand-side constraint becomes
$\labfrom \elab \T \leq (\tlab \tv \t)$, which requires that the type
of the projection of the label $\elab$ at type $\T$ in the global
record-declaration environment can be instantiated into
$(\tlab \tv \t)$.

\parcomment{Polytypes}

When typechecking the polytype unboxing operator $\einst \e$, if $\e$ is
already known to have the type $\tpoly \ts$, then we can simply
instantiate $\ts$.  However, if the type of $\e$ is not yet known---\ie  it is a
(possibly constrained) type variable $\tv$---then we must defer until more
information is available. We capture this behavior with a suspended match
constraint:
\begin{mathpar}
\cinfer {\einst \e} \t \Wide\eqdef
    \cexists \tv \cinfer \e \tv
\cand
    \cmatch  \tv \cbranch {\tpoly \cscm} \cscm \leq \t
\end{mathpar}
%
The match remains suspended until $\tv$ resolves to some type $\tz$. If, upon
resolution, $\tz$ is $\tpoly \ts$, the pattern $\tpoly \cscm$ matches
successfully, binding $\ts$ to the polytype variable $\cscm$ and performs the
instantiation $\cleq \ts \t$. Otherwise, the
pattern does not match and the constraint fails.

\paragraph{Scaling to \ML}

\parcomment{It's easy without polymorphism}

In the absence of (implicit) polymorphism, type inference is solely based on
unification constraints which can be solved in any order; omnidirectional
inference with suspended match constraints is then natural and easy to
implement.

\parcomment{\ML generalization has a fixed order}

The difficulty originates from \ML \emph{implicit} \texttt{let}-polymorphism for which
all known implementations follow the \geninst-order: first typing the binding,
generalizing it into a type scheme, and finally typing the body under the
extended typing environment that binds the generalized scheme. The
Hindley-Milner algorithm $\mathcal{J}$, its variants $\mathcal{W}$ or
$\mathcal{M}$~\citep* {Lee_Yi/algoM@toplas1998}, or more flexible
constraint-based type inference implementations~\citep*
{Remy/mleth,Remy/thesis, Odersky-Sulzmann-Wehr@tpos, Pottier-Remy/emlti} all
follow this strategy, to the best of our knowledge.

\parcomment{Example of why this problematic}

Consider the following program:
\begin{program}[input]
type 'a gpoint = { x : 'a; y : 'a }
let diag (n : 'a) : 'a gpoint = { x = n; y = n }
°\halfline°
let ex_8 gp = let getx p = p.x in getx (diag 42), (getx gp : float) °\ocamlflags 20°
\end{program}
We introduce a new parameterized record type \code{'a gpoint}, whose
fields \code{x} and \code{y} are overloaded---recall that both \code{point} and
\code{gray_point} already define these fields---but here they have a
\emph{polymorphic} projection type $\tfor \tv {}$ \code{'a gpoint -> 'a}. The
function \code{diag} constructs a diagonal point, a point lying on the
diagonal $x = y$, and has the type $\tfor \tv {}$ \code{'a -> 'a gpoint}.

When typechecking \code{ex_8}, we cannot infer the type of \code{getx} first,
since the type of \code{p.x} is still not yet known. Instead, we must typecheck
the body first, where the call \code{getx (diag 42)} reveals that \code{p} has
type \code{'a gpoint}.
%
Failing to do so would make inference incomplete, since the program is clearly
well-typed. Nor can we treat the \Let-binding as monomorphic, since both calls
to \code{getx} use different instantiations of \code{'a}: $\tv \is \tint$ in \code{getx (diag
42)} and $\tv \is \tfloat$ in \code{(getx gp : float)}.

\parcomment{Our solution}

We solve this by introducing \emph{incremental instantiation}, \ie the ability
to instantiate type schemes that are not yet fully determined (so-called
\emph{partial type schemes}) and consequently revisit their instances when they
are being refined, \emph{incrementally}. This allows inferring parts of a
\Let-body to disambiguate its definition, without duplicating
constraint-solving work.

% Something to tie everything back together / also serve as a lookup

\paragraph {The forest, not the trees.}

% Suspended constraints are a framework
Suspended match constraints offer a \emph{general framework} to typing the
features we have considered so far, and more. Some of those features can be
handled using more specialized approaches: for example, \SML employs row
variables to support overloaded fields for structural records, while \GHC uses
qualified types to allow overloading of nominal record fields with a
simple-enough type.

% Suspended constraints are more powerful than the specialized mechanisms

In contrast to these specialized mechanisms, suspended constraints are more
expressive. They can handle cases where the typing rule to use on the subterms
depends on the outcome of disambiguation, such as overloaded polymorphic record
fields or overloaded GADT constructors in patterns.
%
Moreover, these simpler approaches typically lack a declarative semantics that
justify rejecting programs with unresolved disambiguation choices.

\subsection{Limitations}
\label{sec/overview/limitations}

% 1. Our framework is complex --- as a result, everything is hard ://
% 2. Some programs are ill-typed even though they have a principal type.
% 3. We do not provide a formalization of default rules.

\parcomment{Some programs even though they have a principal type are ill-typed}

Omnidirectional inference is powerful, but not omnipotent:
\begin{enumerate*}
  \item some programs are still rejected as ambiguous, even though a unique
  elaboration could
    in principle be chosen---a deliberate ``Goldilocks'' compromise;
  \item the current formalization omits a treatment of default rules; and
  \item the conceptual and implementation complexity of our framework.
\end{enumerate*}

\paragraph{Not too hot, not too cold} Some expressions
must be rejected even though their elaboration would be unambiguous. Consider:

\begin{program}[input,checkocaml]
type cie_color = { x : int; y : int; z : int }
type cie_point = { x : int; y : int; color : cie_color }
let ex_1_0 r = r.color.x °\ocamlflags 11°
\end{program}
Neither field projections in \code{ex_1_0} can individually be
disambiguated. However, if one were allowed to combine the constraints, they
would jointly determine that \code{r} must have type \code{cie_point}: in
\code{gray_point}, the field \code{color} has type \code{int}, and hence
cannot itself be projected, leaving a unique consistent elaboration:
\code{r.cie_point/color.cie_color/x}.

Our framework nonetheless rejects \code{ex_1_0}. This is intentional:
overloaded projections must be elaborated \emph{sequentially}, each in isolation,
rather than jointly with others. We view this restriction as a
``Goldilocks'' compromise: it rules out examples like the above, but avoids
the intractability of full general overloading which is NP-hard, even
without let-polymorphism, as shown by a reduction from 3-SAT~\citep*
{Chargueraud-Bodin-Dunfield-Riboulet/jfla2025}.


\paragraph{No defaults, by default}

\parcomment{No default rules formalism}

We do not yet provide a formal account of \emph{default rules} mentioned in
\cref{sec/default-rules}.  However, our prototype implementation, discussed in
\cref{sec:implementation}, does support (optionally) attaching a default
strategy to each suspended constraint.  \Draft{More details can be found in
Appendix \cref {app/default-rules}.}{} In practice, defaulting proves useful
and appears essential for certain features such as polymorphic parameters
\citep{White/polyparams@ml2023}.
% Alistair: I don't think we should say anything about defaulting beside we don't
% know formally how to do it, we can do it in practice, we think it is important.
% We shouldn't make any claims about being better than OCaml (especially since we
% don't substantiate this claim)
%
% Defaulting behaves better than in OCaml: we propagate more type information,
% so we default less often and never in situations which are obviously incompatible
% with the program context. But it still breaks principality.
Developing a formal treatment of defaulting within our framework is an
important direction for future work.

\paragraph{On complexity budgets}
% All this goodness has a cost

Omnidirectional type inference is conceptually straightforward but technically
challenging. It follows a simple key idea: solving constraints in any order,
suspending when known type information is required. However, realizing this
idea precisely and efficiently comes at a higher complexity cost than
bidirectional or $\Geninst$-directional type inference.

\begin{enumerate}
\item[(\cref{sec/oml})]
  Giving a declarative characterization of \emph{known} type information
  without statically relying on directionality is hard. Our contextual rules
  nicely solve this problem, but their meta-theory is unsurprisingly more
  complex than local rules.

\item[(\cref{sec:implementation})]
  Implementing incremental instantiation efficiently is also tricky: it
  requires triggering re-instantiations upon refinements of partial type
  schemes, while avoiding redundant constraint solving across
  instantiations.
\end{enumerate}
We hope to pay off some of this complexity in future work; in particular, we
believe omnidirectionality is the missing piece to unlock modular
implicits---an approach to generalized overloading and a long-anticipated
feature within the \OCaml community.


\section{The \OML calculus}
\label{sec/oml}

\begin{mathparfig}[t]
  {fig/oml/syntax-xtyping}
  {Syntax and explicit, robust typing rules of \OML.}
  \begin{bnfgrammar}
  \entryset[Type variables]{\tva, \tvb, \tvc}{\TyVars}{}
  \\
  \entry[Types]{\t}{
      \tv \and
      \tunit \and
      \ta \to \tb \and
      \trcd \T \tys \and
      \tpoly \ts
  }
  \\
  \entry[Type schemes]{\ts}{
      \t \and
      \all \tv \ts
  }
  \\
  \entry[Record name]{\T}{}{}
  \\
  \entryset[Ground types]{\gt}{\Ground}
  \\[1ex]
  \entry[Terms]{\e}{
    x \and
    () \and
    \efun x e \and
    \eapp \ea \eb \and
    \elet x \ea \eb \and
    \eannot \e \tvs \t \andcr
     \epoly e \and
     \expoly e \tvs \ts \and
     \einst e \and
     \exinst e \tvs \ts \andcr
    \erecord {\overline{\elab = \e} } \and
    \efield e \elab \and
    \exrecord \T {\overline{\elab = \e}} \and
    \exfield e \T \elab
     }
  \\[1ex]
  \entry[Contexts]{\G}{
     \eset \and
     \G, x : \ts
  }
  \\
  \entry[Label contexts]{\labenv}{
    \eset \and
    \labenv, \labfrom \elab \T : \all \tvs \trcd \T \tvs \to \t
  }
  \\[1ex]
  \entry[Shapes] {\Sh} {\any \tvcs \t \qquad\;\;\ (\Sh \in \Shapes)}
  \\
  \entryset[Canonical principal shapes] {\sh} {\CanonicalShapes \qquad (\CanonicalShapes \subset \Shapes)}
  \end{bnfgrammar}
  \par
  \inferrule[Var]
    {x : \sigma \in \G}
    {\G \th x : \sigma}

  \inferrule[Fun]
    {\G, x : \ta \th e : \tb }
    {\G \th \efun x e : \ta \to \tb}

  \inferrule[App]
    {\G \th \ea : \ta \to \tb \\
     \G \th \eb : \ta}
    {\G \th \eapp \ea \eb : \tb}

  \inferrule[Unit]
    { }
    {\G \th () : 1}

  \inferrule[Gen]
    {\G \th e : \sigma \\ \tv \disjoint \fvs \G}
    {\G \th e : \tfor \tv \sigma}

  \inferrule[Inst]
    {\G \th e : \tfor \tv \ts}
    {\G \th e : \ts \where{\tv \is \t}}

  \inferrule[Let]
    {\G \th \ea : \sigma \\
     \G, x : \sigma \th \eb : \t}
    {\G \th \elet x \ea \eb : \t}

  \inferrule[Annot]
    {\G \th e : \t\where {\tvs \is \tys}}
    {\G \th (e : \exi \tvs \t) : \t\where {\tvs \is \tys}}

  \inferrule [Poly-X]
    {\G \th \e : \ts\where {\tvs \is \tys}}
    {\G \th \expoly \e \tvs \ts : \tpoly {\ts \where {\tvs \is \tys}}}

  \inferrule [Use-X]
    {\G \th \e : \tpoly \ts \where {\tvs \is \tys}}
    {\G \th \exinst e \tvs \ts : \ts \where {\tvs \is \tys}}

  \inferrule[Rcd-X]
    {\dom {(\Labenv[\T])} = \elabs \\
     \parens{{\labfrom \elabi \T} \leq \t \to \ti}\iton \\
     \parens{\G \th \ei : \ti}\iton }
    {\G \th \exrecord \T {\elaba = \ea; \ldots; \elab_n = \en} : \t}

  \inferrule[Rcd-Closed]
    {\labsuni \elabs \T \\
     \G \th \exrecord \T {\overline{\elab = \e}} : \t }
    {\G \th \erecord {\overline{\elab = \e}} : \t}

  \inferrule[Rcd-Proj-X]
    {{\labfrom \elab \T} \leq \tya \to \tyb \\ \G \th \e : \tya }
    {\G \th \exfield \e \T \elab : \tyb}

  \inferrule[Rcd-Proj-Closed]
    {\labuni \elab \T \\
     \G \th \exfield \e \T \elab : \t }
    {\G \th \efield \e \elab : \t}

  \inferrule[Lab-Inst]
    {\labenv(\labfrom \elab\T) = \tfor \tvs {\trcd \T \tvs} \to \t}
    {{\labfrom \elab \T} \leq (\trcd \T \tys \to \t\where{\tvs \is \tys})}
\end{mathparfig}

\parcomment {Running example: polytypes}

\parcomment {We need a spec, but this itself is hard}

% SPJ didn't like 'surface' language here, since OCaml doesn't have
% polytypes in the concrete syntax (though it has them in the abstract syntax,
% but this is an implementation artifact and arguably something I'd like to
% remove).
To prove correctness of type inference, we must define a language and
its type system. Identifying an appropriate declarative type system to use as a
specification is itself a challenging problem. In particular, natural
specifications for fragile features often fail to preserve principality.

\parcomment {Why do natural approaches not guarantee principal types.}

\begin{wraphbox}{}{}
\begin{mathpar}[inline]
  \inferrule[Rcd-Proj-I-Nat]
    {\labfrom \elab \T \leq \ta \to \tb \\\\ \G \th \e : \ta}
    {\G \th \efield \e \elab : \tb}
\end{mathpar}
\end{wraphbox}

Consider records, for instance. We can ask the user to provide a type
annotation by using an explicit record projection $\exfield \e \T \elab$, which
has a simple typing rule (\Rule{Rcd-Proj-X} in \cref{fig/oml/syntax-xtyping}).
By contrast, the natural typing rule for the overloaded projections $\efield \e
\elab$ breaks principality (\Rule{Rcd-Proj-I-Nat}). For example, term $\efun r
\efield r {\ttlab x}$ admits two incompatible types, \code{point -> int} and
\code{gray_point -> int}, as explained in \cref{sec/introduction}.

\parcomment{How we restore principal types (briefly)}

To restore principality for overloaded projections, we require that the type of
$\e$ be \emph{known}---that is, determined to be a specific record type $\trcd
\T \tys$, rather than merely \emph{guessed}. As discussed earlier
(\cref{sec/overview/omni/unicity}), this requirement is expressed via a
\emph{unicity condition}, giving rise to the contextual rule \Rule{Rcd-Proj-I}
(\cref{fig/oml/typing/I}), which elaborates the fragile overloaded projection
$\efield \e \elab$ into its robust explicit counterpart $\exfield \e \T \elab$.

\subsection{Syntax}

\OML (\cref{fig/oml/syntax-xtyping}) extends \ML with two fragile constructs:
polytypes and nominal records.  Variants are not treated formally in \OML, but
behave analogously to records.

\paragraph{Notation for collections}

We write $\overline X$ for a (possible empty) set of elements $\set {X_1,
\ldots, X_n}$ and a (possibly empty) sequence $X_1, \ldots, X_n$. The
interpretation of whether $\overline X$ is a set or a sequence is often
implicit. We write $\overline{X} \disjoint \overline{X'}$ as a shorthand for
when $\overline X \cap \overline {X'} = \emptyset$. We write $\overline X,
\overline {X'}$ as the union or concatenation (depending on the interpretation)
of $\overline X$ and $\overline X'$. We often write $X$ for the singleton set
(or sequence).


\paragraph{Types}

Monotypes (or just types) include, as usual, type variables $\tv$, the unit
type $\tunit$, arrow types, but also nominal record types $\trcd \T \tys$, and
polytypes $\tpoly \ts$. We use a non-standard syntax $\trcd \T \tys$ for record
types, where $\T$ is a \emph{record name} and $\tys$ a list of type
parameters.%
%
\footnote{
  Using $\T$ as an argument of $(\trcd \T \tys)$ rather a
  head constructor $(\Fapp[\T] \tys)$ will make record
  shape patterns easier to understand.
}
%
\Xalistair{For the record, I really don't see the readability benefit of $\trcd
\T \tys$ over $\Tapp \tys$. This would be like writing $\textsf{pair } \ta \;
\tb$ instead of $\ta \times \tb$.}
%
Type schemes $\ts$ are of the form $\all \tvs \t$, they are equal up to the
reordering of binders and removal of useless variables. Standard notions such
as the set of free variables $\fvs \ts$ and capture-avoiding substitutions
$\ts\where{\tv \is \t}$ are defined in the usual way. We write $\TyVars$ for
the set of type variables, and use $\tvs \disjoint \ts$ as a short-hand for
$\tvs \disjoint \fvs \ts$. Finally, $\gt \in \Ground$ denotes a \emph{ground
type}---a type with no free variables.\footnote{$\gt$ can be pronounced
``Fraktur g'' or just ``ground g''.}


\paragraph{Terms}

Terms of \OML are variables~$\x$, the unit literal $\eunit$,
lambda-abstractions $\efun \x \e$, applications $\eapp \ea \eb$, annotations
$\eannot \e \tvs \t$, and let-bindings $\elet x \ea \eb$, extended with the
following expressions:
\begin{itemize}
\item
  For polytypes, we introduce implicit and explicit boxing and unboxing
  forms: $\epoly \e$, $\expoly \e \tvs \ts$, and $\einst \e$, $\exinst \e
  \tvs \ts$ respectively.
\item
  Overloaded record labels include record literals $\erecord { \elaba = \ea;
  \ldots; \elab_n = \en }$ and field projections $\efield \e \elab$. Both
  constructs have explicit counterparts: $\exrecord \T {\elaba = \ea;
  \ldots; \elab_n = \en }$ and $\exfield \e \T \elab$, where the explicit
    record annotation $\labfrom {\_} \T$ indicates that the labels unambiguously
  belong to the record type $\T$.
\TODO{Decide on an ultimate syntax for explicit projections and explicit records.
Didier proposes $\e.\T/\elab$, Alistair proposes $\e.\T.\elab$, Gabriel proposes $\e.\elab^\T$ and $e.\elab@\T$.
For records we have $\T / \erecord { \elab = \e }$,
or $\T.\erecord { \elab = \e }$,
or $\erecord { \elab = \e }^\T$,
or $\erecord { \elab = \e }@\T$.
}
\end{itemize}
All annotations in \OML are closed \ie their quantified variables $\tvs$ are
exactly the free variables of the type $\t$ or type scheme $\ts$.
%
We use $\e^i$ to range over fragile, implicit terms (\eg $\efield \e \elab$) and
$\e^x$ for their explicit counterparts (\eg $\exfield \e \T \elab$).

\parcomment {Continuation onto next sections}

Typing rules for explicit terms are mostly standard; nominal
records require a more intricate (yet largely folklore) treatment of
\emph{closed world} reasoning.
%
The crux of our work is the novel typing of the fragile constructs,
presented in~\cref {sec/oml/typing/I}.

\subsection{Typing rules for robust, explicit constructs}
\label{sec/oml/typing/X}

\parcomment {Simple typing rules explained}

As usual, the main typing judgment $\G \th \e : \ts$
(\cref{fig/oml/syntax-xtyping}) states that in context~$\G$, the expression~$\e$
has the type (scheme) $\ts$.
%
Rules \Rule{Var} through \Rule{Let} are standard. Annotations $\eannot \e \tvs
\t$ (\Rule{Annot}) ensures that the type of $\e$ is (an instance of) the type
$\t$. The type variables $\tvs$ are \emph{flexibly} (or existentially) bound in
$\t$, meaning they may be instantiated to some types $\tys$ so that the
resulting annotation matches the type of $\e$. For instance, the term $(\efun
\x \x + 1 : \exi \tv \tv \to \tv)$ is well-typed under \Rule{Annot} with the
substitution $\where {\tv \is \tint}$.


\paragraph {Explicit polytypes}

% We do *not* write Rule \Rule{X}. X is a noun.
\Rule{Poly-X} serves as the introduction rule: given the (closed) type
scheme~$\ts$, it forms a first-class polytype $\tpoly \ts$, requiring the
expression
$\e$ to be at least as polymorphic as $\ts$. \Rule {Use-X} is the
corresponding
elimination rule, unpacking an expression of polytype $\tpoly \ts$ into
one of polymorphic type $\ts$,
which may be freely instantiated (via \Rule{Inst}). Both rules also allow
polytype annotations to be partial, \ie $\ts$ may have free type variables
$\tvs$, which are existentially quantified to close the annotation, as in
\Rule{Annot}.


\paragraph{Explicit nominal records}
\label {app/typing/X/records}


\parcomment {Label types and contexts}

We assume a global label context $\labenv$ mapping labels to their projection
type, \ie type schemes of the form $\all \tvs \trcd \T \tvs \to \t$. A label
$\elab$ may belong to multiple record types, but is unique within each record
type $\T$. We write $\Labenv$ for the set of labels belonging to the record type $\T$:
%
${\Labenv} \eqdef \set {\elab \mid \labfrom \elab \T : \all \tvs \trcd \T \tvs \to \t \in \labenv}$.
%
We write $\labenv (\labfrom \elab \T)$ for the unique scheme
$\tfor \tvs {\trcd \T \tvs} \to \t$ associated with $\elab$ in $\T$ (if defined).

\parcomment{Typing rules}

Label instantiations are typed by an auxiliary judgment ${\labfrom \elab \T}
\leq \tya \to \tyb$ defined by \Rule{Lab-Inst} and
meaning that $\ta \to \tb$ is an instance of the projection scheme
$\labenv (\labfrom \ell \T)$. Explicit field projections
(\Rule{Rcd-Proj-X}) require that $\exfield \e \T \elab$ projects from a record
$\e$ of type $\tya$ to $\tyb$, provided ${\labfrom \elab \T}
\leq \tya \to \tyb$ holds.
%
Explicit records (\Rule {Rcd-X}) are typed similarly, checking that each field
has the appropriate type. In addition, the premise asserts that the fields
$\elabs$ appearing in the record expression exactly match the labels of $\T$
(\ie $\dom {(\Labenv)}$).

\parcomment {Closed world reasoning}

Following OCaml, our explicit system also supports \emph{closed-world}
reasoning, which exploits the absence of ambiguity in the label context
$\labenv$ to infer record annotations.
%
In particular, in a record expression $\erecord {\elaba = \ea; \ldots; \elab_n
= \en}$, if the set of labels ${ \elaba, \ldots, \elab_n }$ uniquely identifies
a record type $\T$ in the context~$\labenv$, then the record has the
type of $\exrecord \T {\elaba = \ea; \ldots; \elab_n = \en}$ (via
\Rule{Rcd-Closed}).
%
Similarly, if the label $\elab$ is associated with exactly one record type $\T$
in $\labenv$, then the projection $\efield \e \elab$ has the type of $\exfield
\e \T \elab$ (by \Rule{Rcd-Proj-Closed}).

\parcomment {Closed sets}
These two forms of label uniqueness differ. A \emph{closed set} of labels
may uniquely identify a record type even if no individual label is unique.
Conversely, a unique label implies uniqueness of every closed set containing it.
For instance, recall \code{one} from \cref{sec/overview/overloading}:
\begin{program}[input]
type point  = { x : int; y : int }
type gray_point = { x : int; y : int; color : int }
let one = { x = 42; y = 1337 }           °\ocamlflags 00°
\end{program}
Here, the closed set $\set {\text{\code{x}}, \text{\code{y}}}$ in \code{one}
uniquely identifies \code{point}, even though the individual labels \code{x}
and \code{y} also appear in \code{gray_point}.

\parcomment {Formalization of closed-world reasoning}

We formalize this \emph{closed world uniqueness} using the predicates $\labuni \elab \T$ (a label
uniquely identifies $\T$) and $\labsuni \elabs \T$ (a closed label set uniquely
identifies $\T$):
\begin{mathpar}
  \begin{tabular}{C;C;L;C;L}
    \labuni \elab \T &\eqdef& \elab \in \dom {(\Labenv)} &\wedge& \forall \Tp, \; \parens {\elab \in \dom {(\Labenv[\Tp])} \implies {\T} = \Tp} \\
    \labsuni \elabs \T &\eqdef& \dom {(\Labenv)} = \elabs &\wedge& \forall \Tp, \;\parens {\dom {(\Labenv[\Tp])} = \elabs \implies {\T} = \Tp}
  \end{tabular}
\end{mathpar}
These predicates depend only on the global label environment $\labenv$:
they ignore field types and require no contextual type information.
The associated typing rules (\Rule{Rcd-Closed}, \Rule{Rcd-Proj-Closed}) are
therefore \emph{robust}, since disambiguation relies solely on globally known
label information rather than type-directed disambiguation.

\subsection{Shapes}
\label{sec/constraints/shapes}

\parcomment {Why shapes?}

We introduce \emph{shapes} as a generalization of type constructors.
%% for suspended match constraints.  They would provide the same benefit in
%% PolyML without match constraints.
They provide a uniform treatment of both
constructors and polytypes, and are useful in defining polytype
unification (\Cref{sec:implementation}).

\parcomment {Definition of shape}

A shape $\Sh$ is a type with holes, written $\any \tvcs \t$, where $\tvcs$
denotes the set of type variables representing the holes.  By construction, we
require $\tvcs$ to be \emph{exactly} the free variables of $\t$.  Hence, shapes
are closed and do not contain useless binders.  We consider shapes equal up to
$\alpha$-conversion.  When $\t$ is a ground type, we omit the binder and
simply write $\t$ for the shape.
%
We use $\bot$ to denote the shape $\any \tvc \tvc$, which we call the
\emph{trivial} shape. Let $\Shapes$ denote the set of all shapes and
$\Shapesz \subset \Shapes$ the set of non-trivial shapes.

\parcomment {Instantiation order of shapes}

\begin{wraphbox}{}{}
\begin{mathpar}[inline]
  \infer[Inst-Shape]
    {\tvcs_2 \disjoint \any {\tvcs_1} \t}
    {\any {\tvcs_1} \t \preceq
     \any {\tvcs_2} \t \where {\tvcs_1 \is \tys_1}}
\end{mathpar}
\end{wraphbox}
Shapes are equipped with the standard instantiation ordering,
defined by \Rule{Inst-Shape}.
%
When writing $\Sh \preceq \Shp$, we say that $\Sh$ is more general
than~$\Shp$. When $\Sh$ and $\Shp$ are more general than one another, they
are actually equal. The trivial shape $\bot$ is the most general shape.
%
If $\Sh$ is $\any \tvcs \t$, the shape application $\shapp[\Sh] \tys$ is
defined as $\t \where {\tvcs \is \tys}$. We say that $\Sh$ is a shape of
$\t$ when there exists $\tys$ such that $\t = \shapp[\Sh] \tys$; in this
case, we call the pair $(\Sh, \tys)$ a decomposition of $\t$.

\begin{definition}
A non-trivial shape $\Sh \in \Shapesz$ is the principal shape of the type
$\t$ iff:
\begin{enumerate}
  \item
    $\exists \typs,\ \t = \shapp[\Sh] \typs$
  \item
    $\forall \Shp \in \Shapesz, \forall \typs,\ \t = \shapp[\Shp] \typs
    \implies \Sh \preceq \Shp$
\end{enumerate}
\end{definition}

\begin{restatable}[Principal shapes]{theorem}{principalShapesBIS}
  \label{thm:principal-shapes}
Any non-variable type $\t$ has a non-trivial principal shape $\Sh$.
\end{restatable}

\parcomment {Define canonical shape}

A principal shape $\any \tvcs \t$ is \emph{canonical} if its free variables
appear in the sequence $\tvcs$ in the order in which they occur in $\t$. Canonical
principal shapes are written $\sh$, and we write $\CanonicalShapes$ for the set of
all such chapes.
%
Each non-variable type $\t$ has a unique canonical principal shape,
written $\shape \t$. For example, $\shape {\trcd \T \tys}$ is
$\any \tvcs \trcd \T \tvcs$.

\parcomment {Polytypes are constructors}

Shapes are particularly interesting in the context of polytypes, since
polytypes can be decomposed into shapes and thus treated analogously to
type constructors.
\begingroup
\newcommand {\tsh}[1]%
  {\def \tsi {\all \tvb {{\parens{\tvb \to #1}}} \tprod \tvb}
  \tpoly {\all \tva {\parens {\tpoly \tsi} \to \tva} \to \tva}}%
For instance, the polytype $\tsh {\tint \tlist}$ has the principal shape
$\sh$ equal to $\any \tvc {\tsh \tvc}$. The original polytype can thus be
represented as the shape application $\shapp (\tint \tlist)$.
\endgroup


\subsection {Typing rules for fragile, implicit constructs}
\label{sec/oml/typing/I}

\begin{mathparfig}[htpb!]
  {fig/oml/typing/I}
  {Typing rules for fragile, implicitly typed extensions.}
  \begin{bnfgrammar}
    \entry[Terms]{\e}{
      \dots
      \and \emagic \es
    }\\
    \entry[Term contexts]{\E}{
      \hole
      \and \eapp \E \e
      \and \eapp \e \E
      \and \dots
    }
  \end{bnfgrammar}
  \par
  \inferrule[Hole]
    {\parens{\G \th \ei : \ti}\iton}
    {\G \th \emagic \es : \tp}

  \inferrule[Proj-I]
    {\eshape \E \e {\any \tvcs \Pi\iton \tvcs} \\\\
     \G \th \E\where{\exproj \e j n} : \t}
    {\G \th \E\where{\eproj \e j} : \t}

  \inferrule [Use-I]
    {\eshape \E  \e {\any \tvcs \tpoly \ts} \\\\
     \G \th \E\where{\exinst \e \tvcs \ts} : \t}
    {\G \th \E\where{\einst \e} : \t}

  \inferrule [Poly-I]
    {\Eshape \E \e {{\any \tvcs \tpoly \ts}} \\\\
     \G \th \E \where{\expoly \e \tvcs \ts} : \t}
    {\G \th \E \where{\epoly \e} : \t}
\\
  \inferrule[Rcd-I]
    {\Eshape \E \es {\any \tvcs \trcd \T \tvcs} \\
     \G \th \E\where{\exrecord \T {\overline{\elab = \e}}} : \t }
    {\G \th \E\where{\erecord {\overline{\elab = \e}}} : \t}

  \inferrule[Rcd-Proj-I]
    {\eshape \E \e {\any \tvcs \trcd \T \tvcs} \\
     \G \th \E\where{\exfield \e \T \elab} : \t}
    {\G \th \E\where{\efield \e \elab} : \t}

\eshape[\es] \E \e \sh \quad\eqdef\quad
  \forall \G, \t, \gt, \uad
  \G \th \eerase {\E \where {\emagic {\es, \eannot \e {} \gt }}} : \t
      \wide\implies \shape \gt = \sh

\Eshape \E \es \sh \quad\eqdef\quad
  \forall \G, \t, \gt, \uad
      \G \th \eerase {\E\where{\eannotmagic \es {} \gt}} : \t
      \wide\implies \shape \gt = \sh
\end{mathparfig}

\parcomment{Typing rules are contextual. Re-iterate the idea of unicity}

We now turn to the typing of fragile implicit constructs
(\cref{fig/oml/typing/I}). To prevent the kind of uncontrolled guessing
permitted by \emph{natural} typing rules (\eg \Rule{Rcd-Proj-I-Nat}), we adopt
\emph{contextual} typing rules: an implicit term $\e^i$ is typed \emph{within}
a surrounding one-hole term context $\E$ (\cref{fig/oml/typing/I}). The context
$\E$ is used to ensure that the relevant type information (\eg a shape $\sh$)
is \emph{known}---that is, it is the unique shape~$\sh$ that can be inferred
from the context, rather than an arbitrary one that could be \emph{guessed}.
Such rules are therefore inherently non-compositional.

\paragraph{Unicity}

\parcomment{What are unicity conditions? e.g. notation}

The key question for our contextual typing rules is whether the shape $\sh$ of
a term's type is \emph{uniquely determined} by its surrounding context $\E$. We
capture this with our \emph{unicity conditions} $\eshape[\es] \E \e \sh$ and
$\Eshape \E \es \sh$, which state that all valid typings of the (erased)
context $\E$ assign the same canonical shape $\sh$ to the subterm $\e$ and the
context's hole $\hole$, respectively.

In other words, $\eshape[\es] \E \e \sh$ holds when the type of the subterm
$\e$ has a unique shape $\sh$ fixed by its context $\E$ and sibling terms
$\es$, while $\Eshape \E \es \sh$ holds when the expected type of the hole in
$\E$ (with subterms $\es$) has the unique shape $\sh$.
\parcomment{The formal definition / auxiliary notions}
\begin{mathpar}
\def \Eqdef {&\eqdef&}
{\begin{tabular}{RCL}
\eshape[\es] \E \e \sh \Eqdef
  \forall \G, \t, \gt, \uad
  \G \th \eerase {\E \where {\emagic {\es, \eannot \e {} \gt }}} : \t
      \wide\implies \shape \gt = \sh
\\[1ex]
\Eshape \E \es \sh \Eqdef
  \forall \G, \t, \gt, \uad
      \G \th \eerase {\E\where{\eannotmagic \es {} \gt}} : \t
      \wide\implies \shape \gt = \sh
\end{tabular}}
\end{mathpar}
%
To make sense of these definitions, we rely on a few auxiliary notions:
%
\begin{enumerate}
  \item
    We write $(\emagic \es)$ for a \emph{typed hole} carrying subterms $\es$.
    The subterms $\es$ are required to be well-typed in the current
    environment (\Rule{Hole}), but their types are independent of the type of
    the hole: the hole itself may be assigned an arbitrary type.

  \item
    We introduce an \emph{erasure} function $\eerase \e$ that erases all
    not-yet-elaborated implicit constructs $\e^i$ in $\e$ with a typed hole
    around their subterms. For example, $\eerase {\efield \e \elab}$ is
    $\parens{\emagic {\cerase \e}}$. The full definition is given in
    \cref{fig/oml/erasure}.
\end{enumerate}

\begin{mathparfig}
  {fig/oml/erasure}
  {Selected cases of the erasure of $\e$. All other cases are homomorphic.}
\newcommand{\Erule}[2]{\eerase {#1} &\eqdef& {#2}}
  \begin{tabular}{RCL}
  \Erule{\epoly \e}{\emagic {\eerase \e}}\\
  \Erule{\einst \e}{\emagic {\eerase \e}}\\
  \Erule{\exinst \e \tvs \ts}{\exinst {\eerase \e} \tvs \ts}\\
    \Erule{\erecord {\elaba = \ea; \ldots; \elab_n = \en}}{\begin{cases}
      \erecord {\elaba = \eerase \ea; \ldots; \elab_n = \eerase \en} &\text{if } \labsuni \elabs \T \\
      \emagic {\eerase \ea, \ldots, \eerase \en} & \text{otherwise}
    \end{cases}}\\
  \Erule{\exrecord \T {\elaba = \ea; \ldots; \elab_n = \en}}{\exrecord \T {\elaba = \eerase \ea; \ldots; \elab_n = \eerase \en}}\\
    \Erule{\efield \e \elab}{\begin{cases}
      \efield {\eerase \e} \elab & \text{if } \labuni \elab \T \\
      \emagic {\eerase \e} & \text{otherwise}
    \end{cases}}\\
  \Erule{\exfield \e \T \elab}{\exfield {\eerase \e} \T \elab}\\
  \Erule{\emagic \es}{\emagic {\parens {\eerase \ei} \iton}}\\
\end{tabular}
\end{mathparfig}

\parcomment{Typed holes must preserve typability of subterms}

Typed holes ensure that subterms---such as type annotations---remain
present even when the implicit construct containing them is erased. This is
because they may introduce constraints that can still contribute to unicity.
For instance, $\efun r \efield {\eannot r {} {\ttlab {point}}} {\ttlab x}$
would be ill-typed if we erased $\eannot r {} {\ttlab {point}}$.

\parcomment{Erasure induces a causal order}

The erasure of $\eerase{\E\where{\emagic{\eannot \e {} \gt, \es}}}$ and
$\eerase{\E\where{\eannotmagic \es {} \gt}}$ replaces all
not-yet-elaborated implicit constructs with typed holes, ensuring that the
unicity of $\sh$ is determined \emph{only} by implicit terms that have been
elaborated thus far. This induces a causal, or partial, order among implicit
constructs: an implicit term can only be elaborated once all of its
dependencies---those providing the relevant \emph{known} type
information---have themselves been elaborated. This ordering prevents
self-justifying cycles in unicity and rules out ``out-of-thin-air'' guesses.

\parcomment{Well-foundedness}

The attentive reader may notice that our unicity conditions contain a
negative occurrences of the typing judgment. At first glance, this seems
problematic for well-foundedness. However, the issue is easily resolved by
noting that these occurrences arise only as typing assumptions on erased
terms, which do not themselves involve any implicit rules. Formally one
could make this explicit by introducing a restricted typing judgement $\G
\throbust \e : \t$ that exlcudes the implicit rules (\NoRule{*-I}), and by
using this strictly simpler judgment in the antecedent of unicity
conditions. We omit this distinction for simplicity.

\paragraph{The omnidirectional recipe}
%
All typing rules instantiate a common framework---the \emph{omnidirectional
recipe}---which ensures that certain omitted type annotations are uniquely
determined from the context. Each construct, however, requires a specific
instantiation of the framework.
%
We first describe the framework, then present each feature separately.

\begin{enumerate}
  \item[\emph{Step 1}:] \emph{Contextualize.}
    %
    Each implicit fragile term $\e^i$ is typed relative to a surrounding one-hole term context $\E$:
    its rule asserts the typability of $\G \th \E \where{\e^i} : \t$ as the conclusion.

  \item[\emph{Step 2}:] \emph{Select a unicity condition.}
    %
    This is the secret ingredient! The unicity condition ensures that the shape
    $\sh$ is fully determined by the surrounding context $\E$ and
    subexpressions $\es$ (\eg the subexpressions $\es$ in $\erecord
    {\overline{\elab = \e}}$).

    %
    If the $\e^i$ is an introduction form, we infer the shape from the context's
    hole $\Eshape \E \es \sh$. If $\e^i$ is an elimination form, we infer
    the shape from the \emph{principal term} $\e$ (the term whose type
    contains the connective we're eliminating): $\eshape[\es] \E \e \sh$.

  \item[\emph{Step 3}:] \emph{Elaborate.}
    The uniquely inferred shape $\sh$ is used to elaborate $\e^i$ into its
    explicit counterpart~$\e^x$, and the rule asserts $\G \th \E\where{\e^x}
    : \t$ as a premise.
\end{enumerate}

\paragraph{Implicit polytypes}

Unboxing a polytype $\einst \e$ is an \emph{elimination form}. Following the
omnidirectional recipe, \Rule{Use-I} is a contextual rule (\emph{Step 1})
requiring that the principal term $\e$ have the unique shape $\any \tvcs \tpoly
\ts$ in the context $\E$ (\emph{Step 2}). In \emph{Step 3}, we then elaborate
$\einst \e$ into $\exinst \e \tvcs \ts$.
%
Conversely, boxing with $\epoly \e$ is an \emph{introduction form}. In
\Rule{Poly-I}, we require that the expected type of the context's hole $\E$ has
the shape $\any \tvcs \tpoly \ts$ (\emph{Step 2}). We then type $\epoly \e$ as
$\expoly \e \tvcs \ts$ (\emph{Step 3}).


\paragraph{Implicit nominal records}

Overloaded record labels are handled analogously.
%
Typing record projections in \Rule{Rcd-Proj-I} is an \emph{elimination
  form} for the record type $\trcd \T \tys$: the projection $\efield \e \elab$
is typed as $\exfield \e \T \elab$ (\emph{Step 3}) provided the
type of expression $\e$ in context $\E$ has record shape
$\any \tvcs \trcd \T \tvcs$ (\emph{Step 2}).
%
For record construction, $\erecord {\overline{\elab = \e}}$ is an
\emph{introduction form}. In \Rule{Rcd-I}, we type an overloaded record
$\erecord {\overline{\elab = \e}}$ as $\exrecord \T {\overline{\elab = \e}}$
(\emph{Step 3}), provided the context $\E$ with subterms $\es$ expects a
record type of shape $\any \tvcs \trcd \T \tvcs$ (\emph{Step 2}).

\parcomment {Illustrative examples}
We now illustrate the typing of implicit constructs with a few examples.

% Moved global macros in notations.sty
% Order of solving
\begin{example}
  \locallabelreset
  Consider the term $\ttex {3} \eqdef \efun r {\etuple {\efield r \ttx,
  \exfield r \ttpoint \tty}}$ from \cref{sec/overview/overloading}.%
  %
  \footnote{
    The typing rules for tuples are standard and present in Appendix
    \cref{app:full-reference}.
  }
  In \ocaml{ex_3}, $r$ can only be of type \ocaml{point}. Indeed,
  considering the second projection first, we should learn that $r$ is of type
  \ocaml{point} (using \Rule{Rcd-Proj-I}) and since it is
  $\lambda$-bound, this makes the first projection unambiguous. (For record
  types without parameters, we use $\T$ as a shorthand for $\trcd \T \emptyset$.)

  Formally, we derive:
  \begin{mathpar}
    \infer* [Right=Rcd-Proj-I]
      {\eshape \E r {\ttpoint} \\
       \eset \th \E\where{\exfield r \ttpoint \ttx} : \ttpoint \to \tint \tprod \tint}
      %--------------------------------------------------------------------------------------
      {\eset \th \E\where{\efield r \ttx} : \ttpoint \to \tint \tprod \tint }
  \end{mathpar}
  %
  where the context $\E$ is $\efun r {\etuple {\square, \exfield r \ttpoint \tty}}$.
  We have $\eset \th \E\where{\exfield r \ttpoint \ttx} :
  \ttpoint \to \tint \tprod \tint$, indeed.
  %
  Therefore, it remains to show that $\eshape \E r {\ttpoint}$~\llabel
  C. Assume $\eset \th \E\where{\emagic {\eannot r {} \gt}} :
  \t$. Since $\exfield r \ttpoint \tty$ requires $r$ to have the type
  $\ttpoint$ (due to Rule \Rule{Rcd-Proj-X} and \Rule{Lab-Inst}),
  it follows that there is no other choice but to take $\ttpoint$ for $\gt$, which proves~\lref C.
\end{example}

% Unicity is good
\begin{example}
To illustrate a simple case of non-typability, we reconsider the example $\ttex
1 \eqdef \efun r {\efield r \ttx}$ from \cref{sec/overview/overloading}.
%
If there is a derivation of $\ttex 1$, then there must be one of the form:
\begin{mathpar}
  \infer* [Right=Rcd-Proj-I]
    { \eshape \E r {\any \tvcs \trcd \T \tvcs} \\
      \eset \th \E\where{\exfield r \T \ttx} : \t}
    %----------------------------------------------
    { \eset \th \E\where{\efield r \ttx} : \t}
\end{mathpar}
where $\E$ is the term $\efun r \hole$, which is the largest possible
context.
%
Unfortunately, $\eshape \E r {\any \tvcs \trcd \T \tvcs}$ does not hold for any
$\T$. Indeed, we have $\eset \th \E \where {\emagic {\eannot r {} {\gt}}} :
\gt \to \t$.  for any $\gt$ and $\t$.  Hence, $\ttpoint$ and $\ttgraypoint$
are both possible shapes for the type of $r$.
\end{example}

\def \ttciecolor {\ttlab {cie\_color}}
\def \ttciepoint {\ttlab {cie\_point}}

\begin{example}
Considering the example from \cref{sec/overview/limitations}.
\begin{program}[input,checkocaml]
type gray_point = { x : int; y : int; color : int }
type cie_color  = { x : int; y : int; z : int }
type cie_point  = { x : int; y : int; color : cie_color }
let ex_1_0 r = r.color.x °\ocamlflags 11°
\end{program}

\parcomment{Proof that this is ill-typed}

As explained in \cref{sec/overview/limitations}, \code{ex_1_0} is ill-typed
because our unicity conditions enforce that each implicit field projection
must be resolved individually and in a sequential fashion. We view this as a
``Goldilocks'' solution: we deliberately trade a small loss in expressivity
for the benefit of a tractable, backtracking-free inference algorithm.

To type $\ttex {10}$ one must eliminate the final implicit
projection in a context of the form $\E\where{\efield \e \elab}$.
It is a given that neither projection can be resolved by applying
\Rule{Rcd-Proj-Closed} since the labels (\code{color} and \code{x})
are both overloaded. Thus, the implicit projections must
be typed using \Rule{Rcd-Proj-I}.
Two cases arise, neither of which are possible:
\begin{proofcases}
\proofcase{$\E$ is $\efun r {\hole}$}
%
We should have a derivation that ends with
\begin{mathpar}
  \infer*[Right=Rcd-Proj-I]
    {\eshape \E {\efield r \ttcolor} {\ttciecolor} \\
     \eset \th \E \where {\exfield {\efield r \ttcolor} \ttciecolor \ttx} : \t}
    %------------------------------------------------------------------
    {\eset \th \E \where {\efield {\efield r \ttcolor} \ttx} : \t}
\end{mathpar}

However, $\eshape \E {\efield r \ttcolor} {\ttciecolor}$ does not hold. Indeed,
the judgment
%
  $\eset \th \eerase {\E \where {\eannot {\emagic {\efield r \ttcolor}} {}
  {\gt}}} : \gt \to \tp$
%
(\ie $\eset \th \efun r {\eannotmagic r {} {\gt}} : \gt \to \tp$) holds for
any $\gt$.  Hence, the shape of the type of $\efield {\efield r \ttcolor}
\ttx$ is not uniquely determined and this case cannot occur.

\proofcase{$\E$ is $\efun r {\efield {\hole} \ttx}$}
%
The derivation must end with:
\begin{mathpar}
  \infer*[Right=Rcd-Proj-I]
    {\eshape \E {r} {\ttciepoint} \\
     \eset \th \E\where{\exfield r \ttciepoint \tty} : \t}
    %--------------------------------------------------
    {\eset \th \E \where{\efield r \ttcolor} : \t}
\end{mathpar}

However, $\eshape \E {r} {\ttciepoint}$ does not hold
either. Again, the judgment
%
$\eset\th \eerase {\E\where{\eannotmagic r {} {\gt}}} : \tp \to \gt$,
%
(\ie $\eset\th \efun r {\eannotmagic r {} {\gt}} : \tp \to \gt$) holds for
any $\gt$.

\end{proofcases}
\end{example}


\section{Constraints}
\label{sec:constraints}


%% BEWARE: The following two figures only be located as the top of a
%% page---otherwise  it would be  misplaced

\begin{mathparfig}[t]%
  {fig:constraints}%
  {Selected syntax and semantics of constraints.}
\begin{bnfgrammar}
\entry[Constraints]{\c}{
        \ctrue
  \and  \cfalse
  \and  \ca \cand \cb
  \and  \cunif \ta \tb
  \and  \cexists \tv \c
  \and 	\cfor \tv \c
  \nextline
  \and  \clet \x \tv \ca \cb
  \and  \capp \x \t
  %% \nextline
  \and  \cmatch \t \cbrs
}\\[1ex]
\entry[Branches]{\cbr}{\cbranch \cpat \c} \\
\entry[Shape patterns]{\cpat}{
  \cpatwild
  \and \ldots
}{} \\
\entry[Constraint contexts]{\C}{
  \hole
  \and \C \cand \c
  \and \c \cand \C
  \and \cexists \tv \C
  \and \cfor \tv \C
  \nextline
  \and \clet \x \tv \C \c
  \and \clet \x \tv \c \C
} \\[1ex]
\entry[Semantic environments]{\semenv}{
  \emptyset
  \and \semenv\where{\tv \is \gt}
  \and \semenv\where{\x \is \gabs}
} \\
\entryset[Ground types]{\gt}{\Ground} \\
\entrysubseteq[Sets of ground types]{\gabs}{\Ground}{}
\end{bnfgrammar}

  \infer[True]
    { }
    {\semenv \th \ctrue}

  \infer[Conj]
    {\semenv \th \ca \\
     \semenv \th \cb}
    {\semenv \th \ca \cand \cb}

  \infer[Unif]
    {\semenv(\ta) = \semenv(\tb)}
    {\semenv \th \cunif \ta \tb}

  \infer[Exists]
    {\semenv\where{\tv \is \gt} \th \c}
    {\semenv \th \cexists \tv \c}

  \infer[Forall]
    {\forall \gt, ~ \semenv\where{\tv \is \gt} \th \c}
    {\semenv \th \tfor \tv \c}

  \infer[Let]
    {\semenv \th \exists \tv. \ca \\
     \semenv\where{\x \is \semenv(\cabs \tv \ca)} \th \cb}
    {\semenv \th \clet \x \tv \ca \cb}

  \infer[App]
    {\semenv(\t) \in \semenv(\x)}
    {\semenv \th \capp x \t}

  {\let \Eqdef\eqdef \def \eqdef {&\Eqdef&}
  \begin{tabular}[c]{.R.C;.L.}
  \semenv(\cabs \tv \c) \eqdef
    \set {\gt \in \Ground : \semenv\where{\tv \is \gt} \th \c}
  \\
  \ca \centails \cb \eqdef
    \forall \semenv,\ \semenv \th \ca \implies \semenv \th \cb
  \\
  \ca \cequiv \cb   \eqdef
    (\ca \centails \cb) \wide\wedge   (\ca \centails \cb)
  \end{tabular}}
\par
\begin{tabular}{.L.}
  \cmatched \t \sh {\cbranch \cpats \cs} \eqdef \\[.6ex]
\uad \begin{cases}
    \cexists \tvs \cunif \t \shapp \tvs \cand \theta(\ci) &
    \text{if } \cmatches \cpati \sh \tvs \theta\\
    \cfalse & \text{otherwise}
\end{cases}
\end{tabular}
%% Minimal spacing
\qquad\qquad
%% Plus one share of free extra space
\hfil
{\let \Eqdef\eqdef \def \eqdef {&\Eqdef&}\def \EQ{&\partialmap&}
\begin{tabular}[c]{.R.C;.L.}
  \hline \vrule\;
  \cmatches[\EQ] \cpat \sh \tvcs \theta
  \;\vrule \\ \hline
  \noalign{\vskip .6ex}
  \cmatches[\EQ] \cpatwild \sh \tvcs \eset \\
  &\vdots&
\end{tabular}}

\infer[Match-Ctx]
    {\Cshape \C \t \sh \\
      \semenv \th \C\where{\cmatched \t \sh \cbrs}
    }
    {\semenv \th \C\where{\cmatch \t \cbrs}}
\hfill
\begin{tabular}{.L.}
  \Cshape \C \t \sh \eqdef \\[.6ex]
  \quad \forall \semenv, \gt. \uad \semenv \th \cerase {\C\where{\cunif \t \gt}} \implies \shape \gt = \sh
\end{tabular}
\end{mathparfig}

\parcomment{Why bother with a formal constraint language?}

To reason about constraint-based inference, we need more than a procedure for
generating and solving constraints: we require a \emph{formal logic} of
constraints, with a syntax and a declarative semantics that characterizes
satisfiability. This semantics is essential: it validates the design of our
constraint language and provides the foundation for proving the soundness,
completeness, and principality of inference. Without it, the meta-theory of our
approach cannot be stated precisely.

\parcomment{Defining the syntax / semantics}

We now introduce the syntax and semantics of our constraint language.
Building atop the constraint-based inference framework of
\citet*{Pottier-Remy/emlti}, we adopt a constraint language
(\cref{fig:constraints}) that includes both term and type variables. Its
semantics is given by a satisfiability judgment $\semenv \th \c$
(\cref{fig:constraints}). The semantic environment $\semenv$ assigns to each
free type variable $\tv$ a ground type $\gt \in \Ground$ (a type with no
free variables) and to each term variable $\x$ a set of ground
types\footnote {$\gabs$ can be pronounced ``Fraktur S'' or ``ground S''.}
$\gabs \subseteq \Ground$ (the instances of a type scheme bound to
$\x$).
%
We write $\semenv\where{\tv \is \gt}$ and $\semenv\where{\x \is \gabs}$ for
extensions of $\semenv$, and $\semenv(\t)$ for the ground type obtained by
substitution.


\parcomment{Semantics detailed}

Constraints include basic logical forms: tautological $\ctrue$ (\Rule{True}),
unsatisfiable $\cfalse$, and conjunctive $\ca \cand \cb$ (\Rule{Conj})
constraints. The unification constraint $\cunif \ta \tb$ is satisfied when
$\ta$ and $\tb$ are equal (\Rule{Unif}).
%
An existential constraint $\cexists \tv \c$ holds if there exists a witness
$\gt$ for $\tv$ satisfying $\c$ (\Rule{Exists}), while a universal constraint
$\cfor \tv \c$ holds if $\c$ is satisfied for every binding of $\tv$
(\Rule{Forall}).
%
When $\ts$ is a polymorphic type scheme $\tfor \tvs \tp$, we use the notation
$\cleq \ts \t$ as shorthand for the instantiation constraint $\cexists
\tvs \cunif \tp \t$. \OML-specific constraints, such as record label
instantiation $\labfrom \elab \ct \leq \ta \to \tb$
(\cref{sec/overview/omni/match}), are introduced later (\cref{sec/oml/constraints}).

\parcomment{Constraint abstractions}

Polymorphism in the constraint language is expressed through
\emph{generalization} and \emph{instantiation} constraints. In a generalization
constraint (or \emph{let} constraint) $\clet \x \tv \ca \cb$, the definition of
$\x$ is a constraint abstraction $\cabs \tv \ca$, a function that, when applied
to a type $\t$, returns $\ca \where {\tv \is \t}$. The binding of $\x$ is then
available in $\cb$ and refers to this abstraction. Semantically, the
abstraction $\cabs \tv \ca$ is interpreted as a set of ground types that
satisfies $\ca$, and the generalization constraint requires this set to be
non-empty, \ie there is at least one instantiation of $\tv$ that satisfies
$\ca$.
%
Instantiations (or applications) $\capp \x \t$ eliminate abstractions by
applying a type $\t$ to the abstraction bound to $\x$. This holds precisely
when $\semenv(\t) \in \semenv(\x)$, \ie $\t$ is one of the satisfiable
instances of $\x$. The $\lambda$-abstraction and application syntax follows
\citet*{Pottier/inferno@icfp2014}. In other presentations, constraint
abstractions $\cabs \tv {\cexists \tvbs \c}$ are written as constrained type
schemes $\all {\tv, \tvbs} C \Rightarrow \tv$, and instantiation constraints
$\capp \x \t$ are written $\x \leq \t$.

\parcomment {Suspended match constraints}

Finally, we introduce \textit{suspended match constraints} $(\cmatch \t \cbrs)$,
which consist of:
\begin{enumerate}

\item
  A matchee $\t$. The constraint remains suspended
  until the \emph{shape} of $\t$ is determined, \ie
  while $\t$ is a type variable.

\item

  A list of branches $\cbrs$ of the form $\cbranch \cpat \c$, where
  $\cpat$ is a shape pattern.\footnote{The match constraints in this
    paper only use a single branch; but in a larger language there
    would be use-cases for having several branches; for example,
    record projection could be overloaded to work both on nominal
    records and on tuples (or objects, modules, etc.), requiring
    several branches in its generated constraint. So we kept the
    general syntax.} The constraint $\c$ is solved in the extended
  context produced by the matching pattern.

  For example, the wildcard pattern $\cpatwild$ matches any shape,
  and binds nothing.
  %
  To ensure determinism, the set of patterns $\bar \cpat$ must be
  \emph{disjoint}---that is, no shape may be matched by more than one pattern
  in the list.

\end{enumerate}
The formal semantics of suspended match constraints are somewhat involved;
we return to them in the next subsection (\cref{sec/constraints/semantics}).

\begin{wraphbox}{0.2}{0.6}
\begin{mathpar}[inline]
\infer*[right=Exists]
    {\infer*[Right=Unif]
      {\infer*{}{\tint = \tint}}
      {\semenv\where{\tv \is \tint} \th \cunif \tv \tint}}
  {\semenv \th \cexists \tv \cunif \tv \tint}
\end{mathpar}
\end{wraphbox}
Closed constraints are either satisfiable in any semantic environment (\ie
they are tautologies) or unsatisfiable. For example, the satisfiability of
the constraint $\cexists \tv {\cunif \tv \tint}$ is established by the
derivation on the right-hand side.

% Equivalence and entailment

We write $\ca \centails \cb$ to express that $\ca$ \emph{entails} $\cb$,
meaning every solution $\semenv$ to $\ca$ is also a solution to $\cb$.
We write $\ca \cequiv \cb$ to indicate that $\ca$ and $\cb$ are equivalent,
that is, they have exactly the same set of solutions.

\parcomment {Constraint contexts}

Throughout this paper, we will find it convenient to work with \emph{constraint
contexts}. A constraint context $\C$ is simply a constraint with a \emph{hole},
analogous to term contexts $\E$ introduced in \cref{sec/oml}. We write
$\C\where{\c}$ to denote filling the hole of the context $\C$ with the
constraint $\c$. Hole filling may capture variables, so we frequently require
explicit side conditions when variable capture must be avoided. We write $\bvs
\C$ for the set of variables bound at the hole in $\C$.

\subsection{Suspended constraints}
\label{sec/constraints/semantics}

\parcomment{Defining semantics for suspended constraints is hard}

A central difficulty in our work on suspended constraints was defining a
satisfying semantics. The challenge lies in formalizing what it means for type
information to be \emph{known} without presupposing a \emph{static} solving
order. This is the same issue encountered in our typing rules
(\cref{sec/oml/typing/I}), and we address it in the same way: by introducing a
contextual rule together with a unicity condition.

To define the semantics for suspended constraints, we first introduce
\emph{discharged match constraints}.

\begin{definition}[Discharged match constraint]\label{def/discharged}
  Given a suspended constraint $(\cmatch \t \cbrs)$ and a canonical shape
  $\sh$, we introduce the syntactic sugar $(\cmatched \t \sh \cbrs)$ for the
  \emph{discharged match constraint} that selects the branch in $\cbrs$ that
  matches $\sh$:
\begin{mathpar}
  \cmatched \t \sh {\overline {\cbranch \cpat \c}} \uad\eqdef\uad
  \begin{cases}
    \cexists \tvs \cunif \t \shapp \tvs \cand \theta(\ci)
    & \text{if } \cmatches \cpati \sh \tvs \theta\\
    \cfalse & \text{otherwise}
  \end{cases}
\end{mathpar}

The first conjunct ($\tau = \shapp \tvs$) ensures that $\sh$ is indeed the
canonical shape of $\t$, and the second conjunct is the selected branch
constraint $\ci$ under the appropriate substitution. Since the syntax of
suspended match constraints requires that branch patterns are
non-overlapping, the matching branch $\cbranch \cpati \ci$ is uniquely
determined. It may, however, be undefined if the branches are not exhaustive,
in which case the discharged constraint is $\cfalse$.

\end{definition}

The partial function $(\Cmatches \cpat \sh \tvs)$, introduced in
\cref{fig:constraints}, describes how a pattern $\cpat$ is matched against a
canonical principal shape $\sh$. Before matching, $\sh$ is applied with fresh
shape variables $\tvs$ (of the same arity as $\sh$), so that the result of
matching may refer to them. The match either fails or returns a substitution
$\theta$ mapping pattern variables (\eg record name variables $\ct$ from
\cref{sec/overview/omni}) to shape components (\eg record names $\T$).  These
components may themselves mention the freshly introduced variables $\tvs$.
In~\cref {fig:constraints}, we only introduce the wildcard pattern and its
matching rule; in~\cref {sec/oml/constraints} we extend both the syntax of
shape patterns and the matching function itself to handle the patterns that
arise in \OML.


\paragraph {A natural attempt}

As with typing rules for fragile constructs (\eg \Rule{Rcd-Proj-I-Nat}), suspended
match constraints have a \emph{natural rule}:
\begin{mathpar}
  \infer[Match-Nat]
  {\sh = \shape {\semenv(\t)} \\ \semenv \th \cmatched \t \sh \cbrs}
  {\semenv \th \cmatch \t \cbrs}
\end{mathpar}
%
This rule states that a suspended constraint holds whenever the corresponding
discharged constraint holds for the canonical shape of $\semenv(\t)$.

\parcomment {The problem}

Although simple and declarative, this semantics is too
permissive. It allows \emph{guessing} the shape of $\t$ rather than requiring
it to be \emph{known}. For example, $\cexists \tv \cmatch \tv {\cbranch \cpatwild {\cunif
\tv \tint}}$ is satisfiable under the natural semantics:
\begin{mathpar}
\def \cmatchex {\cmatch \tv {\cbranch \cpatwild {\cunif \tv \tint}}}
\def \semenvex {\semenv\where{\tv \is \tint}}
    \infer*[Right=Match-Nat]
    {
      \cmatches \cpatwild \tint \eset \eset
      \\
      \infer*[Right=Unif]
        {\tint = \tint}
	% -------------------------------
    {\semenvex \th \cunif \tv \tint}
}{% ---------------------------------
    \infer*[Right=Exists]
    {\semenvex \th \cmatchex}
  % -----------------------------------
  {\semenv \th \cexists \tv \cmatchex}
}
\end{mathpar}
This ``out-of-thin-air'' behavior does not match the intended
meaning of suspended match constraints and raises several problems:
\begin{enumerate*}
  \item a reasonable solver---one that avoids backtracking---cannot
    be complete with respect to this semantics; and

  \item it breaks the existence of principal solutions, just
    as the \emph{natural} typing rules do (\eg \Rule{Rcd-Proj-I-Nat}).

\end{enumerate*}

\paragraph {Contextual semantics}

To rule out guessing, we instead adopt a \emph{contextual} semantics: a
match constraint is satisfiable only if the shape of the type is determined
by the surrounding context. The corresponding rule for suspended
constraints, \Rule {Match-Ctx} (\cref{fig:constraints}), is the only non-syntax-directed rule in our semantics.
%
\begin{mathpar}
  \infer[Match-Ctx]
    {\Cshape \C \t \sh \\
      \semenv \th \C \where {\cmatched \t \sh \cbrs}
    }
    {\semenv \th \C \where {\cmatch \t \cbrs}}
\end{mathpar}
%
In this rule, a suspended constraint $(\cmatch \t \cbrs)$ in the context
$\C$ can be discharged, provided the shape $\sh$ is not guessed from $\semenv$,
but recovered from the constraint context $\C$. This \emph{unicity} condition
$\Cshape \C \t \sh$ (defined below) ensures that $\sh$ is uniquely determined
by the context $\C$, capturing precisely what it means for the shape of a type
to be \emph{known}.

\begin{definition}[Erasure]
  \label{def:erasure}
  The erasure $\cerase \c$ of a constraint $\c$ is defined as the constraint
  obtained by replacing suspended match constraints in $\c$ with $\ctrue$.
\end{definition}

\begin{definition}[Simple constraints]
  We say that $\c$ is \emph{simple} if it contains no suspended match
  constraints.
\end{definition}

\begin{definition}[Unicity]
  \label{def:unicity}
  We define the unicity condition $\Cshape \C \t \sh$, which states that $\t$
  has a unique canonical shape $\sh$ within the context $\C$ as:
  \begin{mathpar}[inline]
    %% \Cshape \C \t \sh \Wide\eqdef
    \forall \semenv, \gt. \uad
      \semenv \th \cerase {\C\where{\cunif \t \gt}} \implies
          \shape \gt = \sh
  \end{mathpar}.
\end{definition}

\parcomment{Briefly explain unicity (using analogies to typing rules)}

Similarly to the unicity conditions introduced for typing rules in
\cref{sec/oml/typing/I}, unicity
for constraints $\Cshape \C \t \sh$ relies on \emph{erasure} $\cerase
{\C\where{\cunif \t \gt}}$ to restrict attention to previously discharged
match constraints, inducing a causal order between match constraints,
enforced by \Rule {Match-Ctx}.

\parcomment{The interesting cases of unicity}

When $\t$ is not a variable, $\Cshape \square \t \sh$ holds trivially when
$\sh$ is the shape of $\t$. Likewise, when (the erasure of) $\C$ is
unsatisfiable, then $\Cshape \C \tv \sh$ holds vacuously for any $\sh$. The
interesting and nontrivial cases---the ones we illustrate next---arise when
$\t$ is a type variable and $\C$ is satisfiable.

\parcomment {Examples}

\begin{example}
  Recall the problematic example that was satisfiable under the natural
  semantics:
\begin{mathpar}
  \cexists \tv \cmatch \tv {\cbranch \cpatwild {\cunif \tv \tint}}
\end{mathpar}
  Under the contextual semantics, the suspended constraint appears in a
  context $\C$
  with no contextual information. \ie $\C$ is $\cexists \tv \hole$.
  %
  So for any ground type $\gt$, $\cerase{\C\where{\cunif \tv \gt}}$ (\ie
  $\cexists \tv {\cunif \tv \gt}$) is satisfiable, allowing $\gt$ to have an
  arbitrary shape (\eg $\tint$, $\tbool$, \etc). As a result, the uniqueness
  condition $\Cshape \C \tv \sh$ never holds making \Rule{Match-Ctx}
  inapplicable. The constraint is unsatisfiable as intended.

\end{example}

\begin{example}
  Consider the satisfiable constraint:
\begin{mathpar}
\cexists \tv \cunif \tv \tint
  \cand
  \cmatch \tv {\cbranch \cpatwild \ctrue}
\end{mathpar}
  Here, we apply the contextual rule with the context $\C$ equal to $\cexists \tv
  \cunif \tv \tint \cand \hole$. Any solution $\semenv$ of this context
  necessarily satisfies \relax $\cunif \tv \tint$, so we have \relax $\Cshape
  \C \tv \tint$ and the suspended constraint can be discharged.
\end{example}
\begin{example}
  Consider the more intricate example:
\begin{mathpar}
  \cexists {\tva, \tvb}
  {\def \EX
     {\cmatch \tva {\cbranch \cpatwild {\cunif \tvb \tbool}} \and
      \cmatch \tvb {\cbranch \cpatwild \ctrue} \and
      \cunif \tva \tint}
    \False%\True
      {\def \and{\\{}\cand}\Parens
         {\begin{array}{;l}
            \quad \EX
          \end{array}}}
      {\def \and{)\wide\wedge(}\parens{\EX}}
  }
\end{mathpar}
  Suppose we attempt to apply \Rule{Match-Ctx} to the match on $\tvb$ first. We
  want to show $\Cshape \C \tvb \tbool$ for the context $\C$ equal to
  $(\cmatch \tv
  {\cbranch \cpatwild {\cunif \tvb \tbool}}) \cand \hole \cand \cunif \tva
  \tint$. Its erasure $\cerase \C$ is $\ctrue \cand \hole \cand \cunif \tva
  \tint$, which imposes no constraints on $\tvb$. Thus both \relax $\cerase
  {{\C\where{\cunif \tvb \tint}}}$ and \relax $\cerase {\C\where{\cunif \tvb
  \tbool}}$ are both satisfiable: unicity does not hold and \Rule{Match-Ctx}
  cannot be applied.

  By contrast, if we first discharge the match on $\tva$, we consider the
  context $\C$ equal to $\hole \cand (\cmatch \tvb {\cbranch \cpatwild
  \ctrue}) \cand
  \cunif \tva \tint$. Its erasure $\cerase \C$ equal to $\hole \cand \ctrue \cand
  \cunif \tva \tint$ does constraint $\tva$, giving $\Cshape \C \tva \tint$.
  %
  We may therefore discharge the match on $\tva$, rewriting it as
  $(\cmatched \tva \tint {\cbranch \cpatwild {\cunif \tvb \tbool}})$ \ie
  $\tva = \tint \cand \cunif \tvb \tbool$. Substituting back, we
  are left to satisfy the constraint $\C \where {\tva = \tint \cand \cunif \tvb \tbool}$ \ie
\begin{mathpar}[inline]
  \cunif \tva \tint \cand \cunif \tvb \tbool \cand
  (\cmatch \tvb {\cbranch \cpatwild \ctrue}) \cand
  \cunif \tva \tint
\end{mathpar}.
  %
  At this point, unicity for $\tvb$ holds, since the context now includes $\cunif
  \tvb \tbool$. We can therefore apply \Rule{Match-Ctx} to eliminate the final
  match constraint.

  This example demonstrates that suspended constraints must be resolved in
  a dependency-respecting order: attempting to resolve a match constraint too
  early may result in unsatisfiability.
\end{example}

\begin{example}
  Let us consider a constraint with a cyclic dependency between match
  constraints:
\begin{mathpar}
  \cexists {\tva, \tvb}
  {\def \EX
     {\cmatch \tva {\cbranch \cpatwild {\cunif \tvb \tbool}} \and
      \cmatch \tvb {\cbranch \cpatwild {\cunif \tva \tint}}}
    \False%\True
      {\def \and{\\{}\cand}\Parens
         {\begin{array}{;l}
            \quad \EX
          \end{array}}}
      {\def \and{)\wide\wedge(}\parens{\EX}}
  }
\end{mathpar}
  Under the natural semantics this constraint is satisfiable, since one may
  \emph{guess} the assignment $\tva \is \tint, \tvb \is \tbool$,
  making both match constraints succeed. However, our solver and contextual semantics
  reject it.


  Without loss of generality, suppose we attempt to apply \Rule{Match-Ctx} on
  $\tva$ first. We must establish $\Cshape \C \tva \tint$ for the context $\C
  \is \hole \cand \cmatch \tvb {\cbranch \cpatwild {\cunif \tva \tint}}$. But
  the erasure $\cerase \C$ is $\hole \cand \ctrue$, which imposes no constraint on
  $\tva$. Hence unicity fails and \Rule{Match-Ctx} is inapplicable.

\end{example}
\subsection{Constraint generation}
\label{sec/oml/constraints}

\begin{mathparfig}
  {fig:patterns-oml}
  {Patterns for \OML.}
  \begin{bnfgrammar}
   \entry[Record variables]{\ct}{}\\
   \entry[Scheme variables]{\cscm}{}\\
   \entry[Shape patterns]{\cpat}{
      \ldots
      \and \cpatrcd \ct
      \and \cpatpoly \cscm
    } \\[1ex]
    \entry[Constraints]{\c}{
      \dots
      \and \labfrom \elab \ct \leq \ta \to \tb
      \and \dom \ct = \elabs
      \and \cscm \leq \t
      \and \x \leq \cscm
    }
 \end{bnfgrammar}
  \\
  \def \>{\noalign{\vskip .8ex}}
  \newcommand{\Mrule}[5][]{{#2} \Matches {(#3)} \; #4 &\eqdef& {#5} & #1}
  \begin{tabular}{RCLL}
    \Mrule
      {(\cpatrcd \ct)}
      {\any \tvcs \trcd \T \tvcs} \tvcps
      {[\ct \is \T]}
    \\\>
    \Mrule
      {\cpatpoly \cscm}
      {\any \tvcs \tpoly \ts} \tvcps
      {[\cscm \is \ts \where{\tvcs \is \tvcps}]}
  \end{tabular}
  \\
  \newcommand{\Srule}[3][]{{#2} &\eqdef& {#3} & {#1}}
  \newcommand{\Scases}[3]{%
    \left\{
      \begin{array}{l}
      #1\\%[0.4ex]
      #2
      \end{array}%
    \right.
    &
    \hspace{-1ex}\begin{array}{l}
      \text{if } #3\\%[0.4ex]
      \text{otherwise}
    \end{array}%
  }
  \begin{tabular}{RCLL}
    \labfrom \elab \T \leq \ta \to \tb &\eqdef&
      \Scases{\cexists \tvs \cunif \ta  {\trcd \T \tvs} \cand \cunif \tb \t}{\cfalse}{\labenv(\labfrom \elab \T) = \tfor \tvs {\trcd \T \tvs \to \t}}
    \\\>
    \dom{\T} = \elabs &\eqdef&
       \Scases{\ctrue}{\cfalse}{\Dom {\labenv(\T)} = \elabs}
    \\\>
    \Srule
      {(\tfor \tvs \tp) \leq \t}
      {\cexists \tvs \cunif \tp \t}
    \\\>
    \Srule
      {\x \leq (\tfor \tvs \t)}
      {\cfor \tvs \capp \x \t}
  \end{tabular}
\end{mathparfig}

\begin{mathparfig}
  [htpb!]
  {fig/constraint-gen}
  {The constraint generation translation for \OML.}
\newcommand {\Crule}[2]{#1 &\eqdef& #2}
\def \arraystretch{1.1}%4
\begin{tabular}{LCL}
\Crule
   {\cinfer x \t}
   {\cinst x \t}
\\
\Crule
  {\cinfer {()} \t}
  {\cunif \t \tunit}
\\
\Crule
  {\cinfer {\efun \x \e} \t}
  {\cexists {\tva, \tvb}
    \clet \x \tvp {\cunif \tvp \tv} {\cinfer \e \tvb}
    \cand \cunif \t {\tva \to \tvb}}
\\
\Crule
  {\cinfer {\eapp \ea \eb} \t}
  {\cexists {\tva, \tvb}
    \cinfer \ea {\tvb} \cand \cinfer \eb \tva
    \cand \cunif \tvb {\tva \to \t}}
\\
\Crule
  {\cinfer {\elet \x \ea \eb} \t}
  {\clet \x \tv {\cinfer \ea \tv} {\cinfer \eb \t}}
\\
\Crule
  {\cinfer {\eannot \e \tvs \tp} \t}
  {\cexists \tvs  \cinfer \e \tp \cand \cunif \t \tp}
\\
\Crule
  {\cinfer {\expoly \e \tvs \ts} \t}
  {\cexists {\tvs}
    \cinfer \e \ts
    \cand \cunif \t {\tpoly \ts}}
\\
\Crule
  {\cinfer {\exinst \e \tvs \ts} \t}
  {\cexists {\tvs, \tvb}
    \cinfer \e \tvb
    \cand \cunif \tvb {\tpoly \ts}
    \cand \ts \leq \t}
\\
\Crule
  {\cinfer {\einst \e} \t}
  {\cexists \tv
    \cinfer \e \tv
    \cand \cmatch \tv {\cbranch {\cpatpoly \cscm} \cscm \leq \t}}
\\
\Crule
  {\cinfer {\epoly \e} \t}
  {\clet \x \tv {\cinfer \e \tv}
    {\cmatch \t {\cbranch {\cpatpoly \cscm} {\x \leq \cscm}}}}
\\
\Crule
  {\cinfer {\efield \e \elab} \t}
  {\begin{cases}
    \cinfer {\exfield \e \T \elab} \t
    & \text{if } \labuni \elab \T
    \\
    \cexists \tv \cinfer \e \tv \cand
    \cmatch \tv
      {\cbranch {\cpatrcd \ct} {\labfrom \elab \ct \leq \tv \to \t}}
    & \text{otherwise}
  \end{cases}}
\\
\Crule
  {\cinfer {\exfield \e \T \elab} \t}
  {\cexists \tv \cinfer \e \tv \cand
   \labfrom \elab \T \leq \tv \to \t}
\\
\Crule
  {\cinfer {\erecord {\overline{\elab = \e}}} \t}
  {\begin{cases}
    \cinfer {\exrecord \T {\overline{\elab = \e}}} \t
    & \text{if } \labsuni \elabs \T \\
    \cexists \tvs \cAnd\iton \cinfer \ei \tvi & \text{otherwise} \\
    \uad\cand\uad \cmatch \t {\cbranch {\cpatrcd \ct}
      {\parens{\dom \ct = \elabs \cand \cAnd\iton \labfrom \elabi \ct \leq \t \to \tvi}}} &
   \end{cases}}
\\
\Crule
  {\cinfer {\exrecord \T {\overline{\elab = \e}}} \t}
  {\cexists \tvs \cAnd\iton \cinfer \ei \tvi \cand \dom {\T} = \elabs \cand \cAnd\iton \labfrom \elabi \T \leq \t \to \tvi }
\\
\Crule
  {\cinfer {\emagic \es} \t}
  {\cexists \tvs \cAnd\iton \cinfer \ei \tvi}
\\\\
\Crule
  {\cinfer \e {\tfor \tvs \t}}
  {\cfor \tvs \cinfer \e \t}
\end{tabular}
\end{mathparfig}


% Intro
We now present the formal translation from terms $\e$ to constraints $\c$,
such that the resulting constraint is satisfiable if and only if the term is
well typed. The translation is defined as a function $\cinfer \e \t$, where $\e$
is the term to be translated and $\t$ is the expected type of $\e$.
%
The expected type $\t$ is permitted to contain type variables, which can be
existentially bound in order to perform type inference. The models of
constraint $\cinfer \e \t$ interpret the free variables of
$\t$ such that $\t$ becomes a valid type of $\e$. For example, to infer the
entire type of $\e$ we may pick a fresh type variable $\tv$ for $\t$.


\paragraph{Shape patterns}

Thus far, our formal presentation of shape patterns has remained
abstract, deliberately leaving the syntax and semantics of match constraints
partially unspecified to accommodate a range of language features. We now
concretize this by specifying the shape patterns used in \OML
(see \cref{fig:patterns-oml}), and introducing the corresponding constraints
for the variables they bind.
%
Shape patterns include:
% This is a important part of prose. It deserves more space (and thus an enumerate
% over enumerate*)
\begin{enumerate}

  \item Record type patterns $\cpatrcd \ct$, binding the name $\T$
    of a record type $\trcd \T \tys$ to the record variable $\ct$.

  \item Polytype patterns $\cpatpoly \cscm$ matching a polytype $\tpoly \ts$ and
    binding the resulting scheme to the variable $\cscm$.

\end{enumerate}

Each new kind of pattern introduces corresponding constraint formers:
$\labfrom \elab \ct \leq \ta \to \tb$ asserts that $\ta \to \tb$ is a
instance of projection type associated with the explicit label $\labfrom
\elab \ct$; $\dom \ct = \elabs$ ensures that the domain of record type
$\ct$ is the set of labels $\elabs$; and $\cscm \leq \t$ checks that $\t$ is
an instance of $\cscm$, while $\x \leq \cscm$ asserts that every instance of
$\x$ is an instance of $\cscm$.

By definition, each of these constraint forms is unsatisfiable, since we do not
extend the satisfiability relation to include them. Their role is purely
syntactic: they only appear within the branches of a suspended match
constraint. Once such a constraint is discharged $(\cmatched \t \sh {\cbranch
\cpats \cs})$ (\cref {def/discharged}), the substitution $\theta$ produced by
the matching pattern $\cpati$---which binds the pattern variables (\eg
$\ct$, $\cscm$, \etc)---is
applied to the corresponding branch $\ci$. Hence, it suffices to define
the semantics of these constraint formers only for their substituted forms (\eg
$\labfrom \elab \T \leq \ta \to \tb$, $\dom {\T} = \elabs$, \etc), as shown in
\cref{fig:patterns-oml}. We define the semantics of these substituted forms by
translation into existing constraints---that is, as syntactic sugar over the
existing core constraint language.

\paragraph{Constraint generation}

Constraint generation $\cinfer \e {\mathop{\t}}$ is defined in
\cref{fig/constraint-gen}. All generated type variables are fresh with respect
to the expected type $\t$, ensuring capture-avoidance. We now review the cases
of the constraint generator, beginning with the cases corresponding to
traditional \ML constructs.

\parcomment{Simple \ML terms}

Unsurprisingly, variables $\x$ generate an instantiation constraint $\capp \x
\t$. The unit term $\eunit$ requires $\t$ to be the unit type $\tunit$. A
function generates a constraint that binds two fresh flexible type variables
for the parameter and return types.  We use a $\Let$ constraint to bind the
parameter in the constraint generated for the body of the function. The $\Let$
constraint is monomorphic since $\tvp$ is fully constrained by type variables
defined outside the abstraction's scope and therefore cannot be generalized.
Applications introduce two fresh flexible type variables, one for the argument
type and one for the type of the function, typing each subterm with these,
ensuring $\t$ is the expected return type. Let-bindings generate a polymorphic
let constraint; $\cabs \tv {\cinfer \e \tv}$ is a principal constraint
abstraction for $\e$: its intended interpretation is the set of all types that
$\e$ admits. Annotations bind their flexible type variables and enforce the
equality of the annotated type $\tp$ and the expected type $\t$.


\parcomment{Polytypes}

For polytypes, explicit boxing asserts that $\e$ has the polymorphic type $\ts$
(using universal quantification) and that the expected type is the polytype
$\tpoly \ts$. Conversely, explicit unboxing requires that $\t$ be an instance
of $\ts$. These cases introduce no suspended match constrains because the
annotations can the required type information: the polytype $\tpoly \ts$.

By contrast, implicit unboxing suspends until the inferred type of $\e$ is
known to be some non-variable type $\tz$. At that point, the suspended match
constraint attempts to match $\tz$ against the pattern $\cpatpoly \cscm$. If
$\tz = \tpoly \ts$, the match succeeds, binding $\ts$ to $\cscm$ and requiring
$\t$ to be an instance of $\cscm$ (\ie $\ts \leq \t$); otherwise, the match
fails and the term is ill-typed.

Implicit boxing behaves dually: we infer the principal type for $\e$ using a
$\Let$ constraint and suspend until the expected type of the entire term
resolves to some type $\tz$. If $\tz$ matches the same pattern $\cpatpoly
\cscm$ (\ie $\tz = \tpoly \ts$), we assert that the principal type of $\e$ is
at least as general as $\cscm$, via the constraint $\x \leq \cscm$ (\ie $\x
\leq \ts$); otherwise, the match fails.

\parcomment{Records}

% Records
Record constructs are handled in a similar way, but their matching patterns
concern record shapes rather than polytypes. A record projection generate a
fresh variable $\tv$ for the record type and constrain $\e$ to this type,
suspending until the type of $\e$ ($\tv$) resolves to some type $\tz$. When
$\tz = \trcd \T \tys$, the shape pattern $(\cpatrcd \ct)$ matches successfully,
binding $\T$ to the record name variable $\ct$; the matching branch then
retrieves the projected label's type from the global label context $\labenv$
and instantiates it to match $\tv \to \t$. If $\tz$ is not a
record type, the match fails and the term is ill-typed.

For record expressions, we generate a fresh variable $\tvi$ for each field
assignment to capture the type of each $\ei$ as $\tvi$. The rest of the
constraint is deferred until the context determines the type of the whole
record type to be $\tz$. Once known, it is matched against the same pattern
$\parens{\cpatrcd \ct}$. If the match succeeds (\ie $\tz = \trcd \T \tys$ and
$\ct = \T$) the labels are instantiated to match the projection types $\t \to
\tvi$, and we additionally check that the domain of $\ct$ is exactly $\elabs$,
ensuring that every label is defined. Otherwise, the constraint is
unsatisfiable as intended.

Explicit records and projections, along with closed-world disambiguated terms,
bypass suspension and directly instantiate the appropriate labels.

\begin{example}
Considering the example from \cref{sec/overview/overloading}:
\begin{program}[input]
  let ex_4 r = let x = r.x in x + (r : point).y °\ocamlflags 10°
\end{program}
The typing constraint generated for \code{ex_4} contains the following,
where $\tv$ stands for the type of \code{r}:
\begin{mathpar}
  \cexists {\tv}
    \clet x \tvb
      {(\cmatch \tva \dots)}
      {\cinst x \tint \cand \cunif \tv {\ttpoint}}
\end{mathpar}
The suspended constraint can be discharged under our contextual semantics.
We apply the \Rule{Match-Ctx} rule with context
$\C$ equal to
\begin{mathpar}[inline]
  \clet \x \tvb \hole \capp \x \tint \cand
  \cunif \tv \ttpoint
\end{mathpar}.
Although the context includes a \Let-binding---which in practice
involves \Let-generalization---we can still deduce $\Cshape \C \tv
  {\ttpoint}$, since the erased context $\cerase
\C$ contains the unification constraint $\cunif \tv \ttpoint$.

This example illustrates that our formulation of suspended constraints
interacts nicely with \Let-polymorphism. Although the two features are
specified in a modular fashion, they are carefully crafted to work together,
as we further show in our next example.
\end{example}

\begin{example}\label{ex:backprop}
A subtle yet crucial feature of our semantics is its support for
\emph{backpropagation}:
\begin{program}[input,checkocaml]
let ex_1_1 = let getx r = r.x in getx one  °\ocamlflags 10°
\end{program}

As in the previous example, the type of \code{r} cannot be disambiguated from
the \Let-definition alone. There, the ambiguity was resolved when the type
($\tv$) was unified to a known type ($\ttpoint$) in the \Let-body. Here, the
situation is more subtle: an \emph{instance} of \code{getx}'s type scheme is
taken, which only satisfies the application (\ocaml!getx one!) if \code{r}
has either a variable type or the record type \ocaml{point}.  However, the
projection \code{r.x} would be ill-typed if \code{r} had a variable type
(unicity would fail), so \code{point} is the unique consistent solution. This
information therefore flows back from the instances in the \Let-body to the
\Let-definition, a phenomenon we call \emph{backpropagation}.

The constraint generated when typing
\code{ex_1_1} is:
\begin{mathpar}
\begin{tabular}{L.L}
  \cexists \tv {}
  &\clet {getx} \tvd
     {\cexists {\tvb, \tvc} \Parens {\strut
        \cunif \tvd {\tvb \to \tvc} \cand
	\cmatch \tvb \dots
        }}{}
    \cinst {getx} {(\ttpoint \to \tv)}
\end{tabular}
\end{mathpar}
With the context $\C$ equal to $\clet {getx}
\tvd {\cexists {\tvb, \tvc} \cunif \tvd {\tvb \to \tvc} \cand \hole}
{\capp {getx} {(\ttpoint \to \tv)}}$, we can show that the unicity
predicate $\Cshape \C \tvb \ttpoint$ holds.
For any $\gt$, the erasure $\cerase {\C \where {\cunif \tvb \gt}}$
is
\begin{mathpar}[inline]
\clet {getx}
\tvd {\cexists {\tvb, \tvc} \cunif \tvd {\tvb \to \tvc} \cand \cunif \tvb \gt}
{\capp {getx} {(\ttpoint \to \tv)}}
\end{mathpar}.
Since $getx$ is bound to the constraint abstraction
\relax $\cabs \tvd \exists \tvc.\uad \cunif \delta {(\gt \to \gamma)}$,
the instantiation
\relax $\capp {getx} (\ttpoint \to \tv)$
can only be satisfied when $\gt$ is equal to $\ttpoint$. This proves unicity,
hence the generated constraint for \code{ex_1_1} is satisfiable.
\end{example}

\subsection{Metatheory}
\label{sec/oml/metatheory}

Constraint generation is sound and complete with respect to the typing judgment:
%
\begin{theorem}[Constraint generation is sound and complete]
  \label{thm:constraint-gen-is-sound-and-complete}
A closed \OML term $\e$ is typable if and only if
the constraint $\cexists \tv {\cinfer \e \tv}$ is satisfiable.
\end{theorem}

\begin{restatable}[Principal types]{theorem}{principalTypesBIS}
\label{thm:principal-types}
For any well-typed closed \OML term $\e$, there exists a type $\t$
%% , which we call principal,
such that:
\begin{enumerate*}[(\roman*)]
\item
  $\th \e : \t$.
\item
  For any other typing $\th \e : \tp$, then $\tp = \theta(\t)$ for some
  substitution $\theta$.
\end{enumerate*}
\end{restatable}

\begin{version}{}
It is also interesting to discuss the stability of typing by common program
transformations.

\paragraph{Application equi-typability does hold}

\newcommand {\eswap}{{\mathprefix {swap}}}
The expressions $\eappp f \ea \eb$ and $\eappp {\eswap f} \eb \ea$ are
equitypable where $\eswap$ is
$\efun f {\efun \xa {\efun \xb {\eappp f \xb \ea}}}$. We also have
that $\eapp f \e$ and $\eappp {\mathsf{app}} f \e$ and
$\eappp {\mathsf{rev\_app}} \e f$ are equitypable.
\begin{version}{}
\XDR{Definition already made in the intro as a footnote.}
, where $\mathsf{app}$ and $\mathsf{rev\_app}$ are the application function
$\efun f \efun \x \eapp f \x$ and the reverse application function $\efun \x
\efun f \eapp f \x$, respectively.
\end{version}
It is well-known that
bidirectional types inference breaks application
equitypability. Both \geninst-directional and omnidirectional
type inference preserve it.

\paragraph{Factorization does not hold} If $\G \th \e \where {\x \is {\e_0}}
: \t$ with $\x$ appearing in $\e$, we do not necessarily have $\G \th \elet \x
\ez \e : \t$. This is not a defect of our system, but a general property of
all systems that support static overloading: the expanded term $\e \where {x
\is \e_0}$ can pick a different overloading choice for each occurrence of
$\e_0$, and if they are incompatible the factored form may not typecheck.

\paragraph{Inlining does not hold} If $\G \th \elet \x \ez \e : \t$, we do
not necessarily have $\G \th \e \where {\x \is {\e_0}} : \t$. This is
specific to our support for \emph{backpropagation}: the \Let-form will use
information from all occurrences of $\x$ in $\e$ to resolve fragile
constructs in $e_0$, but in the inlined form each copy of $\e$ must resolve
its implicit constructs independently, and it has access to less information
to establish unicity. As a result, the implicit \OML calculus does not
preserve typability in its operational semantics.

\end{version}

\section{Solving constraints}
\label{sec:solving}


% DEMONS BE HIDDEN HERE: This figure must begin at the top. The collection of
% subfigures is carefully crafted to span ~1 1/2 pages. There is some hacking
% to avoid LaTeX's float queue getting stuck (and postponing the figures till
% the very end). We rely on \clearpage to flush the float queue. To avoid this
% being fragile, we use \afterpage (which executes it contents after a
% pagebreak)

%% We can fit more on the first page: switch flags to undo.

\let \More \False

\afterpage{
\clearpage
\begin{figure}[!tp]
\begin{mathparsubfig}
  {fig:unification-syntax-and-semantics}
  {Syntax and semantics of unification problems.}
  \begin{minipage}[l]{0.7\textwidth}%
    \begin{bnfgrammar}[\noleftfill]%
      \entry[Unification problems]{\up}{
        \ctrue \and \cfalse \and \upa \cand \upb \and \cexists \tv \up \and \ueq
        \uad\strut
      } \\
      \entry[Multi-equations]{\ueq}{
        \eset \mid \cunif \t \ueq
      } \\
      \entry[Constraints]{\c}{
        \dots \and \ueq
      } \\
      \entry[Unification context]{\Up}{
        \hole
        \and \Up \cand \upb
        \and \upa \cand \Up
        \and \cexists \tv \Up
      }
    \end{bnfgrammar}
    \hfill
  \end{minipage}
  \hfill
  \vcenter{\hbox{
    \infer[Multi-Unif]
      {\all {\t \in \ueq}\, \semenv(\t) = \gt}
      {\semenv \th \ueq}
  }}
\end{mathparsubfig}

\medskip

\begin{mathparsubfig}
  {fig:unification-algorithm}
  {Unification algorithm as a series of rewriting rules
   $\upa \unif \upb$. All shapes are principal.}
   \rewrite[U-Exists]
      {(\cexists \alpha \upa) \cand \upb }{ \tv \disjoint \upb}
      {\cexists \tv {\upa \cand \upb}}

    \rewrite[U-Cycle]
      {\up }{ \cyclic \up}
      {\cfalse}

    \rewrite[U-True]
      {\up \cand \ctrue}
      {}
      {\up}

    \rewrite[U-False]
      {\Up\where\cfalse }{ \Up \neq \hole}
      {\cfalse}

    \rewrite[U-Merge]
      {\cunif \tv \ueqa \cand \cunif \tv \ueqb}
      {}
      {\cunif \tv {\cunif \ueqa \ueqb}}

    \rewrite[U-Stutter]
      {\cunif \tv {\cunif \tv \ueq}}
      {}
      {\cunif \tv \ueq}

    \rewrite[U-Name]
      {\cunif {\pshapp \parens{\tys, \ti, \typs}} \ueq }
      { \tv \disjoint \tys, \typs, \ueq \\ \ti \notin \TyVars}
      {\cexists \tv {\cunif \tv \ti \cand \cunif {\pshapp \parens{\tys, \tv, \typs}} \ueq}}

    \rewrite[U-Decomp]
      {\cunif {\pshapp \tvs} {\cunif {\pshapp \tvbs} \ueq}}
      {}
      {\cunif {\pshapp \tvs} \ueq \cand \cunif \tvs \tvbs}

    \rewrite[U-Clash]
      {\cunif {\pshapp \tvs} {\cunif {\pshapp[\shp]\tvbs } \ueq }}{
       \sh \neq \shp}
      {\cfalse}

    \rewrite[U-Trivial]
      {\ueq }
      {|\ueq| \leq 1}
      {\ctrue}
\end{mathparsubfig}

\medskip

\begin{mathparsubfig}
  {fig:solver-basic}
  {Basic rewriting rules $\ca \csolve \cb$.}
  \rewrite[S-Unif]
    {\upa}
    {\upa \unif \upb}
    {\upb}

  \rewrite[S-False]
    {\C\where\cfalse}
    {\C \neq \hole}
    {\cfalse}

  \rewrite[S-Let]
    {\clet \x \tv \ca \cb}
    {}
    {\cletr \x \tv \eset \ca \cb}

  \rewrite[S-Exists-Conj]
    {(\cexists \alpha \ca) \cand \cb}
    {\tv \disjoint \cb}
    {\cexists \tv {\ca \cand \cb}}

  \rewrite[S-Let-ExistsLeft]
    {\cletr \x \tv \tvs {\cexists \tvb \ca} \cb}
    {\tvb \disjoint \tv, \tvs, \cb}
    {\cletr \x \tv {\tvs, \tvb} \ca \cb}

  \rewrite[S-Let-ExistsRight]
    {\cletr \x \tv \tvs \ca {\cexists \tvb \cb}}
    {\tvb \disjoint \tv, \tvs, \ca}
    {\cexists \tvb {\cletr \x \tv \tvs \ca \cb}}

  \rewrite[S-Let-ConjLeft]
    {\cletr \x \tv \tvs {\ca \cand \cb} \cc}
    {\ca \disjoint \tv, \tvs}
    {\ca \cand \cletr \x \tv \tvs \cb \cc}

  \rewrite[S-Let-ConjRight]
    {\cletr \x \tv \tvs \ca (\cb \cand \cc)}
    {\x \disjoint \cb}
    {\cb \cand \cletr \x \tv \tvs \ca \cc}
\end{mathparsubfig}

\medskip

\begin{mathparsubfig}
  {fig:constraint-let-regions}
  {Syntax and semantics of region-based $\Let$ and
   incremental instantiation constraints.}
  \begin{bnfgrammar}
    \entry[Instantiation variables]{\inst}{}\\
    \entry[Constraints]{\c}{
       \dots \let \and \wand
       \and \cletr \x \tv \tvs \ca \cb
       \and \cexistsi \inst \x \c
       \and \cpinst \inst \tv \t
    }\\[1ex]
    \entry[Ground regions]{\gr}{
      \greg \tv \semenv \qquad\qquad\! \parens{\gr \in \GroundRegion}
    }\\
    \entrysubseteq[Sets of ground regions]{\gabsr}
      {\GroundRegion}\\
    \entry[Semantic environments]{\semenv}{
      \dots \let \and \wand
      \and \semenv\where{\x \is \gabsr}
      \and \semenv\where{\inst \is \semenvp}
    }
  \end{bnfgrammar}
  \par
  \semenv(\cabsr \tv \tvs \c) \uad\eqdef\uad
    \set {\greg \tv {\semenv\where{\tv \is \gt, \tvs \is \gts}} \in \GroundRegion
         : \semenv\where{\tv \is \gt, \tvs \is \gts} \th \c
         }
  \par
  \infer[LetR]
    {\semenv \th \cexists {\tv, \tvs} \ca \\\\
     \semenv\where{\x \is \semenv(\cabsr \tv \tvs \ca)} \th \cb}
    {\semenv \th \cletr \x \tv \tvs \ca \cb}

  \infer[AppR]
    {\greg \tv \semenvp \in \semenv(\x) \\\\
     \semenv(\t) = \semenvp(\tv) }
    {\semenv \th \capp \x \t}

  \infer[Exists-Inst]
    {\greg \tv\semenvp \in \semenv(\x) \\\\
     \semenv\where{\inst \is \semenvp} \th \c}
    {\semenv \th \cexistsi \inst \x \c}

  \infer[Incr-Inst]
    {\semenv(\inst)(\tv) = \semenv(\t) }
    {\semenv \th \cpinst \inst \tv \t}
\end{mathparsubfig}
%% This figure is too high for the page.
%% This is to say that 1pt is ok and prevent a warning.
\vskip -1pt
\end{figure}

% This is the start of the next subfigure, but it is placed
% in a separate figure to span multiple pages

\clearpage
\begin{figure}[!t]\ContinuedFloat
\begin{mathparsubfig}
  {fig:solver-schemes}
  {Solving rules for let-bindings and instantiations.}

\rewrite[S-Inst-Copy]
  {\cletr \x \tv \tvs {\c} \C\where{\cpapp \x \tvb \tvc \inst}}
   {\acyclic {\c} \\\\
    \x \disjoint \bvs \C \\
    \c = \cp \cand \cunif \tvb {\cunif {\shapp \tvbs} \ueq}\\
    \tvb \in \reg \tv \tvs \\
    \tvbs' \disjoint \tvb, \tvc, \tvbs}
  {\cletr \x \tv \tvs {\c}
    \C\where{\cexists {\tvbs'}
        \cunif \tvc {\shapp \tvbs'} \cand \cpapp \x {\tvbs} {\tvbs'} \inst}}

\rewrite[S-Inst-Unify]
  {\cpinst \inst \tvb \tvca \cand \cpinst \inst \tvb \tvcb}
  {}
  {\cpinst \inst \tvb \tvca \cand \cunif \tvca \tvcb}

\rewrite[S-Inst-Poly]
  {\cletr \x \tv {\tvs} {\ueqs \cand \c}
      {\C\where{\cpapp \x \tvp \tvc \inst}}}
  {\cfor \tvp \cexists {\tv} {\ueqs} \cequiv \ctrue \\\\
   \tvp \in \reg \tv \tvs \\
   \tvp \disjoint \c \\
   \inst.\tvp \disjoint \insts \C \\
   \x \disjoint \bvs \C}
  {\cletr \x \tv {\tvs} {\ueqs \cand \c} {\C\where\ctrue}}

\rewrite[S-Inst-Mono]
  {\cletr \x \tv \tvs \c {\C\where{\cpapp \x \tvb \tvc \inst}}}
  {\tvb \notin \reg \tv \tvs \\ \x, \tvb \disjoint \bvs \C}
  {\cletr \x \tv \tvs \c {\C\where{\cunif \tvb \tvc}}}

\rewrite[S-Let-AppR]
  {\cletr \x \tv \tvs \c {\C\where{\capp \x \t}}}
  {\tvc \disjoint \t \\
   \x \disjoint \bvs \C}
  {\cletr \x \tv \tvs \c
     {\C\where{\cexistsi {\tvc, \inst} \x
              {\cpinst \inst \tv \tvc \cand \cunif \tvc \t}}}}

\rewrite[S-Let-Solve]
  {\cletr \x \tv \tvs \ueqs \c\\}
  {\cexists {\tv, \tvs} \ueqs \cequiv \ctrue \\
   \x \disjoint \c}
  {\c}

\rewrite[S-Compress]
  {\cletr \x \tv {\tvs, \tvb} {\ca \cand \cunif \tvb {\cunif \tvc \ueq}} {\cb}}
  {\tvb \neq \tvc}
  {\cletr \x \tv {\tvs}
     {\ca\where{\tvb \is \tvc}
      \cand \cunif \tvc {\ueq\where{\tvb \is \tvc}}}
     {\cb\where{\x.\tvb \is \tvc}}}

\rewrite[S-Exists-Lower]
  {\cletr \x \tv {\tvas, \tvbs} \ca \cb\\}
  {\th \cdetermines {\cexists {\tv, \tvas} \ca} \tvbs}
  {\cexists \tvbs \cletr \x \tv \tvas \ca \cb}

\inferrule
  [Det-Dom]
  {\tvc \disjoint \tvbs, \tvas \\ \tvs \subseteq \fvs \ueq}
  {\th \cdetermines {\cexists \tvbs \c \cand \cunif \tvc \ueq} \tvs}

\inferrule
  [Det-Esc]
  {\fvs \t \disjoint \tvs, \tvbs}
  {\th \cdetermines {\cexists \tvbs \c \cand \cunif \tvs {\cunif \t \ueq}} \tvs}
\end{mathparsubfig}

\medskip

\begin{mathparsubfig}
  {fig:solver-susp}
  {Rewriting rules for suspended match constraints.}
\rewrite[S-Match-Ctx]
  {\C\where{\cmatch \t \cbrs}}
  {\th \Cshape \C \t \sh}
  {\C\where{\cmatched \t {\sh} \cbrs}}

\infer[Uni-Var]
  {\color{gray}\tv \disjoint \bvs \Cb}
  {\th \Cshape {\Ca\where{\cunif \tv
     {\cunif \t \ueq} \cand \Cb\where{\hole}}} \tv {\shape \t}}

\infer[Uni-Type]
  {{\color{gray}\t \notin \TyVars}}
  {\th \Cshape \C \t {\shape \t}}

\infer[Uni-BackProp]
  {\th \Cshape{\parens{\cletr \x \tv \tvs {\Ca\where{\ctrue}}
  {\Cb\where{\cpapp \x \tvp \tvc \inst \cand \hole}}}} \tvc \sh \\
   \color{gray}\tvp \in \tv, \tvs \\
   \color{gray}\x \disjoint \bvs \Cb \\
   \color{gray}\tvp \disjoint \bvs \Ca}
  {\th \Cshape
     {\parens{\cletr \x \tv \tvs {\Ca\where{\hole}}
                  {\Cb\where{\cpapp \x \tvp \tvc \inst}}}}
     \tvp \sh}
\end{mathparsubfig}
\end{figure}
}
% ends \afterpage

\parcomment{Intro}

We now present a machine for solving constraints in our language. The solver
operates as a rewriting system on constraints $\c \csolve \cp$. Once no
further transitions are applicable, \ie $\c \cnsolve$, the constraint $\c$
is either a solved form---from which we can read off a most general
solution---or unsatisfiable (if the constraint
$\c$ is closed).

\subsection{Unification}

Our constraints ultimately reduce to equations between types, which we solve
using first-order unification. Like our solver, we specify unification as a
non-deterministic rewriting relation between \emph{unification problems} $\upa
\unif \upb$, that eventually reduces to a solved form $\hat\up$ or to $\cfalse$.

\parcomment{Unification problems and multi-equations}

Unification problems $\up$
(\cref{fig:unification-syntax-and-semantics}) are a restricted subset
of constraints, extended with \emph{multi-equations}
\citep*{Pottier-Remy/emlti}---a multi-set of types considered
equal. These generalize binary equalities: $\semenv$ satisfies
a multi-equation $\ueq$ if all of its members are mapped to a single
ground type $\gt$ (\Rule{Multi-Unif}). Multi-equations are
considered equal modulo permutation of their members.

The unification rules are listed in \cref{fig:unification-algorithm}. Rewriting
proceeds under an arbitrary context $\Up$, modulo $\alpha$-equivalence and
associativity and commutativity of conjunctions.
%
Our algorithm is largely standard~\citep*{Pottier-Remy/emlti}, with its main
novelty being the use of \emph{canonical principal shapes} in place of type
constructors. This uniform treatment of monotypes and polytypes simplifies
unification and improves on the previous treatment of polytype unification
\citep{Garrigue-Remy/poly-ml}.

\parcomment{Explanation of the rules}

We briefly summarize the role of each rule. \Rule{U-Exists} lifts existential
quantifiers, enabling applications of \Rule{U-Merge} and \Rule{U-Cycle} since
all multi-equations eventually become part of a single conjunction.
\Rule{U-Merge} combines multi-equations sharing a common variable and
\Rule{U-Stutter} removes duplicate variables. \Rule{U-Decomp} decomposes equal
types with matching shapes into equalities between their subcomponents, while
\Rule{U-Clash} detects shape mismatches that result in failure. \Rule{U-Name}
introduces fresh variables for subcomponents, ensuring unification operates on
\emph{shallow terms}, making sharing of type variables explicit and avoiding
copying types in rules such as \Rule{U-Decomp}. \Rule{U-True} and
\Rule{U-Trivial} eliminate trivial constraints, and \Rule{U-False} propagates
failure.
%
Finally, \Rule{U-Cycle} implements the \emph{occurs check}, ensuring that a
type variable does not occur in the type it is being unified with. This is a
necessary condition for unification, as it would otherwise lead to infinite
types.
This is formalized by the relation $\tv \prec_\up \tvb$ indicating that
$\tv$ occurs in a type equated to $\tvb$ in $\up$, that is, when $\up = \Up\where{\cunif \tvb {\cunif \t \ueq}}$,
$\tv \in \fvs \t$ and $\tv, \tvb \disjoint \bvs \Up$.
%
A unification problem $\up$ is said to be cyclic, written $\cyclic \up$, if
$\tv \prec_\up^+ \tv$ for some $\tv$.

% Something fishy appears to be happening here.
% DR: It was detected here, when the page is flushed, but the origin is the
% previous figure with was too high. I fixed it.

\begin{definition}[Solved form $\hat\up$]
\label{def:solved-form}
  We write $\hat\up$ for constraints in \emph{solved form}, that is,
  constraints of the form $\cexists \tvs {\cAnd \iton \ueqi}$, where:
\begin{enumerate*}
  \item
    each $\ueqi$ contains at most one non-variable type;
  \item
    each type variable may occur as a member of at most one multi-equation
    $\ueqi$;
    %% head variables do not occur in multiple equations;
  \item the constraint is acyclic.
\end{enumerate*}\relax
\end{definition}

\XDR[Alternative notations]{
  $\cpinst \inst \tv \t$ could be written
  $\tv \leadsto^i \t$ or
  $\tv \mathop{\smash\leadsto\vphantom-}\limits^i \t$
}

\Xalistair{
  Dislike the first one --- $\tvs$ are technically bound after $\tv$ ( since
  regions are $\exists \tvs. $ prefixes in $\Let$ constraints).  Also, for
  \texttt{let and}, we can just generate normal let constraints.  Perhaps
  you mean \texttt{let} patterns? If so, then we would still typically
  generate a single constraint for the entire pattern. So multiple
  abstractions doesn't really help us. We would want something like $\Let
  \overline{(\x : \tv)} \; \where\tvbs = \ca \In \cb$.

  The second one is okay, but I'm not a huge fan of the subscript.
}
\XDR
  {I was not pushing for making any change, just some thought I had when
  reading. For let-and, the $\tvs$ should contain $\tv$ and be unordered,
  and then have one or several explicit entry points.}

\subsection{Solving rules}

% What we do (introduce / explain the solver)

We now gradually introduce the rules of the constraint solver itself
(\cref {fig:solver-basic,fig:solver-schemes,fig:solver-susp}).
These rules define a non-deterministic rewriting
system, operating modulo $\alpha$-equivalence, and the associativity and
commutativity of conjunction. Rewriting takes place under an arbitrary
one-hole constraint context $\C$.
%
% Solved forms
A constraint $\c$ is satisfiable if it rewrites to a solved form $\hat\up$
(\cref{def:solved-form}); otherwise it gets stuck---in particular, $\cfalse$ is
considered a stuck constraint.

\paragraph{Basic rules}

\parcomment{Unification}

\Rule{S-Unif} (\cref{fig:solver-basic}) invokes the unification algorithm on the
current unification problem. The unification algorithm itself is treated as a
black box by the solver, so the system could be extended with any
equational theory of types implemented by the unification algorithm.

\parcomment{Regional let constraints}

In general, existential quantifiers $\cexists \tv \c$ are lifted to the nearest
enclosing $\Let$, if one exists, or otherwise to the top of the constraint. The
resulting existential prefix $\exists \tvs$ is called a \emph{region}. To make
regions explicit, we introduce the syntax $\cletr \x \tv \tvs \ca \cb$, where
$\tv$ is the \emph{root} of the region and $\tvs$ are auxiliary existential
variables. The order of $\tvs$ is immaterial; regions are considered equal up
to permutation of these variables.

Satisfiability of regional \Let-constraints is defined in
\cref{fig:constraint-let-regions}. The semantics of an
abstraction with a region, written $\semenv(\cabsr \tv \tvs \c)$, is a set of
\emph{ground regions} that satisfy $\c$. A ground region $\gr$ is a satisfying
interpretation $\semenvp$ for the region with a designated \emph{root} variable
$\tv$, written $\greg \tv \semenvp$.
\Xgabriel{I would consider changing $(\alpha, \phi)$
into $(\tau, \phi)$ where $\tau$ is $\phi(\alpha)$.
Alistair is okay.}
Regional \Let-constraints strictly
generalize ordinary let constraints, as captured by the equivalence:
\begin{mathpar}
  \clet \x \tv \ca \cb \Wide\cequiv \cletr \x \tv \eset \ca \cb
\end{mathpar}
%
\parcomment{Explanation of rules}
%
In \cref{fig:solver-basic}, \Rule{S-Let} rewrites let constraints into regional
form.
%
\Rule{S-Exists-Conj} lifts existentials across conjunctions;
\Rule{S-Let-ExistsLeft} and \Rule{S-Let-ExistsRight} lift existentials across
let-binders; \Rule{S-Let-ConjLeft}, \Rule{S-Let-ConjRight} hoist constraints
out of let-binders when they are independent of the local variables.
%
Collectively, these lifting rules normalize the structure of each region into a
block of existentially bound variables, whose body consists of a conjunction of
solved multi-equations followed by a residual constraint---typically an
instantiation, let-binding, or suspended constraint.


\parcomment{\OML constraints do not need dedicated rules}

\OML-specific constraints, such as the label and polytype instantiation
constraints ($\labfrom \elab \ct \leq \ta \to \tb$, $\cleq \cscm \t$,
\etc), require no special treatment in our solver. Once their pattern variables
are substituted---after solving a match constraint---they are desugared into
constraints already handled by the solver.

\paragraph{The trouble with lets}

% Let-constraint solving is generalization

\parcomment{Previous ways of solving abstractions (aka generalization)}

% This paragraph is titled Let constraints (seems odd to start off with
% 'Generalization constraints ...'

Solving $\Let$ constraints is deceptively difficult.
Naively, let constraints (or \emph{generalization} constraints) could be solved by
copying constraints:
\begin{mathpar}
\hfil
\rewrite
    {\cletr \x \tv \tvs \ca {\C\where{\capp \x \t}}}
    {\tv, \tvs \disjoint \t \\ x \disjoint \bvs \C}
    {\cletr \x \tv \tvs \ca {\C\where{\cexists {\tv, \tvs} \cunif \tv \t \cand \ca}}}
\eqno
{\DefTirName{S-Let-App-Beta}}
\end{mathpar}
%
This rule, due to \citet*{Pottier-Remy/emlti}, resembles $\beta$-reduction,
except that the original let-constraint is retained. While this rule is sound
when $\ca$ is a \emph{simple} constraint, it becomes unsound for abstractions
containing suspended constraints.

%
\Xgabriel{I had to change the example below and tried to reformulate the
  following discussion to match.}
%
\Xalistair{I changed the example again. There are two points that we want to
  convey: we cannot make $\Let$ monomorphic, we cannot apply $\beta$-reduction.
  The previous example demonstrates the former but not the latter.}
%
\Xgabriel{I am fine with the new writing and I propose to leave it at that, but for the record I feel that there are several points to be made, I don't know if we should try to handle them together or separately, and it is not obvious that everything is covered. A first point worth making is that (A) the constraint-duplication rule is inefficient, so ML engines generalize eagerly. Then one can point (as you do here) that (B) duplication in fact becomes incorrect in our setting. One can also point out independently that (C) it is not obvious how to generalize 'eagerly' in presence of suspended constraints. I also tried to say that (D) when we encounter a use of the term variable later on, suspended the instantiation completely would be incomplete (note: this does not require back-propagation, as I already observed this with Olivier before we introduced the idea of back-propagation), and finally (E) back-propagation can occur. The previous structure talked about (A) first and then (C, D) (and (E) in passing but it was not the focus). The new structure talks about (B) first, then (A), and then (E) (reading (D) between the lines). My favorite treatment of this discussion would be to explain (A) first (it corresponds to previous work), then (B) as a remark, then (C) and (D) together which I think are the key take-aways. Then we could remark (E) that we obtain back-propagation as an emergent property of the design (with the same example or a different example, depending on what we find that is simple and natural).


 (this is interesting and worth pointing out, but I don't know that it is required to explain incremental generalization; it could be said separately, maybe first). We can also point out that we are stuck during generalization }
%
\Xalistair{You raise several good points here! There is one that you missed,
(F) that making let bindings monomoprhic is incomplete. I can add (C) to the
current version. Order would be: E, A, B, C, F. I'm not too sure D with worth
talking about (since back-propagation makes it clear that this approach doesn't
work). If we didn't have back-propagation, then perhaps? Back-propagation isn't
an emergent property of the solver's design, but a consequence of the design of
our unicity conditions.}

\begin{local}
\def \xgetx{{\ttlab {getx}}}
\def \xdiag{{\ttlab {diag}}}
\def \elabx{{\ttlab x}}
\def \tvp {{\tv_{\ttlab p}}}
\def \tvgp {{\tv_{\ttlab {gp}}}}
\def \tvgetx {{\tva_\xgetx}}
\def \tvgetxret {\tvc}
\def \tgpoint {\ttlab {gpoint}}

\parcomment{Example}
To see why, consider again the example \code{ex_8} from \cref{sec/overview/omni}:%
%
\begin{program}[input]
type 'a gpoint = { x : 'a; y : 'a }
let diag (n : 'a) : 'a gpoint = { x = n; y = n }
°\halfline°
let ex_8 gp = let getx p = p.x in getx (diag 42), (getx gp : float) °\ocamlflags 20°
\end{program}
%
As explained earlier, this program is clearly well-typed: the type of \code{p}
is unambiguously determined as \code{'b gpoint} through \emph{back-propagation}
from the first application of \code{getx} (\code{getx (diag 42)}).
%
The (simplified) generated constraint for \code{ex_8} contains:
\begin{mathpar}
  \cexists {\tvgp, \tvd}
  \clet \xgetx \tvgetx
  {\cexists {\tvp, \tvgetxret}
    {\bigwedge \PARENS {
      \cunif \tvgetx {\tvp \to \tvgetxret} \\
      \cmatch \tvp {\cbranch {(\cpatrcd \ct)}
         {\labfrom {\ttlab \x} \ct \leq \tvp \to \tvgetxret}}
    }}}
  {\\ \capp \xgetx \parens{\trcd \tgpoint \tint \to \tvd}
      \cand \capp \xgetx \parens{\tvgp \to \tfloat}}
\end{mathpar}
  If we now apply \Rule{S-Let-App-Beta} to both applications of $\xgetx$
  (removing the $\Let$ constraint for concision), we obtain:
\begin{mathpar}
  \begin{array}{rl}
    \cexists {\tvgp, \tvd} {} &

  \cexists {\tvp_1, \tvgetxret_1}
\bigwedge \PARENS {
  \cunif {\trcd \tgpoint \tint \to \tvd} {\tvp_1 \to \tvgetxret_1} \\
      \cmatch {\tvp_1} {\cbranch {(\cpatrcd \ct)}
         {\labfrom {\ttlab \x} \ct \leq \tvp_1 \to \tvgetxret_1}}
    } \\

    \cand & \cexists {\tvp_2, \tvgetxret_2}
\bigwedge \PARENS {
  \cunif {\tvgp \to \tfloat} {\tvp_2 \to \tvgetxret_2} \\
      \cmatch {\tvp_2} {\cbranch {(\cpatrcd \ct)}
         {\labfrom {\ttlab \x} \ct \leq \tvp_2 \to \tvgetxret_2}}
    }
  \end{array}
\end{mathpar}
The first conjuct is satisfiable, since $\tvp_1$ is unified with $\trcd \tgpoint \tint$, thereby discharging
the match constraint on $\tvp_1$.
The second, however, is not: $\tvp_2$ remains underdetermined, leaving its match
constraint unsatisfiable. Thus, although the original constraint was
satisfiable, the application of \Rule{S-Let-App-Beta} makes it unsatisfiable.
By copying the abstraction, we lose the essential \emph{sharing} between both
instantiations---namely, that both copies of $\tvp$ ($\tvp_1$ and $\tvp_2$) must
have the same shape $\any \tvc \trcd \tgpoint \tvc$. This loss of sharing is precisely
why \Rule{S-Let-App-Beta} is unsound in the presence of suspended constraints.

\parcomment{What about treating let constraints monomorphically?}

A tempting alternative, used in
\citet*{Vytiniotis-Peyton-Jones-Schrijvers-Sulzmann/outsidein@jfp2011} and
\citet*{benevs2025simple}, is to treat the $\Let$-bindings
\emph{monomorphically}, sharing $\tvp$ directly between both applications.
However, this is incomplete: in \code{ex_8}, the two calls to \code{getx}
require different instantiations of $\tvp$---\code{int gpoint} and \code{float
gpoint}, respectively.

\parcomment{Generalization in the context of constraint solving}

Even setting soundness aside, \Rule{S-Let-App-Beta} is inefficient.
Each application duplicates constraint solving work for the same abstraction.
%
A more efficient approach first solves the abstraction once---\eg reducing it
to $\cabsr \tv \tvs \ueqs$, where $\tvs$ are generalizable variables---and then
reuses the result at each instantiation site by only copying the solved
constraint $\ueq$. This mirrors the generalization and instantiation steps of
\ML inference algorithms such as $\mathcal{W}$: $\cabsr \tv \tvs \ueqs$
corresponds to the type scheme $\tfor \tvs {\sub(\tv)}$, where $\sub$ is the
most general unifier of $\ueqs$.
%
\citet*{Pottier-Remy/emlti} formalize this connection, and the optimized
treatment is naturally expressed as a strategy on top of their
\Rule{S-Let-App-Beta} rule.

\parcomment{The two problems and the solution}

We therefore face two related challenges: \begin{enumerate*}
  \item handle instantiation soundly in the presence of suspended constraints, and
  \item to avoid redundant work by reusing \emph{partial} results across instantiations.
\end{enumerate*}

To address both, we introduce \emph{partial type schemes}, our second novel
mechanism for omnidirectional inference. Partial type schemes are type schemes
that delay commitment to certain quantifications (\eg~$\tvp$ and~$\tvgetxret$).
Such \emph{partially generalized} variables are treated as generalized, but can
be incrementally refined in future as suspended constraints are discharged.

Returning to our running example, we begin with the partial type scheme $\all
{\tvp, \tvgetxret} \tvp \to \tvgetxret$ for \code{getx}, since the suspended
match constraint in its abstraction cannot yet be solved. We continue solving
the body of the $\Let$-constraint, tracking every instances of this partial
scheme. As type information flows back---\eg when the shape of $\tvp$ becomes
known via back-propagation from the first application of \code{getx}---the
scheme is refined to $\all {\tvb} \trcd \tgpoint \tvb \to \tvb$. The second
application of \code{getx} then updates its instantiation accordingly, unifying
$\tvp_2$ with $\trcd \tgpoint \tvb_2$ and $\tvgetxret_2$ with $\tvb_2$.

\end{local}


\paragraph{Incremental instantiation}

\parcomment {Intro partial instantiations}

To support partial type schemes, we extend the constraint language with
\emph{incremental instantiation constraints}
(\cref{fig:constraint-let-regions}).
We introduce two new constraint formers:
%
\begin{enumerate}
  \item
    $\cexistsi \inst \x \c$, which binds a fresh instantiation $\inst$ of $\x$'s
    region within $\c$, and
  \item
    $\cpinst \inst \tv \t$, which asserts that the copy of $\tv$ in $\inst$
    equals~$\t$.
\end{enumerate}
The instantiation variable $\inst$ is required to ensure all incremental
instantiations $\cpinst \inst \tv \t$ are solved uniformly.

% I wanted to put this before the syntax introduction, but I think this order
% works better since it is hard to conceptually understand incremental instantiations
% without the syntax

\parcomment {Conceptual model}

Within the solver, we view incremental instantiations as markers indicating
which parts of the abstraction still need to be copied, and abstractions
themselves are treated as partial type schemes.

This mechanism enables efficient handling of constraint
instantiations: solved parts are reused immediately, while suspended
constraints can be solved later, further refining the abstraction and
propagating new equations to all instantiation sites.

\parcomment{Semantics of incremental instantiations}

We now turn to the semantics of incremental instantiations (\cref{fig:constraint-let-regions}).
The existential constraint $\cexistsi \inst \x \c$ is satisfiable
(\Rule{Exists-Inst}) if
there exists a region $\semenvp$ that
satisfies the regional constraint abstraction bound to $\x$, and the body $\c$ is satisfied
with $\inst$ mapped to $\semenvp$.

Incremental instantiations (\Rule{Incr-Inst}) equate the copy of $\tv$ in
$\inst$ with $\t$. Intuitively, $\inst$ represents a distinct instantiation of
$\x$'s abstraction, and each $\cpinst \inst \tv \t$ links the corresponding
copy of $\tv$ to its instantiated type $\t$.
%
The domain of incremental instantiation constraints must lie within the closure
of the abstraction or among the regional variables of $\x$. Consequently, the
variables $\tv, \tvs$ bound by the \Let-constraint $\cletr \x \tv \tvs \ca \cb$
are bound not only in the abstraction body $\ca$, but also in the constraint
$\cb$, where they may appear in incremental instantiations of $\x$ in the
domain of renamings---and only there. Hence, they cannot appear in $\cb$ when
the corresponding variable $\x$ does not itself appear in $\cb$.


\parcomment{Solving incremental instantiations}

Incremental instantiation constraints are reduced using the following rules, summarized in
\cref{fig:solver-schemes}:
\begin{enumerate}

\item
  \Rule{S-Inst-Copy} copies the shape of a type to the instantiation site,
    introducing fresh variables for each subcomponents and marking them with
    corresponding instantiation constraints.
    %
    We write $\cpapp \x \tvb \t \inst$ as a shorthand for $\cpinst \inst \tvb \t$
    when $\inst$ is bound with $\exists \inst^\x$ in the context. To ensure
    termination, the abstraction must contain acyclic types.

  \item \Rule{S-Inst-Unify} unifies two instantiations if they both refer to the
    same source variable $\tvb$ at the same instantiation site $\inst$.
\end{enumerate}
There are three cases in which an instantiation constraint is eliminated:
\begin{enumerate}
  \item
    A nullary shape is copied and no further instantiations are needed
    (\Rule{S-Inst-Copy}).

  \item
    The copied variable $\tvb$ is polymorphic, and thus the instantiation
    constraint imposes no restriction (\Rule{S-Inst-Poly}), provided no
    other instantiations of $\tvb$ remain at the same instantiation side (if
    not, then apply \Rule{S-Inst-Unify}).

  \item
    The copy is monomorphic and in scope, so we unify it directly
    (\Rule{S-Inst-Mono}).
\end{enumerate}

\parcomment{S-Let-AppR}

\Rule{S-Let-AppR} rewrites an instantiation constraint $\capp \x \t$ introducing an
incremental instantiation constraint $\cpinst \inst \tv \tvc$. Here, $\inst$ is
a fresh instantiation of $\x$, $\tv$ is the \emph{root} of $\x$'s region, and
$\tvc$ is a fresh alias for $\t$. We introduce $\tvc$ explicitly, since our
rewriting rules for incremental instantiations generally assume that the copied
type is a variable rather than an arbitrary type.


%% Uncomment to restore the old style
%% \let \rewrite \hrewrite

\paragraph{Let constraints}

\parcomment{Cleaning up partial instantiations and let constraints}

\Rule{S-Let-Solve} removes a $\Let$ constraint when the bound term
variable is unused and the abstraction is satisfiable. \Rule{S-Compress}
determines that a regional variable $\tvb$ is an an alias for $\tvc$. We
replace every free occurrence of $\tvb$ with $\tvc$---\emph{including} the
domains of any partial instantiation constraints, written as the
substitution $\where{\x.\tvb \is
\tvc}$.

Conceptually, \Rule{S-Compress} acts a variable-level analogue of
\Rule{S-Inst-Copy}: both rules copy solved constraints $\ueq$ from an
abstraction to its instantiation constraints. While \Rule{S-Inst-Copy}
propagates the head shape of a multi-equation, \Rule{S-Compress} propagates
equalities between variables within a multi-equation, thereby enabling
subsequent applications of \Rule{S-Inst-Unify}.

\XDR{I don't know what a \emph{copy rule} means. Besides, this is not
any copying/unsharing, but the oppposite...}
\Xalistair{Does the updated text help you?}
\XDR{I now understand what you meant. It is better, as the rule you refer to
is explicit. I still do not think it helps understand what S-Compress
does. S-Inst-Copy propagate some structure and allocate some structure while
S-Compress propayages and equality and its effect will be to share some
structure}
\Xalistair{I would spin it slightly different. Both are rules that are used to
copy solved constraints $\ueq$ from an abstraction. \Rule{S-Inst-Copy} and \Rule{S-Compress}
do this, the latter copies the head shape of a multi-equation and the former copies
the equalities between variables within a multi-equation. I've updated the text to reflect
my understanding, what do you think?}

\parcomment{Lowering}

\Rule{S-Exists-Lower} implements the non-trivial case of lowering
existentials across \Let-binders. It identifies a subset of variables in
the region of a $\Let$ constraint that are unified with variables from
outside the region. Such variables are considered monomorphic and thus
cannot be generalized; they can instead be safely lowered to the outer
scope.

\parcomment {Determines}

This is the case when the types of $\tvbs$ are \emph{determined} in a unique
way. In short, $\c$ determines $\tvbs$ if and only if the solutions for
$\tvbs$ are uniquely fixed by the solutions to other variables in $\c$.
\begin{definition}
  $\cdetermines \c \tvbs$ if and only if every ground assignments $\semenv$
  and $\semenvp$ that satisfy (the erasure of) $\c$ and coincide outside of
  $\tvbs$ coincide on $\tvbs$ as well.
  \belowdisplayskip 0em
  \begin{mathpar}
    \cdetermines \c \tvb \uad\eqdef\uad \all {\semenv, \semenvp} \uad
      \semenv \th \cerase \c
      \wedge \semenvp \th \cerase \c
      \wedge \semenv =_{\setminus \tvbs} \semenvp
      \implies
      \semenv = \semenvp
  \end{mathpar}
\end{definition}
\parcomment {How the determines relation corresponds to ML}
Conceptually, this corresponds to the negation of the generalization
condition in \ML: a type variable \emph{cannot} be generalized if it appears
in the typing context. In the constraint setting, it \emph{cannot} be
generalized if it depends on variables from outside the region. For
instance, $\cexists \tvb \cunif \tv {\tvb \to \tvc}$ determines $\tvc$,
as $\tv$ is free and therefore constrains the solution of $\tvc$.

\parcomment{How to decide the relation}

To decide when $\cdetermines \c \tvs$, we introduce the judgment $\th
\cdetermines \c \tvs$, which syntactically proves that $\tvs$ are determined
in $\c$.
%
If $\c$ is of the form $\cexists \tvbs \cp$ where $\tvbs \disjoint \tvs$, then
we search for a multi-equation $\ueq$ in $\cp$ of the form:
\begin{enumerate*}
  \item[(\Rule{Det-Dom})]
    $\cunif \tvc \ueq'$ where $\tvc \disjoint \tvs, \tvbs$ and
    $\tvs \subseteq \fvs {\ueq'}$, or
  \item[(\Rule{Det-Esc})]
    $\cunif \tvs {\cunif \t \ueq'}$ where
    $\fvs \t \disjoint \tvs, \tvbs$.
\end{enumerate*}
%
This syntactic relation coincides with the semantic definition of
determinacy whenever $\c$ is in solved form. Otherwise, it is a
sound approximation of the semantic definition.

\parcomment{Why we lower?}

Lowering such variables improves solver efficiency. It avoids unnecessary
duplication of work that would otherwise occur via \Rule{S-Inst-Copy}. Because
these variables are determined by monomorphic ones, lowering allows the solver
to apply \Rule{S-Inst-Mono} directly at instantiation sites, rather than
duplicating equivalent constraints that ultimately express the same semantic
fact---that the variable is monomorphic and shared across instances.

\paragraph{Suspended match constraints}

\parcomment{S-Match-Ctx}

\Rule{S-Match-Ctx} (\cref{fig:solver-susp}) solves suspended match
constraints. It would not be effective to allow rewriting whenever the
unicity condition $\Cshape \C \t \sh$ holds, because it is not
a-priori feasible to check or decide this semantic condition which
quantifies over all solutions. Instead we introduce a restricted,
decidable subset of the unicity condition that relies on syntactic
conditions, which we describe as three ``unicity rules'' for a judgment
written $\th \Cshape \C \t \sh$.

The unicity rule \Rule{Uni-Type} applies when $\t$ is a non-variable
type $\t$, in which case the shape is simply $\shape \t$. The unicity
rule \Rule{Uni-Var} applies when the scrutinee is a variable $\tv$ and
the context establishes that $\tv$ is equal to some non-variable type
$\t$ by exhibiting an equality $\cunif \tv {\cunif \t \ueq}$ and $\t$ is
a non-variable type. In this case, the shape of $\tv$ is $\shape \t$.

Finally, the unicity rule \Rule{Uni-BackProp} expresses
\emph{backpropagation}, previously illustrated in \cref{ex:backprop}. In
particular, the shape of a regional variable can sometimes be determined
from its instantiations. If an abstraction contains a regional variable
$\tvp$, and the constraint context includes an incremental instantiation
$\cpapp \x \tvp \tvc \inst$ such that the instance $\tvc$ of $\tvp$ has
the unique shape $\sh$, then $\tvp$ must also have shape $\sh$, as any
other shape would render the instantiation unsatisfiable. This rule is
well-founded because the regional depth of the hole strictly decreases
in the premise.

For a solved context $\hat\C$, this syntactic is moreover sound and
complete with respect to the semantic definition of unicity:
$\Cshape \C \tau \sh$ holds if and only if $\th \Cshape \C \tau \sh$ is
derivable.\Xgabriel{Is this proved somewhere?}

\subsection{Metatheory}

We establish the correctness of our solver. Correctness follows from three
standard metatheoretic properties: \emph{progress}, \emph{preservation}, and
\emph{termination}.  Together, they ensure that every satisfiable
(term-variable-closed) constraint eventually reduces to an equivalent solved
form.

% Properties
% May look like space hacks, but restatable introduces its own space

\begin{definition}
A constraint $\c$ is term-variable-closed if all its term variables $\x$ are
bound \ie $\fvs \c \subseteq \TyVars$.
\end{definition}

\begin{lemma}[Scope preservation]
If $\ca \csolve \cb$, then $\fvs \ca \supseteq \fvs \cb$.
\end{lemma}

\begin{theorem}[Closed Progress]
  If a term-variable-closed constraint $\c$ cannot take a step $\c \csolve \cp$,
  then either:
  \begin{enumerate}
    \item $\c$ is solved.
    \item $\c$ is $\cfalse$.
    \item for every match constraint $\hat\C\where{\cmatch \tv \cbrs}$ in
        $\c$, $\Cshape {\hat\C} \tv \sh$ does not hold for any $\sh$.
  \end{enumerate}
\end{theorem}

\vskip -\lastskip
\begin{restatable}[Termination]{theorem}{terminationBIS}
  \label{thm:termination}
  The constraint solver terminates on all inputs.
\end{restatable}
\vskip -\lastskip
\begin{restatable}[Preservation]{theorem}{preservationBIS}
  \label{thm:preservation}
  If $\ca \csolve \cb$, then $\ca \cequiv \cb$.
\end{restatable}

\begin{corollary}[Correctness]
  For the term-variable-closed constraint $\c$, $\c$ is satisfiable if and
  only if\/ $\c \csolve^* \hat\up$ and $\hat\up$ is a solved form equivalent
  to $\c$.
\end{corollary}

\section{Implementation}
\label{sec:implementation}

We have two working prototypes implementing the \OML language with suspended
match constraints and partial type schemes, in which we have reproduced the
various type-system features and examples presented in this work.  One
closely follows the constraint-based presentation described
here\footnote{The other prototype is a direct implementation of type
inference based on semi-unification. We mention it here only it indicate
that we have explored multiple implementation strategies leading to the same
results.}. It is public and open-source\footnote {Link omitted for
anonymity.}. Its implementation is inspired by previous work such as
\Inferno~\citep {Pottier/inferno@icfp2014, Pottier/inferno@opam}.  It uses
state-of-the-art implementation techniques for efficiency, such as a
Tarjan's union-find data structure for unification
\citep*{journals/jacm/Tarjan75} and \emph{ranks} (or \emph{levels}) for
efficient generalization \citep*{Remy/mleth}. Let us discuss a few salient
points.

\paragraph{Unification and scheduling}

Each unsolved unification variable maintains a \emph{wait list} of suspended
constraints that are blocked until the variable is unified with a concrete
type. When such a unification occurs, the wait list is flushed: the suspended
constraints are scheduled on the global constraint scheduler, which is
responsible for eventually solving them.

\paragraph{From a stack to a tree}

% To implement generalization (the \Rule{S-Exists-Lower} rule)
% efficiently, we follow the classic rank-based approach to
% generalization. Each $\Let$ constraint and type variable is allocated
% an integer \emph{rank}, which informs us the depth of the region
% within the constraint. Type variables of rank $0$ are bound at the
% top-level region, and type variables of rank $r \geq 1$ are bound in
% the region of $\Let$ constraint at depth $r$.

% As inference progresses, unification may widen the scope of variables,
% thereby lowering their rank. The set of variables eligible for
% generalization at a given region consists precisely of those
% whose rank remains equal to that of the region.

The efficient implementations of \ML type inference, introduced by \citet
{Remy/mleth} and reused in \Inferno, represents the solver state as a linear
\emph{stack} of inference regions, from the outermost variable scope to the
current region.
%
Each let-binding and, more generally, each generalization location,
introduces a new scope, hence a new region.
%
Since the solver visits the source terms, hence introduces local regions in
a depth-first manner, these variable scopes may be represented by an integer
\emph{rank} (or \emph{level}) attached to each variable---actually to each
type node.  Unification maintains these ranks to their minimum value when
merging types, which amounts to compute the least common ancestor of the
region types should belong to.  When leaving a region,
%
The use of integer levels to represent regions, which is quite efficient and
simple to implement, does not suffice for partial generalization.
If generalization at some region contain a variable appearing in a suspended
match constraint, the region must be kept alive while we continue inference
in other regions. Hence, later parts of the constraint may introduce a new
\Let-region at the same rank that is unrelated to the suspended one---neither its
ancestor nor its descendant---breaking the linear assumption that allowed
the representation of variable scopes by integer ranks.

We must instead use a \emph{tree} of nested \Let-regions to represent
scopes. Under this scheme, ranks no longer uniquely determine a variable's
region.  Instead, we interpret a rank relative to a path in the region tree
from the root. When two variables are unified, they must always lie on some
shared path---by scoping invariants---so computing their minimum rank (along
this path) still suffices to determine the least common ancestor: we keep
the efficient integer comparisons of generalization.

\paragraph{Partial generalization}

\noindent
\XDR
  {We agreed that this could be left as such, and perhaps improved
  later. Just kept as a reminder.}
\Xalistair
  {I agree that it is a little too abstract. To make it concrete, I would
  suggest code snippets (I rarely understand implementation mechanisms
  without code). But at the same time, this feels out of place for the
  paper.  Additionally, the final version will link the URL of the open
  source implementation, which when combined with this description ought to
  be sufficient to reproduce our approach}

Generalization is the process of determing which type variables are
polymoprhic and which are monomorphic (\ie implementing
\Rule{S-Exists-Lower}).  \emph{Partial generalization} arises when a region
cannot be fully generalized due to suspended constraints that may still
update its variables. To manage this, we classify type variables into four
categories:
\begin{itemize}
  %% \let \\ \relax
  \let \Item \item
  \renewcommand \item [1][]{\Item[(\textbf{#1})]}
  \item[I] Variables are yet to be generalized. \\
    \emph{Introduced by instantiations or source types in constraints}

  \item[G] Variables that are generalized. \\
    \emph{Not accessible from any instance type. Definitely polymorphically.}

  \item[PG] Variables that are partially generalizable. \\
    \emph{Generalizable variables mentioned by suspended match constraint or partial
    instantiations. Maybe polymorphic, maybe monomorphic.}

  \item[PI] Variables that were previously partially generalized
    but have since been updated.  \\
    \emph{Awaiting re-generalization. Introduced by the unification of partial
    generics.}
\end{itemize}
At generalization time, we conservatively approximate whether a variable may
be updated in the future using \emph{guards}. A guard is a mark on a
variable that indicates the variable is captured by some suspended
constraint that has not yet been solved. Guarded variables are generalized
as partial generics (\textbf{PG}); unguarded ones are fully generalized
(\textbf{G}).'

When an instance is taken from a partial generic, we retain a forward reference
from the partial generic (\textbf{PG}) to the instance. This enables the
generic to notify the instance that it has been updated, propagating the
updated type structure to all instances. This mirrors, in reverse, the way our
formalized solver uses incremental instantiation constraints to track copies. In
addition, the instance remains guarded by the partial generic until the latter
is either lowered or fully generalized.

Once a suspended match constraint is solved, it removes the guards it
introduced. This may enable previously partial generics to become fully
generalizable. Conversely, if a partially generalized variable is lowered (\eg
by \Rule{S-Exists-Lower}), it must be unified with all its instances.

\paragraph{Lazy generalization} Repeatedly generalizing a region after every
update is expensive.  Instead we generalize on demand. We mark regions as
``stale'' when they may require re-generalization. When an instance is
taken, we re-generalize the stale descendants of the region in the region
tree. Although this avoids performing re-generalization too early before
instantiation, there are still scenarios in which re-generalization of a
region before instantiation from this region may require significantly more
work that instantiation before re-generalization, because instantiation may
unlock suspended constraints that will in turn induce considerable
simplification in that region.


\XDR{@Allistair: I added two sentences above, but feel free to modify or
delete if you disagree.}

\XDR{I move the discussion on efficient re-generalization to our pad.}


\paragraph {Semi-unification.}

Our second prototype is based on semi-unification.  Semi-unification,
introduced in the late 80's in the context of term rewriting, has also been
used for type inference with polymorphic recursion in \Miranda and \ML~\cite
{Henglein/phd,Henglein/tiasu@toplas93}.
%
Widely studied in the early 90's, it has soon been proved undecidable~\cite
{Kfoury-Tiuryn-Urzyczyn/ic1993} and almost abandoned for type inference
purposes---even though \citeauthor{Henglein/phd} observed that he was unable
to find an example of type inference for which his semi-algorithm would
actually loop\footnote {Although, we knew from the undecidability of
semi-unification that such example existed, it is only much later that the
class of examples for which the semi-algorithm would loop could be
characterized~\cite {Figueiredo-Camarao/semiunif-ptm@unpub2024}: these
patterns are extremely complex, which suggests that they are quite unlikely
to appear in practice, unless crafted especially.}.  Besides, in the absence
of polymorphic recursion, semi-unification problems are acyclic and thus
guaranteed to terminate.

Quite interestingly, semi-unification generalizes unification constraints to
instantiation constraints, which can both be solved incrementally in any
order. Therefore, semi-unification based type inference is a good basis
candidate for performing omnidirectional type inference in \ML.  We first
enriched semi-unification edges which a notion of scopes that forms the same
tree-like structure as the one presented above: this simultaneous allows for
a more direct encoding of type inference problems into semi-unification
constraints and for a more efficient treatment of monomorphization,
\ie. instantiation edges that must be turned into unification edges.
%
In the absence of fragile constructs (and of polymorphic recursion), there
was no real benefit to use semi-unification for \ML type inference, yet
\citeauthor{Henglein/phd}'s algorithm was omnidirectionality-ready,
just waiting for fragile constructs, to find a new application and be
resurrected.

Although started from a different point of view, the implementation issues
and their solutions are unsurprisingly similar to the previous approach.  In
fact, this close correspondence is also an invitation to generalize
typing constraints with partial type scheme and incremental instantiation to
also cope for polymorphic recursion.

\XDR{I wonder whether semi-unification fits better in the implementation or
related-work section.  The goal is to explain that semi-unification was
omnidirectionality-ready but also to mention extensions and say that
implementation details meet the other implementation.}


\section{Related work}
\label{sec:related-work}

\XDR
  {Semi-unification implementation of type inference, which according to
  Henglein has been implemented for SML, does implement incremental
  instantiation efficiently.}%
\TODO{Include a paragraph of Related Work to mention this.}

\paragraph{Overloading}

Qualified types~\citep*{jones-qualified-types}, best known for their use in
\Haskell's type-classes, are related to our suspended match constraints: both
represent constraints on types or type variables that are delayed. At
generalization time, constraints on generalizable variables are retained in the
type scheme, yielding a \emph{constrained type scheme} $\tfor \tvs {\c
\Rightarrow \t}$. This is much simpler to implement than our partial type
schemes, but it provides a different behavior: each instance may resolve $\c$
differently (as the constraint is copied on instantiation).
%
Qualified types are excellent choice when this is the desired behavior,
typically for \emph{dynamic overloading} \citep{conf/popl/WadlerB89}. But they
are insufficient when we require a unique resolution of the constraint across
all instances---as in \emph{static overloading}.

\citet{Leijen-Ye/prefix@pldi2025} recently proposed a bidirectional account of
generalized static overloading within \ML. However, their approach is limited by
its reliance on fixed directionality (\cref{sec/overview/directional}).
%
Variational typechecking \citep{Chen-Erwig-Walkingshaw/variational@toplas2014}
was originally developed for reasoning about well-typed CPP \code{#ifdef}-style
macros, introducing \emph{choices}  $a \angles {\ea, \eb}$, where $a$ is a
\emph{dimension} with \emph{alternatives} $\ea$ and $\eb$. Once dimensions are
fixed, we are able to project a well-typed non-variational program.
\citet{benevs2025simple} apply this machinery to recast static overloading as
variational typing, with a resolution algorithm that uniquely selects the
dimensions.
%
However, their system removes \emph{local let-generalization} and requires an
exponential-time resolution procedure---an unavoidable consequence of the
NP-hardness of \emph{general} static overloading \citep*
{Chargueraud-Bodin-Dunfield-Riboulet/jfla2025}.

Partial type schemes provide an alternative that preserves \ML's local
let-generalization while suspended constraints offer a tractable account of
static overloading. By enforcing resolution using \emph{known} type information
(captured by our novel unicity condition) rather than \emph{guessed}
information, our approach remains tractable. Our experience suggests that this
is a ``goldilocks'' solution: expressive enough for most applications, yet
tractable, and (crucially) compatible with \ML's \emph{local
let-generalization}.


\paragraph{Suspended constraints}

Suspending constraints that cannot be solved yet is not a novel idea: it is a
standard approach to implement unification dependently-typed systems. This goes
back to Huet's algorithm for higher-order unification~\citep*{huet-unif} and
pattern unification~\citep*{Miller/pattern-unif@iclp91} where flexible-flexible
pairs are delayed until at least one side becomes rigid.
%
Our contribution lies in combining constraint suspension with \ML-style
implicit polymorphism--- largely absent from dependently typed systems---and in
formulating a declarative constraint semantics.

Conditional constraints \citep*{journals/njc/Pottier00} also delay resolution,
waiting until the top-level constructor of a type is known. They provide an
\code{if-then-else}-like primitive, but differ crucially from our suspended
constraints: in Pottier's system, an unresolved conditional constraint is
considered satisfiable, whereas in ours, an unresolved suspended constraint is
not. This difference forces our semantics to track what is \emph{known} in a
context. Consequently, unresolved conditional constraints may enter a
generalized type scheme as a form of qualified types, while our suspended
constraints cannot. These semantic differences lead the two approaches to
address very different user-facing type system features.


\OutsideIn~\citep*{conf/icfp/SchrijversJSV09} is a type system for GADTs that
introduces \emph{delayed implications} of the form $\where{\tvs}(\all \tvbs \ca
\Rightarrow \cb)$. Constraint solving for delayed implications proceeds in two
steps; solving simple constraints first and then solving delayed implications.
The deferral ensures that inference for GADT match branches occurs when more is
known about the scrutinee and expected return type from the context.
%
To ensure principality, \OutsideIn enforces an algorithmic restriction: the
variables $\tvs$ must already be instantiated to concrete type constructors
before they may be unified by the implication's conclusion $\cb$. This ensures
information only flows from the outside into the implication's conclusion.
Notably, they do give a declarative specification for this restriction, using
an elegant but mysterious quantification on all possible ways to type the
context outside the GADT clauses. Using our new perspective on \emph{known}
type information, we can say that their semantics enforces that only
\emph{known} information from outside GADT clauses can be used inside.
%
Later work on \OutsideIn argues
\citep*{Vytiniotis-Peyton-Jones-Schrijvers-Sulzmann/outsidein@jfp2011} that
delayed implication constraints make local let-generalization all but
unmanageable, both in theory and implementation. Their proposed fix is to
abandon local let-generalization altogether. By contrast, our work shows that
the difficult interactions between let-generalization and suspended constraints
can be resolved. Furthermore, \OutsideIn forgoes a declarative specification
complete with respect to its inference algorithm, on the grounds that such a
specification would be ``as complicated and hard to understand as the
[inference] algorithm''. We believe that our \emph{omnidirectional recipe}
could provide a declarative specification: one capable of being principal and
complete for GADTs, and we would be interested in studying this application.

\paragraph{Higher-rank polymorphism}

Polytypes are not \emph{higher-rank} in the usual sense; our interest in them
stems from their role in \OCaml's inference of polymorphic methods. Many
systems for higher-rank polymorphism exist; here we highlight a few in the
context of \ML.

\MLF is an extension of \ML that supports first-class
polymorphism that goes beyond the power of System $\mathsf{F}$, while retaining
type inference. It is a generalization of \citet{Garrigue-Remy/poly-ml}'s
polytypes, relying on \geninst-directionality, but it remains unclear how to
effectively scale \MLF to the rest of \OCaml's features.
%
\FreezeML is an impredicative type inference system in which polymorphic
variables can be \emph{frozen}, written $\lceil x \rceil$, and only allowing
instantiation on ordinary (unfrozen) variables $x$. Unlike polytypes,
\FreezeML permits higher-rank types directly in the syntax of types, though these
can be encoded back into polytypes. The essential difference is that
generalization of higher-rank types is implicit, inferring the most-general
type (if one exists).

\QuickLook \citep{journals/pacmpl/SerranoHJV20}, \Haskell's latest approach at
impredicative higher-rank polymorphism, uses bidirectional propagation to take
a ``quick look'' at the spine of an application to guide instantiation of
higher-rank functions. This permits \emph{argument reordering}, by allowing
later arguments to influence the type of earlier ones within the same
application spine. It mitigates the order-dependence of fixed directionality
(\cref{sec/overview/directional}) but still enforces a global direction of
type propagation between surrounding constructs (\eg \Let-bindings). In this
sense, \QuickLook is \emph{locally omnidirectional} but \emph{globally
directional}: they reduce order sensitivity within applications, but do not
provide a declarative account of type flow between arbitrary constructs, as
required for omnidirectionality. More recently, \Frost \citep{frost} follows a
similar line but extends it to handle $\eta$-expansion and introduces a
freezing operator, akin to that of \FreezeML, that helps propagate type
information when the default order of inference is insufficient.


\paragraph{Type-based disambiguation in \SML}

In \SML, tuples are treated as structural records with numeric labels, and
projections such as \code{#1} or \code{#2} are type-directed. This relies on
row typing: if $\e$ has the type $\tsrecord {j = \tj; \varrho}$, where
$\varrho$ is a row describing the remaining tuple fields, then $\eproj \e j$
has type $\tj$. The same mechanism also underlies record projections.

However, \SML does not support row-polymorphic definitions: restricting to
monomorphic rows allows a simple yet efficient compilation strategy for records
and tuples. To enforce this, \SML adds a prose side-condition to the typing
rules requiring that each row variable be fully determined by the ``program
context''~\citep*[Section 4.11]{sml-definition}. Our unicity conditions
provides a precise, declarative specification for this informal restriction,
filling a gap in the original specification.

\citet*[page 36]{rossberg-hamlet} remarks that this restriction interacts
poorly with \texttt{let}-polymorphism and therefore not supported in practice:
%
\begin{quotation}
  Under item 1 the Definition states that “the program context” must
  determine the exact type of flexible records, but it does not
  specify any bounds on the size of this context. Unlimited context is
  clearly infeasible since it is incompatible with let polymorphism:
  at the point of generalisation the structure of a type must be
  determined precisely enough to know what we have to quantify
  over. We thus restrict the context for resolving flexible records to
  the innermost surrounding value declaration, as most other SML
  systems seem to do as well. This is in par with our treatment of
  overloading (see 5.8).
\end{quotation}
%
We have solved this difficult interaction between disambiguation and
\texttt{let}-polymorphism. In particular, we implemented tuples in our
prototype and formalize their meta-theory in Appendix
\cref{app:full-reference}; the system behaves as expected.

We also remark that \PolySML~\citep*{sml-polyml} goes further than other \SML
implementations on this front, supporting examples that even rely on
back-propagation
\begin{program}[input,deletekeywords={true,false}]
let fun fst r = #1 r in (fst (1, 2), fst (true, false)) end; °\polysmlflags 00°
\end{program}
%% We never show the return types. The green flag is enough.
%% \programjoin
%% \begin{program}[output,deletekeywords={true,false}]]
%% val it = (1, true): int * bool
%% \end{program}

\paragraph{Record field overloading in \GHC}

\Haskell 98 defines derives selector functions for each record field, an
approach that precludes declaring two record types with the same field names
within a single module. \GHC 6.8.1 (2007) introduced the
\texttt{DisambiguateRecordField} extension for type-directed disambiguation of
record labels,%
%
\footnote{
  \url{https://ghc.gitlab.haskell.org/ghc/doc/users_guide/exts/disambiguate_record_fields.html}
}
%
and \GHC 8.0.1 (2016) added \texttt{DuplicateRecordFields}, allowing distinct
record types with overlapping field names in the same module.

However, the \GHC developers found that type-based disambiguation using
bidirectional type inference is not sufficiently predictible enough in
practice for users. As a result, \GHC is gradually moving away from
type-based disambiguation towards an approach based on qualified types
constraints of the form \code{HasField a "foo" b}
(\texttt{OverloadedRecordDot}, 2017).%
%
\footnote{
  \url{https://ghc.gitlab.haskell.org/ghc/doc/users_guide/exts/hasfield.html}
}
For instance, the projection \code{ex_1} from
\cref{sec/overview/overloading} would be written as:
\begin{program}[input]
ex_1 :: HasField a "x" b => a -> b °\ghcflags 01°
ex_1 r = getField @"x" r
\end{program}
\Xgabriel{This example shows that this is doing dynamic overloading rather
than static overloading. We could discuss this, and mention that it is
\emph{possible} to restrict this to static overloading, but having
declarative semantics for this requires our unicity conditions.}

This \texttt{HasField} approach, however, does not extend to record fields with
polymorphic types,%
%
\footnote{
  \url{https://ghc.gitlab.haskell.org/ghc/doc/users_guide/exts/hasfield.html\#solving-hasfield-constraints}
}
%
revealing difficulties in scaling it to features where the typing rule
itself depends on disambiguation---such as polymorphic records and GADT
patterns in \OCaml.  In contrast, omnidirectional inference naturally
accommodates such non-uniformity in typing rules.
p[ro

\XDR{Reading the Haskell doc, I thought that HasField was using the type of
the field to disambiguate, but trying examples it apparently does not
(neither do we).}

\Xalistair{Haskell only uses the record type \eg to solve \code{HasField a "x"
b}, it waits until \code{a} has some type constructor $\Tapp \tys$ and then
reduces the constraint.}

\XDR{Are we sure that we can can ourselves do disambiguation
with polymorphic record fields? Well, sine we do not use
the type of fields, yes we do. But that's not a big deal.}

\Xalistair{I'm practically certain we can. I can prototype this next week to
prove us correct.}




\section{Conclusions}
\label{sec:discussion}
\label{sec:future-work}

We presented a constraint-based framework for omnidirectional type inference,
scaled to \ML with \emph{local let-generalization}. Central to our approach is
a new declarative account of when a type is \emph{known} from the context,
rather than \emph{guessed}. Our constraint solver is omnidirectional:
constraints may be solved in \emph{any way}, enabled by partial type schemes.
Through two instantiations of our \emph{omnidirectional recipe}, we obtained a
sound, complete, and \emph{principal} type inference algorithm---in short,
principality held \emph{anyway}, precisely because of omnidirectionality.

\paragraph{Future work}

We aim to extend our framework to support more advanced features. One direction
is generalized \emph{static overloading}; another is \emph{higher-rank
polymorphism}. We also plan to investigate \emph{default rules}---a mechanism
where ambiguity is resolved by falling back on a default, non-principal choice
\eg \OCaml selects the most recent matching record type in scope for ambiguous
field names.

\section*{Acknowledgments}

The insatisfactory situation of type-directed propagation in OCaml has
been on our mind for years; a first concrete step to which this work
can be traced to a specific
example\footnote{\url{https://github.com/ocaml/ocaml/issues/7388}} of
inconvenient order of propagation (between pattern and expression in
a let-binding) presented by Andreas Rossberg as part of general
feedback on the use of OCaml in the \texttt{wasm} reference
interpreter \citep*{rossberg-wasm}. Gabriel Scherer proposed to delay
the resolution of ambiguous constraints and advised Olivier Martinot
as an intern on this topic in Summer
2020~\citep*{https://inria.hal.science/hal-03510890}\XDR{Missing bibtex reference}. Together they
identified the implementation and specification difficulties; their
implementation would only handle simple cases where no incremental
generalization is necessary (in particular it did not support
back-propagation), and they did not have a precise declarative
specification. Gabriel Scherer kept working on this infrequently until
2024, identified the need for incremental generalization and
instantiation, but its implementations and declarative specifications
remained incomplete.

Alistair O'Brien started working on suspended constraints in Fall
2024, coming from work on constraint-based presentations of GADT
checking -- suspended constraints could delay the checking of GADT
patterns until the equalities to be introduced are known to be between
rigid types. He proposed an alternative declarative implementation,
and implemented a working prototype of incremental generalization
instantiation, described in the present paper. Gabriel Scherer and
Alistair O'Brien converged to a semantics using unicity conditions as
presented here.

Didier Remy started working on suspended constraints in Winter
2024-2025, coming from questions of principal type inference in
presence of modular implicits, in collaboration with Samuel
Vivien. Didier Remy implemented a second prototype of delayed
constraint based on semi-unification, which is fairly different from
the prototype described in the present paper.

All co-authors contributed to the writing, with Alistair O'Brien doing
the majority of the writing, and also most of the proofs of the
meta-theory.

We wish to thank anonymous reviewers for their feedback and questions,
as well as Simon Peyton-Jones whose questions, we hope, helped us make
this work more approachable.

%% \bibliographystyle{ACM-Reference-Format}
\begin{local}
\let \t \latext \let \c \latexc

\bibliography{suspended}
\end{local}

\newpage
\FloatBarrier % prevents figures floating till the end of the document
\appendix

\section*{\Large Organization of appendices}
\label{app:outline}


\paragraph{Reference appendix}
\cref{app:full-reference} gives a full reference for all
  definitions, grammars and figures in the paper, including all cases
  (even those omitted from the main paper for reasons of space).

\paragraph{Proof appendices} These appendices contain proofs for the
formal claims in the article. They are typically written tersely.
\begin{itemize}
\item \cref{app:proofs-constraints} proves properties of the
  constraint language and its semantics. The main result is
  canonicalization, which morally establishes that uses of the
  contextual rule \Rule{Match-Ctx} can be ``permuted down'' in the
  proof until they are all at the bottom of the derivation, followed
  by a proof on a simple constraint.
\item \cref{app:proofs-solving} proves the correctness of the
  constraint solver with respect to the semantics.
\item \cref{app/oml/proofs} proves the properties about the \OML
  type system, in particular the correctness of constraint generation.
\end{itemize}

\clearpage
\section{Full technical reference}
\label{appendix:figures}
\label{app:full-reference}

This section repeats all the technical definitions mentioned in the paper,
including the cases, rules, and definitions that were omitted from the main
paper to save space. It can serve as a useful cheatsheet to understand a
definition in full, or when studying the meta-theory of the system.

\begin{mathpar}
\begin{bnfgrammar}%
\entryset[Type variables]{\tva, \tvb, \tvc}{\TyVars}{}
\\
%% \entryset[Types]{\t}{\Types}\\
\entry[Types]{\t}{
    \tv \and
    \tunit \and
    \ta \to \tb \and
    \Pi \iton \ti \and
    \trcd \T \tys \and
    \tpoly \ts
}\\
\entry[Type schemes]{\ts}{
    \t \and
    \all \tv \ts
}\\[1ex]
\entry[Ground types]{\gt}{}{}\\
\entry[Ground region]{\gr}{\greg \tv \semenv}\\
\entrysubseteq[Sets of ground types]{\gabs}{\Ground}{}{}\\
\entrysubseteq[Sets of ground regions]{\gabsr}{\GroundRegion}{}\\
\entry[Constraints]{\c}{
        \ctrue
  \and  \cfalse
  \and  \ca \cand \cb
  \and  \cexists \tv \c
  \and 	\cfor \tv \c
  \and  \cunif \ta \tb
  \nextline
  \and  \clet \x \tv \ca \cb
  \and  \capp \x \t
  \nextline
  \and  \cmatch \t \cbrs
  \nextline
  \and \ueq
  \and \cletr \x \tv \tvs \ca \cb
  \and \cexistsi \inst \x \c
  \and \cpinst \inst \tv \t
  \nextline
  \and \labfrom \elab \ct \leq \ta \to \tb
  \and \dom \ct = \elabs
  \and \cscm \leq \t \mid \x \leq \cscm
}\\[1ex]
\entry[Branches]{\cbr}{\cbranch \cpat \c} \\
\entry[Shape patterns]{\cpat}{
  \cpatwild \and \cpatprod \tv j \and \cpatrcd \ct \and \cpatpoly \cscm
} \\[1ex]
\entry[Semantic environment]{\semenv}{
  \eset \and \semenv\where{\tv := \gt}
  \and \semenv\where{\x := \gabs}
  \and \semenv\where{\x := \gabsr}
  \and \semenv\where{\inst := \semenvp}
}\\[1ex]
\entry[Unification problems]{\up}{
  \ctrue \and \cfalse \and \upa \cand \upb \and \cexists \tv \up \and \ueq
} \\
\entry[Multi-equations]{\ueq}{
  \eset \mid \cunif \t \ueq
} \\[1ex]
\entry[Constraint contexts]{\C}{
  \hole
  \and \C \cand \c
  \and \c \cand \C
  \and \cexists \tv \C
  \and \cfor \tv \C
  \nextline
  \and \clet \x \tv \C \c
  \and \clet \x \tv \c \C
  \nextline
  \and \cletr \x \tv \tvs \C \c
  \and \cletr \x \tv \tvs \c \C
  \and \cexistsi \inst \x \C
} \\
\entry[Shapes] {\Sh} {
  \any \tvcs \t
}
\\
\entry[Canonical principal shapes] {\sh} {} {}\\
\entry[Terms]{\e}{
  x \and
  () \and
  \efun x e \and
  \eapp \ea \eb \and
  \elet x \ea \eb \and
  \eannot \e \tvs \t \andcr
  \erecord {\overline{\elab = \e} } \and
  \efield e \elab \and
  \exrecord \T {\overline{\elab = \e}} \and
  \exfield \e \T \elab
  \andcr
   (\ea, \ldots, \en) \and
   \efield e j \and
   \exfield e n j \andcr
   \epoly e \and
   \expoly e \tvs \ts \and
   \einst e \and
   \exinst e \tvs \ts
   \nextline
   \and \emagic \es
}\\
\entry[Term contexts]{\E}{
  \hole
  \and \eapp \E \e
  \and \eapp \e \E
  \and \elet \x \E \e
  \and \elet \x \e \E
  \and \eannot \E \tvs \t
  \andcr \erecord {\elaba = \ea\; \ldots\; \elabi = \E\; \ldots\; \elab_n = \en}
  \and \efield \E \elab
  \andcr \exrecord \T {\elaba = \ea\; \ldots\; \elabi = \E\; \ldots\; \elab_n = \en}
  \and \exfield \E \T \elab
  \andcr (\ea, \ldots, \E, \ldots, \en)
  \and \eproj \E j
  \and \exproj \E j n
  \andcr \epoly \E
  \and \expoly \E \tvs \ts
  \and \einst \E
  \and \exinst \E \tvs \ts
  \andcr
  \emagic {\ea, \ldots, \E, \ldots, \en}
}\\
\entry[Typing contexts]{\G}{
   \eset \and
   \G, x : \ts
}\\
\entry[Label environment]{\labenv}{
  \eset \and \labenv, \elab : \tfor \tvs {\trcd \T \tvs \to \t}
}
\end{bnfgrammar}
\end{mathpar}



\begin{judgboxmathpar}
  {\semenv \th \c}
  {Under the environment $\semenv$, the constraint $\c$ is satisfiable.}

  \infer[True]
    { }
    {\semenv \th \ctrue}

  \infer[Conj]
    {\semenv \th \ca \\
     \semenv \th \cb}
    {\semenv \th \ca \cand \cb}

  \infer[Exists]
    {\semenv\where{\tv \is \gt} \th \c}
    {\semenv \th \cexists \tv \c}

  \infer[Forall]
    {\forall \gt, ~ \semenv\where{\tv \is \gt} \th \c}
    {\semenv \th \tfor \tv \c}

  \infer[Unif]
    {\semenv(\ta) = \semenv(\tb)}
    {\semenv \th \cunif \ta \tb}

  \infer[Let]
    {\semenv \th \exists \tv. \ca \\
     \semenv\where{\x \is \semenv(\cabs \tv \ca)} \th \cb}
    {\semenv \th \clet \x \tv \ca \cb}

  \infer[App]
    {\semenv(\t) \in \semenv(\x)}
    {\semenv \th \capp \x \t}

  \infer[Match-Ctx]
    {\Cshape \C \t \sh \\
      \semenv \th \C\where{\cmatched \t \sh \cbrs}
    }
    {\semenv \th \C\where{\cmatch \t \cbrs}}

  \infer[Multi-Unif]
    {\forall \t \in \ueq,~ \semenv(\t) = \gt}
    {\semenv \th \ueq}

  \infer[LetR]
    {\semenv \th \cexists {\tv, \tvs} \ca \\
     \semenv\where{\x \is \semenv(\cabsr \tv \tvs \ca)} \th \cb}
    {\semenv \th \cletr \x \tv \tvs \ca \cb}

  \infer[AppR]
    {\greg \tv \semenvp \in \semenv(\x) \\
     \semenv(\t) = \semenvp(\tv) }
    {\semenv \th \capp \x \t}

  \infer[Exists-Inst]
    {\greg \tv\semenvp \in \semenv(\x) \\
     \semenv\where{\inst \is \semenvp} \th \c}
    {\semenv \th \cexistsi \inst \x \c}

  \infer[Incr-Inst]
    {\semenv(\inst)(\tv) = \semenv(\t) }
    {\semenv \th \cpinst \inst \tv \t}
\\
  \newcommand{\Srule}[3][]{{#2} &\eqdef& {#3} & {#1}}
  \newcommand{\Scases}[3]{%
    \left\{
      \begin{array}{l}
      #1\\[0.4ex]
      #2
      \end{array}%
    \right.
    &
    \hspace{-1ex}\begin{array}{l}
      \text{if } #3\\[0.4ex]
      \text{otherwise}
    \end{array}%
  }
  \begin{tabular}{RCLL}
    \labfrom \elab \T \leq \ta \to \tb &\eqdef&
      \Scases{\cexists \tvs \cunif \ta  {\trcd \T \tvs} \cand \cunif \tb \t}{\cfalse}{\labenv(\labfrom \elab \T) = \tfor \tvs {\trcd \T \tvs \to \t}}
    \\[1ex]
    \dom{\T} = \elabs &\eqdef& \Scases{\ctrue}{\cfalse}{\Dom {\labenv(\T)} = \elabs} \\[1ex]
    \Srule
      {(\tfor \tvs \tp) \leq \t}
      {\cexists \tvs \cunif \tp \t}
    \\[1ex]
    \Srule
      {\x \leq (\tfor \tvs \t)}
      {\cfor \tvs \capp \x \t}
    \\[1ex]
    \cmatched \t \sh {\cbranch \cpat \cs} &\eqdef&
    \Scases
      {\cexists \tvs \cunif \t \shapp \tvs \cand \theta(\ci)}
      {\cfalse}
      {\cmatches \cpati \sh \tvs \theta}
  \end{tabular}
\\
  \semenv(\cabsr \tv \tvs \c) \uad\eqdef\uad \set{\greg \tv {\semenv\where{\tv \is \gt, \tvs \is \gts}} \in \GroundRegion :
    \semenv\where{\tv \is \gt, \tvs \is \gts} \th \c}
\\
\semenv(\cabs \tv \c) \Wide\eqdef \
  \set {\gt \in \Ground : \semenv\where{\tv \is \gt} \th \c}
\\
\Cshape \C \t \sh \Wide\eqdef \
  \forall \semenv, \gt. \uad
      \semenv \th \cerase {\C\where{\cunif \t \gt}} \implies \shape \gt = \sh
\end{judgboxmathpar}

\begin{judgboxmathpar}
  {\Sh \preceq \Shp}
  {The shape $\Shp$ is an instance of $\Sh$. Alternatively, $\Shp$ is more general than $\Sh$.}
  \infer[Inst-Shape]
    {\tvcs_2 \disjoint \any {\tvcs_1} \t}
    {\any {\tvcs_1} \t \preceq
     \any {\tvcs_2} \t \where {\tvcs_1 \is \tys_1}}
\end{judgboxmathpar}

We write $\bot$ for the trivial shape $\any \tvc \tvc$. $\Shapes$ denotes the set of shapes and $\Shapesz$ is the set of non-trivial shapes.

\begin{definition}
A non-trivial shape $\Sh \in \Shapesz$ is the principal shape of the type
$\t$ iff:
\begin{enumerate}
  \item
    $\exists \typs,\ \t = \shapp[\Sh] \typs$
  \item
    $\forall \Shp \in \Shapesz, \forall \typs,\ \t = \shapp[\Shp] \typs
    \implies \Sh \preceq \Shp$
\end{enumerate}

A principal shape $\any \tvcs \t$ is \emph{canonical} if the sequence of its
free variables $\tvcs$ appear in the order in which the variables occur in
$\t$. $\shape \t$ is the canonical principal shape of $\t$.
\end{definition}

\begin{judgboxmathpar}
  {\cmatches \cpat \sh \tvcs \theta}
  {The pattern $\cpat$ matches the shape $\sh$ with fresh components $\tvcs$ binding\\pattern variables in $\theta$.}
  \\
  \newcommand{\Mrule}[5][]{{#2} \Matches {(#3)} \; #4 &\eqdef& {#5} & #1}
  \begin{tabular}{RCLL}
    \Mrule[\text{if } n \geq j]
      {\cpatprod \tv j}
      {\any \tvcs \Pi\iton \tvcs} \tvcps
      {[\tv \is \tvc_j']}
    \\[1ex]
    \Mrule
      {\cpatrcd \ct}
      {\any \tvcs \trcd \T \tvcs} \tvcps
      {[\ct \is \T]}
    \\[1ex]
    \Mrule
      {\cpatpoly \cscm}
      {\any \tvcs \tpoly \ts} \tvcps
      {[\cscm \is \ts \where{\tvcs \is \tvcps}]}
  \end{tabular}
\end{judgboxmathpar}

\begin{judgboxmathpar}
  {\c \simple}
  {The constraint $\c$ is simple.}
  \label{fig:simple}
  \inferrule[Simple-True]
    { }
    {\ctrue \simple}

  \inferrule[Simple-False]
    { }
    {\cfalse \simple}

  \inferrule[Simple-Conj]
    {\ca \simple \\ \cb \simple}
    {\ca \cand \cb \simple}

  \inferrule[Simple-Exists]
    {\c \simple}
    {\cexists \tv \c \simple}

  \inferrule[Simple-Forall]
    {\c \simple}
    {\cfor \tv \c \simple}

  \inferrule[Simple-Unif]
    { }
    {\cunif \ta \tb \simple}

  \inferrule[Simple-Let]
    {\ca \simple \\ \cb \simple}
    {\clet \x \tv \ca \cb \simple}

  \inferrule[Simple-App]
    { }
    {\capp \x \t \simple}

  \inferrule[Simple-LetR]
    {\ca \simple \\ \cb \simple}
    {\cletr \x \tv \tvs \ca \cb \simple}

  \inferrule[Simple-Exists-Inst]
    {\c \simple}
    {\cexistsi \inst \x \c \simple}

  \inferrule[Simple-Incr-Inst]
    { }
    {\cpinst \inst \tv \t \simple}
\end{judgboxmathpar}

We write $\semenv \thsimple \c$ for the satisfiability of a simple constraint
$\c$. \\

\begin{judgboxmathpar}
  {\C \simple}
  {The constraint context $\C$ is simple.}
  \label{fig:simple-context}

  \inferrule[Simple-Ctx-Hole]
    { }
    {\hole \simple}

  \inferrule[Simple-Ctx-Conj-Left]
    {\C \simple \\ \c \simple}
    {\C \cand \c \simple}

  \inferrule[Simple-Ctx-Conj-Right]
    {\C \simple \\ \c simple}
    {\c \cand \C \simple}

  \inferrule[Simple-Ctx-Exists]
    {\C \simple}
    {\cexists \tv \C \simple}

  \inferrule[Simple-Ctx-Forall]
    {\C \simple}
    {\cfor \tv \C \simple}

  \inferrule[Simple-Ctx-Let-Abs]
    {\C \simple \\ \c \simple}
    {\clet \x \tv \C \c \simple}

  \inferrule[Simple-Ctx-Let-In]
    {\c \simple \\ \C \simple}
    {\clet \x \tv \c \C \simple}

  \inferrule[Simple-Ctx-Exists-Inst]
    {\C \simple}
    {\cexistsi \inst \x \C \simple}
\end{judgboxmathpar}

\begin{judgboxmathpar}
  {\cerase \c}
  {The erasure of $\c$.}
  \label{fig:erasure}
\newcommand{\Erule}[2]{\cerase {#1} &\eqdef& {#2}}
\begin{tabular}{RCL}
  \Erule{\ctrue}{\ctrue} \\
  \Erule{\cfalse}{\cfalse} \\
  \Erule{\ca \cand \cb}{\cerase \ca \cand \cerase \cb} \\
  \Erule{\cexists \tv \c}{\cexists \tv \cerase \c} \\
  \Erule{\cfor \tv \c}{\cfor \tv \cerase \c} \\
  \Erule{\cunif \ta \tb}{\cunif \ta \tb} \\
  \Erule{\clet \x \tv \ca \cb}{\clet \x \tv {\cerase \ca} {\cerase \cb}} \\
  \Erule{\capp \x \t}{\capp \x \t} \\
    \Erule{\cmatch \t {\cbranch {\bar \cpat} {\bar \c}}}{\ctrue} \\
  \Erule{\cletr \x \tv \tvs \ca \cb}{\cletr \x \tv \tvs {\cerase \ca} {\cerase \cb}} \\
  \Erule{\cexistsi \inst \x \c}{\cexistsi \inst \x \cerase \c}\\
  \Erule{\cpinst \inst \tv \t}{\cpinst \inst \tv \t}
\end{tabular}
\end{judgboxmathpar}

\begin{judgboxmathpar}
  {\semenv \Th \c}
  {Under the semantic environment $\semenv$,
   the constraint $\c$ is canonically satisfiable.}
  \label{fig:canonical-sem}
  \inferrule[Can-Simple]
    {\semenv \thsimple \c}
    {\semenv \Th \c}

  \inferrule[Can-Match-Ctx]
    {\Cshape \C \t \sh \\ \semenv \Th \C\where{\cmatched \t \sh \cbrs}}
    {\semenv \Th \C\where{\cmatch \t \cbrs}}
\end{judgboxmathpar}

\begin{judgboxmathpar}
  {{\labfrom \elab \T} \leq \tya \to \tyb}
  {The label $\elab$ of the record $\T$ has the field type $\tyb$ and record type $\tya$.}
  \inferrule[Lab-Inst]
    {\labenv(\labfrom \elab \T) = \tfor \tvs {\trcd \T \tvs} \to \t }
    {{\labfrom \elab \T} \leq \trcd \T \tys \to \t\where{\tvs \is \tys} }
\end{judgboxmathpar}

\judgbox
  {\labuni \elab \T}
  {The label $\elab$ infers the unique record name $\T$.}

\begin{judgboxmathpar}
  {\labsuni \elabs \T}
  {The \emph{closed} set of labels $\elabs$ infer the unique record name $\T$.}
  \infer[Lab-Uni]
    {\elab \in \Dom {\labenv(\T)} \\
     \forall \Tp, \uad \elab \in \Dom {\labenv(\Tp)} \implies {\T} = \Tp}
    {\labuni \elab \T}

  \infer[Labs-Uni]
    {\Dom {\labenv(\T)} = \elabs \\
     \forall \Tp, \uad \Dom{\labenv(\Tp)} = \elabs \implies {\T} = \Tp}
    {\labsuni \elabs \T}
\end{judgboxmathpar}

\begin{judgboxmathpar}
  {\G \th \e : \ts}
  {Under the typing context $\G$, the term $\e$ is assigned the type $\ts $}
  \inferrule[Var]
    {x : \sigma \in \G}
    {\G \th x : \sigma}

  \inferrule[Fun]
    {\G, x : \ta \th e : \tb }
    {\G \th \efun x e : \ta \to \tb}

  \inferrule[App]
    {\G \th \ea : \ta \to \tb \\
     \G \th \eb : \ta}
    {\G \th \eapp \ea \eb : \tb}

  \inferrule[Unit]
    { }
    {\G \th () : 1}

  \inferrule[Annot]
    {\G \th e : \t\where {\tvs \is \tys}}
    {\G \th (e : \exi \tvs \t) : \t\where {\tvs \is \tys}}

  \inferrule[Gen]
    {\G \th e : \sigma \\ \tv \disjoint \G}
    {\G \th e : \tfor \tv \sigma}

  \inferrule[Inst]
    {\G \th e : \tfor \tv \ts}
    {\G \th e : \ts \where{\tv \is \t}}

  \inferrule[Let]
    {\G \th \ea : \sigma \\
     \G, x : \sigma \th \eb : \t}
    {\G \th \elet x \ea \eb : \t}
%% \color{gray}{
%%   \inferrule[Hole]
%%     {\G \th \e : \t}
%%     {\G \th \emagic \e : \tp}
%% }

%% \Subparabox[fig/ref/typing/core]{Typing rules --- Structural tuples}{}
  \inferrule[Tuple]
    {\parens{\G \th \ei : \ti}\iton}
    {\G \th (\ea, \ldots, \en) : \Pi\iton \ti}
\\
  \inferrule[Proj-X]
    {\G \th \e : \Pi\iton \ti \\
     1 \leq j \leq n}
    {\G \th \exproj \e j n : \tj}

  \inferrule[Proj-I]
    {\eshape \E \e {\any \tvcs \Pi\iton \tvcs} \\
     \G \th \E\where{\exproj \e j n} : \t}
    {\G \th \E\where{\eproj \e j} : \t}

  \inferrule [Poly-X]
    {\G \th \e : \ts\where {\tvs \is \tys}}
    {\G \th \expoly \e \tvs \ts : \tpoly {\ts \where {\tvs \is \tys}}}

  \inferrule [Poly-I]
    {\Eshape \E \e {{\any \tvcs \tpoly \ts}} \\
     \G \th \E \where{\expoly \e \tvcs \ts} : \t}
    {\G \th \E \where{\epoly \e} : \t}

  \inferrule [Use-X]
    {\G \th \e : \tpoly \ts \where {\tvs \is \tys}}
    {\G \th \exinst e \tvs \ts : \ts \where {\tvs \is \tys}}

  \inferrule [Use-I]
    {\eshape \E  \e {\any \tvcs \tpoly \ts} \\
     \G \th \E\where{\exinst \e \tvcs \ts} : \t}
    {\G \th \E\where{\einst \e} : \t}
\\
  \inferrule[Rcd-X]
    {\Dom {\labenv(\T)} = \elabs \\ \parens{{\labfrom \elabi \T} \leq \t \to \tyi}\iton \\ \parens{\G \th \ei : \tyi}\iton }
    {\G \th \exrecord \T {\elaba = \ea; \ldots; \elab_n = \en} : \t }
\\
  \inferrule[Rcd-Closed]
    {\labsuni \elabs \T \\ \G \th \exrecord \T {\overline{\elab = \e}} : \t}
    {\G \th \erecord {\overline{\elab = \e}} : \t}

  \inferrule[Rcd-I]
    {\Eshape \E \es {\any \tvcs \trcd \T \tvcs} \\ \G \th \E\where{\exrecord \T {\overline{\elab = \e}}} : \t }
    {\G \th \E\where{\erecord {\overline{\elab = \e}}} : \t}
\\
  \inferrule[Rcd-Proj-X]
    {{\labfrom \elab \T} \leq \tya \to \tyb \\ \G \th \e : \tya }
    {\G \th \exfield \e \T \elab : \tyb}
\\
  \inferrule[Rcd-Proj-Closed]
    {\labuni \elab \T \\ \G \th \exfield \e \T \elab : \t }
    {\G \th \efield \e \elab : \t}

  \inferrule[Rcd-Proj-I]
    {\eshape \E \e {\any \tvcs \trcd \T \tvcs} \\
     \G \th \E\where{\exfield \e \T \elab} : \t}
    {\G \th \E\where{\efield \e \elab} : \t}
\\
  \inferrule[Hole]
    {\parens{\G \th \ei : \ti}\iton}
    {\G \th \emagic {\bar \e} : \tp}

\def \Eqdef {&\eqdef&}
{\begin{tabular}{RCL}
\eshape \E \e \sh \Eqdef
  \forall \G, \t, \gt, \uad
  \G \th \eerase {\E \where {\emagic {\eannot \e {} \gt }}} : \t
      \wide\implies \shape \gt = \sh
\\[1ex]
\Eshape \E {\bar\e} \sh \Eqdef
  \forall \G, \t, \gt, \uad
      \G \th \eerase {\E\where{\eannotmagic {\bar \e} {} \gt}} : \t
      \wide\implies \shape \gt = \sh
\\[1ex]
\end{tabular}}
\end{judgboxmathpar}


\judgbox
  {\cinfer {\G \th \e} \t}
  {$\cinfer {\G \th \e} \t$ is satisfiable iff $\e$ has the expected \emph{known} type $\t$ under \emph{known} context $\G$.}

\judgbox
  {\cinfer \e \ts}
  {$\cinfer \e \ts$ is satisfiable iff $\e$ has the expected \emph{known} type scheme $\ts$.}

\begin{judgboxmathpar}
  {\cinfer \e \t}
  {$\cinfer \e \t$ is satisfiable iff $\e$ has the expected \emph{known} type $\t$.}
\newcommand {\Crule}[2]{#1 &\eqdef& #2}
\def \arraystretch{1.2}%4
\begin{tabular}{LCL}
\Crule
   {\cinfer x \t}
   {\cinst x \t}
\\
\Crule
  {\cinfer {()} \t}
  {\cunif \t \tunit}
\\
\Crule
  {\cinfer {\efun \x \e} \t}
  {\cexists {\tva, \tvb}
    \clet \x \tvp {\cunif \tvp \tv} {\cinfer \e \tvb}
    \cand \cunif \t {\tva \to \tvb}}
\\
\Crule
  {\cinfer {\eapp \ea \eb} \t}
  {\cexists {\tva, \tvb}
    \cinfer \ea {\tvb} \cand \cinfer \eb \tva
    \cand \cunif \tvb {\tva \to \t}}
\\
\Crule
  {\cinfer {\elet \x \ea \eb} \t}
  {\clet \x \tv {\cinfer \ea \tv} {\cinfer \eb \t}}
\\
\Crule
  {\cinfer {\eannot \e \tvs \tp} \t}
  {\cexists \tvs  \cinfer \e \tp \cand \cunif \t \tp}
\\
\Crule
  {\cinfer {\etuple {\ea, \ldots, \en}} \t}
  {\cexists \tvs \cunif \t {\tProd \tvs}
    \cand \cAnd \iton \cinfer \ei {\tv_i}}
\\
\Crule
  {\cinfer {\exproj \e j n} \t}
  {\cexists {\tv, \tvs}
    \cinfer \e \tv
    \cand \cunif \tv {\tProd \tvs}
    \cand \cunif \t {\tvj}}
\\
\Crule
  {\cinfer {\eproj \e j} \t}
  {\cexists \tv \cinfer \e \tv
    \cand \cmatch \tv {\cbranch {\cpatprod \tvb j} {\cunif \t \tvb}}}
\\
\Crule
  {\cinfer {\expoly \e \tvs \ts} \t}
  {\cexists {\tvs}
    \cinfer \e \ts
    \cand \cunif \t {\tpoly \ts}}
\\
\Crule
  {\cinfer {\exinst \e \tvs \ts} \t}
  {\cexists {\tvs, \tvb}
    \cinfer \e \tvb
    \cand \cunif \tvb {\tpoly \ts}
    \cand \ts \leq \t}
\\
\Crule
  {\cinfer {\einst \e} \t}
  {\cexists \tv
    \cinfer \e \tv
    \cand \cmatch \tv {\cbranch {\cpatpoly \cscm} \cscm \leq \t}}
\\
\Crule
  {\cinfer {\epoly \e} \t}
  {\clet \x \tv {\cinfer \e \tv}
    {\cmatch \t {\cbranch {\cpatpoly \cscm} {\x \leq \cscm}}}}
\\
\Crule
  {\cinfer {\efield \e \elab} \t}
  {\begin{cases}
    \cinfer {\exfield \e \T \elab} \t
    & \text{if } \labuni \elab \T
    \\
    \cexists \tv \cinfer \e \tv \cand
    \cmatch \tv
      {\cbranch {\cpatrcd \ct} {\labfrom \elab \ct \leq \tv \to \t}}
    & \text{otherwise}
  \end{cases}}
\\
\Crule
  {\cinfer {\exfield \e \T \elab} \t}
  {\cexists \tv \cinfer \e \tv \cand
   \labfrom \elab \T \leq \tv \to \t}
\\
\Crule
  {\cinfer {\erecord {\overline{\elab = \e}}} \t}
  {\begin{cases}
    \cinfer {\exrecord \T {\overline{\elab = \e}}} \t
    & \text{if } \labsuni \elabs \T \\
    \cexists \tvs \cAnd\iton \cinfer \ei \tvi & \text{otherwise} \\
    \uad\cand\uad \cmatch \t {\cbranch {\cpatrcd \ct}
      {\parens{\dom \ct = \elabs \cand \cAnd\iton \labfrom \elabi \ct \leq \t \to \tvi}}} &
   \end{cases}}
\\
\Crule
  {\cinfer {\exrecord \T {\overline{\elab = \e}}} \t}
  {\cexists \tvs \cAnd\iton \cinfer \ei \tvi \cand \dom {\T} = \elabs \cand \cAnd\iton \labfrom \elabi \T \leq \t \to \tvi }
\\
\Crule
  {\cinfer {\emagic \es} \t}
  {\cexists \tvs \cAnd\iton \cinfer \ei \tvi}
\\\\
\Crule
  {\cinfer \e {\tfor \tvs \t}}
  {\cfor \tvs \cinfer \e \t}
\\\\
\Crule
  {\cinfer {\eset \th \e} \t}
  {\cinfer \e \t}
\\
\Crule
  {\cinfer {\x : \ts, \G \th \e} \t}
  {\clet \x \tv {\ts \leq \tv} {\cinfer {\G \th \e} \t}}
\end{tabular}
\end{judgboxmathpar}
\emph{Note: When $\t$ is $\tv$ it is considered an \emph{unknown} expected type.}



\begin{judgboxmathpar}
  {\e \simple}
  {The term $\e$ is simple.}
  \inferrule[Simple-Var]
    { }
    {\x \simple}

  \inferrule[Simple-Fun]
    {\e \simple}
    {\efun \x \e \simple}

  \inferrule[Simple-App]
    {\ea \simple \\ \eb \simple}
    {\eapp \ea \eb \simple}

  \inferrule[Simple-Unit]
    { }
    {\eunit \simple}

  \inferrule[Simple-Let]
    {\ea \simple \\ \eb \simple}
    {\elet \x \ea \eb \simple}

  \inferrule[Simple-Annot]
    {\e \simple}
    {\eannot \e \tvs \t \simple}

  \inferrule[Simple-Tuple]
    {\parens {\ei \simple}\iton}
    {\etuple {\ea, \ldots, \en} \simple}

  \inferrule[Simple-Proj-X]
    {\e \simple}
    {\exproj \e j n \simple}

  \inferrule[Simple-Poly-X]
    {\e \simple}
    {\expoly \e \tvs \ts \simple}

  \inferrule[Simple-Use-X]
    {\e \simple}
    {\exinst \e \tvs \ts \simple}

  \inferrule[Simple-Rcd-X]
    {\parens {\ei \simple} \iton}
    {\exrecord \T {\elaba = \ea\; \ldots\; \elab_n = \en}}

  \inferrule[Simple-Rcd-Closed]
    {\parens {\ei \simple} \iton \\ \labsuni \elabs \T}
    {\erecord {\elaba = \ea\; \ldots\; \elab_n = \en}}

  \inferrule[Simple-Rcd-Proj-X]
    {\e \simple}
    {{\exfield \e \T \elab} \simple}

  \inferrule[Simple-Rcd-Proj-Closed]
    {\e \simple \\ \labuni \elab \T}
    {\efield \e \elab \simple}

  \inferrule[Simple-Hole]
    {\parens{\ei \simple}\iton}
    {\emagic \es \simple}
\end{judgboxmathpar}

\label {app/ref/eerase}
\begin{judgboxmathpar}
  {\eerase \e}
  {The erasure of $\e$.}
\newcommand{\Erule}[2]{\eerase {#1} &\eqdef& {#2}}
  \begin{tabular}{RCL}
  \Erule{\x}{\x} \\
  \Erule{\efun \x \e}{\efun \x \eerase \e} \\
  \Erule{\eapp \ea \eb}{\eapp {\eerase \ea} {\eerase \eb}} \\
  \Erule{\eunit}{\eunit} \\
  \Erule{\elet \x \ea \eb}{\elet \x {\eerase \ea} {\eerase \eb}} \\
  \Erule{\eannot \e \tvs \t}{\eannot {\eerase \e} \tvs \t} \\
  \Erule{\etuple {\ea, \ldots, \en}}{\etuple {\eerase \ea, \ldots, \eerase \en}} \\
  \Erule{\eproj \e j}{\emagic {\eerase \e}} \\
  \Erule{\exproj \e j n}{\exproj {\eerase \e} j n} \\
  \Erule{\expoly \e \tvs \ts}{\expoly {\eerase \e} \tvs \ts} \\
  \Erule{\epoly \e}{\emagic {\eerase \e}}\\
  \Erule{\einst \e}{\emagic {\eerase \e}}\\
  \Erule{\exinst \e \tvs \ts}{\exinst {\eerase \e} \tvs \ts}\\
    \Erule{\erecord {\elaba = \ea; \ldots; \elab_n = \en}}{\begin{cases}
      \erecord {\elaba = \eerase \ea; \ldots; \elab_n = \eerase \en} &\text{if } \labsuni \elabs \T \\
      \emagic {\eerase \ea, \ldots, \eerase \en} & \text{otherwise}
    \end{cases}}\\
  \Erule{\exrecord \T {\elaba = \ea; \ldots; \elab_n = \en}}{\exrecord \T {\elaba = \eerase \ea; \ldots; \elab_n = \eerase \en}}\\
    \Erule{\efield \e \elab}{\begin{cases}
      \efield {\eerase \e} \elab & \text{if } \labuni \elab \T \\
      \emagic {\eerase \e} & \text{otherwise}
    \end{cases}}\\
  \Erule{\exfield \e \T \elab}{\exfield {\eerase \e} \T \elab}\\
  \Erule{\emagic \es}{\emagic {\parens {\eerase \ei} \iton}}\\
\end{tabular}
\end{judgboxmathpar}

\begin{judgboxmathpar}
  {\G \thsimplesd \e : \t}
  {Under the typing context $\G$, the simple term $\e$ has the type $\t$.}
\\
  \inferrule[Var-SD]
    {x : \tfor \tvs \t \in \G}
    {\G \thsimplesd x : \t\where{\tvs \is \tys}}

  \inferrule[Let-SD]
    {\G \thsimplesd \ea : \ta\\
     \tvs \disjoint \fvs \G \\
     \G, x : \tfor \tvs \ta \thsimplesd \eb : \tb}
    {\G \thsimplesd \elet x \ea \eb : \tb}

  \inferrule [Poly-X]
    {\G \thsimplesd \e : \t\where {\tvs \is \tys} \\ \tvbs \disjoint \G}
    {\G \thsimplesd \expoly \e \tvs {\tfor \tvbs \t} : \tpoly {\ts \where {\tvs \is \tys}}}

  \inferrule [Use-X]
    {\G \th \e : \tpoly {\tfor \tvbs \t} \where {\tvs \is \tys}}
    {\G \th \exinst e \tvs {\tfor \tvbs \t} : \t \where {\tvs \is \tys, \tvbs \is \typs}}
\end{judgboxmathpar}

\begin{judgboxmathpar}
  {\Th \e : \t}
  {The term $\e$ canonically has the type $\t$.}
  \inferrule[Can-Base]
    {\eset \thsimplesd \e : \t}
    {\Th \e : \t}

  \inferrule[Can-Proj-I]
    {\eshape \E \e {\any \tvcs \Pi\iton \tvcs} \\
     \Th \E\where{\exproj \e j n} : \t}
    {\Th \E\where{\eproj \e j} : \t}

  \inferrule [Can-Poly-I]
    {\Eshape \E \e {{\any \tvcs \tpoly \ts}} \\
     \Th \E \where{\expoly \e \tvcs \ts} : \t}
    {\Th \E \where{\epoly \e} : \t}

  \inferrule [Can-Use-I]
    {\eshape \E  \e {\any \tvcs \tpoly \ts} \\
     \Th \E\where{\exinst \e \tvcs \ts} : \t}
    {\Th \E\where{\einst \e} : \t}

  \inferrule[Can-Rcd-I]
    {\Eshape \E \es {\any \tvcs \trcd \T \tvcs} \\
     \Th \E\where{\exrecord \T {\elaba = \ea; \ldots; \elab_n = \en}} : \t}
    {\Th \E\where{\erecord {\elaba = \ea; \ldots; \elab_n = \en}} : \t}

  \inferrule[Can-Rcd-Proj-I]
    {\Eshape \E \e {\any \tvcs \trcd \T \tvcs} \\
     \Th \E\where{\exfield \e \T \elab} : \t}
    {\Th \E\where{\efield \e \elab} : \t}
\end{judgboxmathpar}


\label {app/rules/unif}
\begin{judgboxmathpar}
  {\up \unif \upp}
  {The unifier rewrites $\up$ to $\upp$.}
   \rewrite[U-Exists]
      {(\cexists \alpha \upa) \cand \upb }{ \tv \disjoint \upb}
      {\cexists \tv {\upa \cand \upb}}

    \rewrite[U-Cycle]
      {\up }{ \cyclic \up}
      {\cfalse}

    \rewrite[U-True]
      {\up \cand \ctrue}
      {}
      {\up}

    \rewrite[U-False]
      {\Up\where{\cfalse}}
      { \Up \neq \hole}
      {\cfalse}

    \rewrite[U-Merge]
      {\cunif \tv \ueqa \cand \cunif \tv \ueqb}
      {}
      {\cunif \tv {\cunif \ueqa \ueqb}}

    \rewrite[U-Stutter]
      {\cunif \tv {\cunif \tv \ueq}}
      {}
      {\cunif \tv \ueq}

    \rewrite[U-Name]
      {\cunif {\pshapp \parens{\tys, \ti, \typs}} \ueq }
      {\tv \disjoint \tys, \typs, \ueq \\ \ti \notin \TyVars}
      {\cexists \tv
        {\cunif \tv \ti \cand
         \cunif {\pshapp \parens{\tys, \tv, \typs}} \ueq}}

    \rewrite[U-Decomp]
      {\cunif {\pshapp \tvs} {\cunif {\pshapp \tvbs} \ueq}}
      {}
      {\cunif {\pshapp \tvs} \ueq \cand \cunif \tvs \tvbs}

    \rewrite[U-Clash]
      {\cunif {\pshapp \tvs} {\cunif {\pshapp[\shp]\tvbs } \ueq }}{
       \sh \neq \shp}
      {\cfalse}

    \rewrite[U-Trivial]
      {\ueq}
      {|\ueq| \leq 1}
      {\ctrue}
\end{judgboxmathpar}


\begin{judgboxmathpar}
  {\c \csolve \cp}
  {The constraint solver rewrites $\c$ to $\cp$.}

  \rewrite[S-Unif]
    {\upa}
    {\upa \unif \upb}
    {\upb}

  \rewrite[S-True]
    {C \cand \ctrue}
    {}
    {C}

  \rewrite[S-False]
    {\C\where\cfalse}
    {\C \neq \hole}
    {\cfalse}

  \rewrite[S-Let]
    {\clet \x \tv \ca \cb}
    {}
    {\cletr \x \tv \eset \ca \cb}

  \rewrite[S-Exists-Conj]
    {(\cexists \alpha \ca) \cand \cb }{
     \tv \disjoint \cb}
    {\cexists \tv {\ca \cand \cb}}

  \rewrite[S-Let-ExistsLeft]
    {\cletr \x \tv \tvs {\cexists \tvb \ca} \cb }{
     \tvb \disjoint \tv, \tvs, \cb}
    {\cletr \x \tv {\tvs, \tvb} \ca \cb}

  \rewrite[S-Let-ExistsRight]
    {\cletr \x \tv \tvs \ca {\cexists \tvb \cb} }{
     \tvb \disjoint \tv, \tvs, \ca}
    {\cexists \tvb {\clet \x \tvs \ca \cb}}

  \rewrite[S-Let-ConjLeft]
    {\cletr \x \tv \tvs {\ca \cand \cb} \cc }{
     \ca \disjoint \tv, \tvs}
    {\ca \cand \cletr \x \tv \tvs \cb \cc}

  \rewrite[S-Let-ConjRight]
    {\cletr \x \tv \tvs \ca (\cb \cand \cc) }{
     \x \disjoint \cc}
    {\cc \cand \Clet \x \tv \ca \cb}

  \rewrite[S-Match-Ctx]
    {\C\where{\cmatch \t \cbrs}}
    {\th \Cshape \C \t \sh}
    {\C\where{\cmatched \t {\sh} \cbrs}}

   \rewrite[S-Inst-Name]
    {\cpinst \inst \tv \t}
    {\t \notin \TyVars}
    {\cexists \tvc \cunif \tvc \t \cand \cpinst \inst \tv \tvc}

  \rewrite[S-Let-AppR]
    {\cletr \x \tv \tvs \c {\C\where{\capp \x \t}}}
    {\tvc \disjoint \t \\ \x \disjoint \bvs \C}
    {\cletr \x \tv \tvs \c {\C\where{\cexistsi
      {\tvc, \inst} \x { \cpinst \inst \tv \tvc \cand \cunif \tvc \t}}}}

  \rewrite[S-Inst-Copy]
    {\cletr \x \tv \tvs {\c} \C\where{\cpapp \x \tvp \tvc \inst}}
    {\c = \cp \cand \cunif \tvp {\cunif {\shapp \tvbs} \ueq}\\
     \tvp \in \reg \tv \tvs \\\\
     \neg \cyclic {\c} \\
     \tvbs' \disjoint \tvp, \tvc, \tvbs \\
     \x \disjoint \bvs \C}
    {\cletr \x \tv \tvs {\c}
      \C\where{\cexists {\tvbs'}
         {\cunif \tvc {\shapp \tvbs'} \cand \cpapp \x {\tvbs} {\tvbs'} \inst}}}

  \rewrite[S-Inst-Unify]
    {\cpinst \inst \tv \tvca \cand \cpinst \inst \tv \tvcb}
    {}
    {\cpinst \inst \tv \tvca \cand \cunif \tvca \tvcb}

  \rewrite[S-Inst-Poly]
    {\cletr \x \tv {\tvs} {\ueqs \cand \c}
        {\C\where{\cpapp \x \tvp \tvc \inst}}}
    {\cfor \tvp \cexists {\tv} {\ueqs} \cequiv \ctrue \\\\
     \tvp \in \reg \tv \tvs \\
     \tvp \disjoint \c \\
     \inst.\tvp \disjoint \insts \C \\
     \x \disjoint \bvs \C}
    {\cletr \x \tv {\tvs} {\ueqs \cand \c} {\C\where\ctrue}}

  \rewrite[S-Inst-Mono]
    {\cletr \x \tv \tvs \c {\C\where{\cpapp \x \tvb \tvc \inst}}}
    {\tvb \notin \reg \tv \tvs \\ \x, \tvb \disjoint \bvs \C}
    {\cletr \x \tv \tvs \c {\C\where{\cunif \tvb \tvc}}}

  \rewrite[S-Compress]
    {\cletr \x \tv {\tvs, \tvb}
       {\ca \cand \cunif \tvb {\cunif \tvc \ueq}} {\cb}}
    {\tvb \neq \tvc}
    {\cletr \x \tv {\tvs}
       {\ca\where{\tvb \is \tvc} \cand \cunif \tvc {\ueq\where{\tvb \is \tvc}}}
       {\cb\where{\x.\tvb \is \tvc}}}

  \rewrite[S-Gc]
    {\cletr \x \tv {\tvs, \tvb} {\ca \cand \cunif \tvb \ueq} \cb}
    {\tvb \disjoint \ca, \ueq, \cb}
    {\cletr \x \tv {\tvs} {\ca \cand \ueq} \cb}

  \rewrite[S-Exists-Lower]
    {\cletr \x \tv {\tvas, \tvbs} \ca \cb}
    {\th \cdetermines {\cexists {\tv, \tvas} \ca} \tvbs}
    {\cexists \tvbs \cletr \x \tv \tvas \ca \cb}

  \rewrite[S-Exists-Exists-Inst]
    {\cexistsi \inst \x \cexists \tv \c}
    {}
    {\cexists \tv \cexistsi \inst \x \c}

  \rewrite[S-Exists-Inst-Conj]
    {\cexistsi \inst \x \ca \cand \cb}
    {\inst \disjoint \ca}
    {\ca \cand \cexistsi \inst \x \cb}

  \rewrite[S-Exists-Inst-Let]
    {\cletr \x \tv \tvs \ca {\cexistsi \inst \xp \cb}}
    {\x \neq \xp}
    {\cexistsi \inst \xp \cletr \x \tv \tvs \ca \cb}

  \rewrite[S-Exists-Inst-Solve]
    {\cexistsi \inst \x \c}
    {\inst \disjoint \c}
    {\c}

  \rewrite[S-All-Conj]
    {\cfor \tvs {\cexists \tvbs {\ca \cand \cb}}}
    {\tvs, \tvbs \disjoint \ca}
    {\ca \cand \cfor \tvs {\cexists \tvbs \cb}}

  \rewrite[S-Exists-All]
    {\cfor \tvs {\cexists {\tvbs, \tvcs} \c}}
    {\th \cdetermines {\cexists {\tvs, \tvbs} \c} \tvcs}
    {\cexists \tvcs \cfor \tvs {\cexists \tvbs \c}}

  \rewrite[S-All-Escape]
    {\cfor {\tvs, \tv} {\cexists \tvbs {\c \cand \ueqs}}}
    {\tv \prec_{\ueqs}^* \tvc \\
     \tvc \disjoint \tv, \tvbs \\
     \tv \disjoint \tvbs}
    {\cfalse}

  \rewrite[S-All-Rigid]
    {\cfor {\tvs, \tv}
      {\cexists \tvbs \c \cand \cunif \tv {\cunif \t \ueq}}}
    {\t \notin \TyVars \\ \tv \disjoint \tvbs}
    {\cfalse}

  \rewrite[S-All-Solve]
    {\cfor \tvs \cexists \tvbs \ueqs}
    {\cexists \tvbs \ueqs \cequiv \ctrue}
    {\ctrue}
\end{judgboxmathpar}

\begin{judgboxmathpar}
  {\th \Cshape \C \t \sh}
  {Under $\C$, the type $\t$ has the provably unique canonical shape $\sh$.}
  \infer[S-Uni-Var]
    {\color{gray}\tv \disjoint \bvs \Cb}
    {\th \Cshape {\Ca\where{\cunif \tv {\cunif \t \ueq} \cand \Cb\where{-}}} \tv {~\shape \t}}

  \infer[S-Uni-Type]
    {{\color{gray}\t \notin \TyVars}}
    {\th \Cshape \C \t {~\shape \t}}

  \infer[S-Uni-BackProp]
    {\th \Cshape{\cletr \x \tv \tvs {\Ca\where{\ctrue}} {\Cb\where{\cpapp \x \tvp \tvc \inst \cand -}}} \tvc \sh \\
     \color{gray}\tvp \in \tv, \tvs \\
     \color{gray}\x \disjoint \bvs \Cb \\
     \color{gray}\tvp \disjoint \bvs \Ca}
    {\th \Cshape{\cletr \x \tv \tvs {\Ca\where{-}} {\Cb\where{\cpapp \x \tvp \tvc \inst}}} \tv \sh}

\end{judgboxmathpar}

\begin{definition}
  $\cdetermines \c \tvbs$ if and only if every ground assignments
  $\semenv$ and $\semenvp$ that satisfy (the erasure of) $\c$ and coincide outside of $\tvb$
  coincide on $\tvbs$ as well.
  \begin{mathpar}
    \cdetermines \c \tvb \uad\eqdef\uad \all {\semenv, \semenvp} \uad
      \semenv \th \cerase \c
      \wedge \semenvp \th \cerase \c
      \wedge \semenv =_{\setminus \tvbs} \semenvp
      \implies
      \semenv = \semenvp
  \end{mathpar}
\end{definition}

\begin{judgboxmathpar}
  {\th \cdetermines \c \tvs}
  {$\c$ provably determines $\tvs$.}

  \inferrule
    [S-Det-Dom]
    {\tvc \disjoint \tvbs, \tvas \\ \tvs \subseteq \fvs \ueq}
    {\th \cdetermines {\cexists \tvbs \c \cand \cunif \tvc \ueq} \tvs}

  \inferrule
    [S-Det-Esc]
    {\fvs \t \disjoint \tvs, \tvbs}
    {\th \cdetermines {\cexists \tvbs \c \cand \cunif \tvs {\cunif \t \ueq}} \tvs}
\end{judgboxmathpar}

\begin{judgboxmathpar}
  {\insts \c}
  {The set of instantiations in $\c$.}
  \newcommand{\Srule}[2]{#1 &\eqdef& #2}
  \begin{tabular}{RCL}
    \Srule{\insts \ctrue}{\eset}\\
    \Srule{\insts \cfalse}{\eset}\\
    \Srule{\insts {\ca \cand \cb}}{\insts \ca \cup \insts \cb}\\
    \Srule{\insts {\cexists \tv \c}}{\insts \c}\\
    \Srule{\insts {\cfor \tv \c}}{\insts \c}\\
    \Srule{\insts {\cunif \t \tp}}{\eset}\\
    \Srule{\insts {\clet \x \tv \ca \cb}}{\insts \ca \cup \insts \cb}\\
    \Srule{\insts {\capp \x \t}}{\eset}\\
    \Srule{\insts {\ueq}}{\eset}\\
    \Srule{\insts {\cletr \x \tv \tvs \ca \cb}}{\insts \ca \cup \insts \cb}\\
    \Srule{\insts {\cexistsi \inst \x \c}}{\insts \c}\\
    \Srule{\insts {\cpapp \x \tv \tvc \inst}}{\set {\inst.\tv}}
  \end{tabular}
\end{judgboxmathpar}


\begin{definition}[Measure]
  For the relation $\semenv \th \c$, the following measure enables a useful
  induction principle:
    \begin{mathpar}
    \cmeasure \c \uad\eqdef\uad \angles{\cnmatches \c, \csize \c}
  \end{mathpar}
  where $\angles \ldots$ denotes a pair with lexicographic ordering, and:
  \begin{enumerate}

    \item $\cnmatches \c$ is the number of $\cmatch \t \cbrs$ constraints in
      $\c$.

    \item the last component $\csize \c$ is a structural measure of constraints \ie a
      conjunction $\ca \cand \cb$ is larger than the two conjuncts $\ca,
      \cb$.

  \end{enumerate}
\end{definition}

\begin{judgboxmathpar}
  {\ctxinfer \E \tp \t}
  {$\ctxinfer \E \tp \t$ is a satisfiable context iff the context $\E$ has \\ the expected type $\t$ given the hole has the type $\tp$.}
\newcommand {\Crule}[2]{#1 &\eqdef& #2}
\def \arraystretch{1.4}%4
  \scalebox{0.9}{
\begin{tabular}{LCL}
  \Crule{\ctxinfer \hole \t \t}{\hole}\\
  \Crule{\ctxinfer {\parens{\eapp \E \e}} \tp \t}{\cexists {\tva, \tvb} \cunif \tva {\tvb \to \t} \cand \ctxinfer \E \tp \tva \cand \cinfer \e \tvb}\\
  \Crule{\ctxinfer {\parens{\eapp \e \E}} \tp \t}{\cexists {\tva, \tvb} \cunif \tva {\tvb \to \t} \cand \cinfer \e \tva \cand \ctxinfer \E \tp \tvb}\\
  \Crule{\ctxinfer {\parens{\elet \x \E \e}} \tp \t}{\clet \x \tv {\ctxinfer \E \tp \tv} {\cinfer \e \t}}\\
  \Crule{\ctxinfer {\parens{\elet \x \e \E}} \tp \t}{\clet \x \tv {\cinfer \e \tv} {\ctxinfer \E \tp \t}}\\
  \Crule{\ctxinfer {{\eannot \E \tvs \tpp}} \tp \t}{\cexists \tvs \cunif \t \tpp \cand \ctxinfer \E \tp \t}\\
  \Crule{\ctxinfer {\etuple {\ea, \ldots, \E_j, \ldots, \en}} \tp \t}{\cexists \tvs \cunif \t {\tProd \tvi} \cand \cAnd_{i \neq j} \cinfer \ei \tvi \cand \ctxinfer {\E_j} \tp \tvj} \\
  \Crule{\ctxinfer {\parens{\exproj \E j n}} \tp \t}{\cexists {\tv, \tvs} \ctxinfer \E \tp \tv \cand \cunif \tv {\tProd \tvi} \cand \cunif \t \tvj}\\
  \Crule{\ctxinfer {\parens{\eproj \E j}} \tp \t}{\cexists \tv \ctxinfer \E \tp \tv \cand \cmatch \tv {\cbranch {\cpatprod \tvc j} {\cunif \t \tvc}}}\\
  \Crule{\ctxinfer {\expoly \E \tvs \ts} \tp \t}{\cexists \tvs \ctxinfer \E \tp \ts \cand \cunif \t {\tpoly \ts}}\\
  \Crule{\ctxinfer {\exinst \E \tvs \ts} \tp \t}{\cexists {\tvs, \tvb} \ctxinfer \E \tp \tvb \cand \cunif \tvb {\tpoly \ts} \cand \cleq \ts \t}\\
  \Crule{\ctxinfer {\epoly \E} \tp \t}{\clet \x \tv {\ctxinfer \E \tp \tv} {\\&&\cmatch \t {\cbranch {\cpatpoly \cscm} {\cleq \x \cscm}}}}\\
  \Crule{\ctxinfer {\einst \E} \tp \t}{\cexists \tv {\ctxinfer \E \tp \tv \cand \cmatch \tv {\cbranch {\cpatpoly \cscm} {\cleq \cscm \t}}}}\\
\Crule
  {\ctxinfer {\parens{\efield \E \elab}} \tp \t}
  {\begin{cases}
    \ctxinfer {\parens{\exfield \E \T \elab}} \tp \t & \text{if } \labuni \elab \T \\
    \cexists \tv \ctxinfer \E \tp \tv \\ \cand ~ \cmatch \tv {\cbranch {\cpatrcd \ct} {\labfrom \elab \ct \leq \tv \to \t}} & \text{otherwise}
  \end{cases}}
\\
\Crule
  {\ctxinfer {\parens{\exfield \E \T \elab}} \tp \t}
  {\cexists \tv \ctxinfer \E \tp \tv \cand \labfrom \elab \T \leq \tv \to \t}
\\
\Crule
  {\ctxinfer {\erecord {\elaba = \ea; \ldots; \elab_j = \E_j; \ldots; \elab_n = \en}} \tp \t}
  {\begin{cases}
    \ctxinfer {\exrecord \T { \ldots; \elab_j = \E_j; \ldots}} \tp \t & \text{if } \labsuni \elabs \T \\
    \cexists \tvs \cAnd_{i \neq j} \cinfer \ei \tvi \cand \ctxinfer {\E_j} \tp \tvj & \text{otherwise} \\
    \uad\cand~\cmatch \t {\cbranch {\cpatrcd \ct}
      {\dom {\ct} = \elabs \\
      \hspace{3cm}\cand \cAnd\iton \labfrom \elabi \ct \leq \t \to \tvi}}
  \end{cases} }
\\
\Crule
  {\ctxinfer {\exrecord \T {\elaba = \ea; \ldots; \elab_j = \E_j; \ldots; \elab_n = \en}} \tp \t}
  {\cexists \tvs \cAnd_{i \neq j} \cinfer \ei \tvi \cand \ctxinfer {\E_j} \tp \tvj \cand \dom {\T} = \elabs \\
  &&\cand \cAnd\iton \labfrom \elabi \T \leq \t \to \tvi }
\\
\Crule
  {\ctxinfer {\parens{\emagic {\ea, \ldots, \E_j, \ldots, \en}}} \tp \t}
  {\cexists \tvs \cAnd_{i \neq j} \cinfer \ei \tvi \cand \ctxinfer {\E_j} \tp \tvj}
\\\\
\Crule
  {\ctxinfer \E \tp {\tfor \tvs \t}}
  {\cfor \tvs \ctxinfer \E \tp \t}
\\\\
\end{tabular}
  }
\end{judgboxmathpar}



\section{Properties of the constraint language}
\label{app:proofs-constraints}

This appendix establishes key properties of the constraint language. The first
is the principality of shapes \cref{thm:principal-shapes}: any non-variable type
$\t$ admits a non-trivial principal shape $\sh$.

The second is the canonicalization of satisfiability derivations $\semenv \th
\c$, which enables a simple induction principal for reasoning about unicity.
This canonical form for derivations is a crucial tool in our proof of
soundness and completeness in \cref{app/oml/proofs}.

\subsection{Principality of shapes}

\principalShapesBIS*
\begin{proof}
  Let us assume $\t$ is a non-variable type.

  \begin{proofcases}
    \proofcase{$\t$ is a type constructor $\tconstr \tys$}

    $\tconstr$ is a top-level type constructor of arity $n$, which in
    our setting may be the nullary $\tunit$, the binary arrow,
    the $n$-ary product, or a nominal record type. In all
    these cases, the shape of $\t$ is $\any \tvcs \tconstr \tvcs$
    where $\tvcs$ is a sequence of $n$ distinct type variables. This
    is clearly principal.

    \proofcase{$\t$ is a polytype $\tpoly {\tfor \tvs \t}$}

    We may assume \Wlog that each variable of $\tvs$ occurs free in
    $\t$.
    %
    Let $(\pi_i)\iton$ be the sequence of shortest paths in $\t$ that cannot be
    extended to reach a (polymorphic) variable in $\tvas$, in lexicographic
    order and $\tvcs$ be a sequence $(\tvci)\iton$ of distinct variables that do
    not appear in~$\t$.
    %
    Let $\tyz$ be $\t \where {\pi_i \is \tvci}\iton$, \ie the term $\t$ where each
    path $\pi_i$ has been substituted by the variable $\tvci$.  Let $\Sh$ be the
    shape $\any \tvcs {\tpoly {\all \tvs \tyz}}$.
    We claim that $\Sh$ is actually the principal shape of $\tpoly {\all \tvs
    \t}$.

    \medskip
    \locallabelreset

    By construction, $\t$ is equal to $\shapp[\Sh] \tys$~\llabel 1.
    where $\tys$ is the sequence composed of $\ti$ equal to $\t/\pi_i$
    for $i$ ranging from $1$ to $n$.
    %
    Indeed, by
    definition, $\shapp[\Sh] \tys$ is equal to $(\t\where {\pi_i \is \tvci}\iton)
    \where {\tvci \is \ti}$ which is obviously equal to $\t$.
    The remaining of the proof checks that $\Sh$ is minimal~\llabel 2, that is,
    we assume that $\Sh'$ is another shape such that $\tpoly {\all\tvs\t}$ is
    equal to $\shapp [\Shp] \typs$ for some $\typs$~\llabel H and show that $\Sh
    \preceq \Shp$~\llabel C.

    \medskip

    It follows from~\lref H that
      $\Shp$ must be a polytype shape, \ie of the form $\any \tvcps {\tpoly
      {\all \tvbs \typ}}$ and
      $\tpoly {\all \tvs \t}$ is equal to $\tpoly {\all\tvbs \tp} \where {\tvcps
      \is \typs}$~\llabel{P}.
    \relax
    We may assume \Wlog that $\tvbs$ and $\tvcps$ are disjoint, that
    $\tvcps$ does not contain useless variables, \ie
    that they all appear in $\tp$ and that they actually appear in lexicographic
    order.
    \relax
    Now that never term contains useless variables, \lref P implies that the
    sequences $\tvas$ and $\tvbs$ can be put in one-to-one correspondences.
    Besides, since they all ordered in the order of appearance in terms, they
    the correspondence respects the ordering. Hence, the substitution $\where
    {\tvbs \is \tvas}$ is a renaming. Therefore, we can assume \Wlog that
    $\tvbs$ is $\tvas$,
    \relax
    That is, \lref P becomes that $\tpoly {\all \tvs \t}$ is equal to $\tpoly
    {\all \tvs \typ \where {\tvcps \is \typs}}$, which given that variables
    $\tvs$ appear in the same order in both terms, implies that $\t$ is
    equal to $\typ \where {\tvcps \is
    \typs}$~\llabel T.

    \relax

    \medskip

    Since $\typs$ does not contain any variable in $\tvs$, every path $\pi_i$
    is a path in $\typ$. Thus, we may write $\typ$ as
    \relax $\typ \where {\pi_i \is \tyi''}\iton$ where $\tyi''$ is $\typ/\pi_i$.
    This is also equal to
    \relax $(\typ \where {\pi_i \is \tvci}\iton) \where {\tvci \is \tyi''}\iton$,
    that is $\tyz\where {\tvci \is \tyi''}\iton$.
    %
    In summary, we have $\typ$ is equal to
    \relax $\tyz \where {\tvci \is \tyi''}\iton$,
    which implies that
    \relax  $\tpoly {\all \tvs \typ}$ is equal to
    \relax  $\tpoly {\all \tvs {\tyz \where {\tvci \is \tyi''}\iton}}$, \ie
    \relax  $\tpoly {\all \tvs \tyz} \where {\tvci \is \tyi''}\iton$~\llabel E.
    %
    By \Rule {Inst-Shape}, we have
    \begin{mathpar}[inline]
    \any \tvcs  \tpoly {\all \tvs \tyz} \preceq
    \any \tvcps\tpoly {\all \tvs \tyz} \where {\tvci \is \tyi''}\iton,
    \end{mathpar}
    which, given~\lref E, is exactly~\lref C.

  \end{proofcases}
\end{proof}

\subsection{Canonicalization of satisfiability}

They key result in this section is that our semantic derivations $\semenv \th
\c$ can always be rewritten to only apply the rule \Rule{Match-Ctx} at the very
bottom of the derivation, rather than in the middle of derivations. This
corresponds to explicitating the unique shapes of all suspended constraints (in
some order that respects the dependency between suspended constraints), and
then continuing with a syntax-directed proof of a fully-discharged constraint.

We did not impose this ordering in our definition of the semantics to make it
more flexible and more declarative, but the inversion principle that it
provides will be helpful when reasoning about the solver in
\cref{app:proofs-solving}.

We define in \cref{fig:simple} a formal judgment $\c \simple$ that says that
$\c$ does not contain any suspended match constraint, and extend it trivially
to constraint contexts: $\C \simple$. In particular, the erasure $\cerase \c$
of a constraint (\cref{def:erasure}) is always simple. We then introduce in
\cref{fig:canonical-sem} a ``canonical'' semantic judgment $\semenv \Th \c$
that enforces the structure we mentioned: its derivation starts by discharging
suspended constraints, until eventually we reach a simple constraint $\c$.
Below we prove that any semantic derivation $\semenv \th \c$ can be turned into
a canonical semantic derivation $\semenv \Th \c$.

We can think of this result as controlling the amount of non-syntax-directness in our rules: we need some of it, but it suffices to have it only at the outside, and it contains a more standard derivation that is easy to reason about.

\paragraph{Inversion} When $\c$ is simple, a derivation of $\semenv \th \c$
does not use the contextual rule (it is a derivation in $\semenv \thsimple
\c$), so it enjoys the usual inversion principle on syntax-directed judgments;
for example, if $\semenv \thsimple {\ca \cand \cb}$ then by inversion $\semenv
\thsimple \ca$ and $\semenv \thsimple \cb$, etc.

\paragraph{Congruence} Congruence does not hold in general in our system due to
the contextual rule. For example, $\ca \eqdef (\cmatch \tva {\cbranch \wild
\ctrue})$ is unsatisfiable so we have $\ca \cequiv \cfalse$, but for $\C \eqdef
(\cexists \tva \cunif \tva \tint \cand \hole)$ we have $\C \where \ca \cequiv
\ctrue$ and $\C \where \cfalse \equiv \cfalse$. It holds simply for simple
constraints.

\begin{lemma}[Simple congruence]
  \label{lem:cong-simple}
  Given simple constraints $\ca, \cb$ and simple context $\C$.
  If \\$\ca \centails \cb$, then $\C\where{\ca} \centails \C\where{\cb}$.

  \begin{proof}
    Induction on the derivation of $\C \simple$.
  \end{proof}
\end{lemma}

\paragraph{Composability}

The composability result below is an important test of our definition of the
unicity condition $\Cshape \C \t \sh$, which is in part engineered for this
lemma to be simple to prove. In the past we used a definition of unicity
that also required $\C \where \ctrue$ to be satisfiable, which broke the
composability property.
\begin{lemma}[Composability of unicity]
  \label{lem:compose-unicity}
  If $\Cshape \Ca \t \sh$, then $\Cshape {\Cb\where\Ca} \t \sh$.
  \begin{proof}
    Induction on the structure of $\Cb$.
    \begin{proofcases}
      \proofcase{$\hole$}
        immediate.
      \proofcase{$\Cc \cand \c$}

	\begin{llproof}
	  \shapePf{\Ca}{\t}{\sh}{Premise}
	  \shapePf{\Cc\where\Ca}{\t}{\sh}{By \ih}
	  \ForallPf{\semenv, \gt}{}{Definition of $\Cshape {\parens{\Cc\where\Ca \cand \c}} \t \sh$}
	  \vdashPf{\semenv}{\cerase {\Cc\where\Ca\where{\cunif \t \gt}} \cand \cerase \c}{$\implies$I}
	  \vdashPf{\semenv}{\cerase {\Cc\where\Ca\where{\cunif \t \gt}}}{Simple inversion}
	  \eqPf{\shape \gt}{\sh}{$\implies$E on $\Cshape {\Cc\where\Ca} \t \sh$}
\Hand 	  \shapePf{\parens{\Cc\where\Ca \cand \c}}{\t}{\sh}{Above}
	\end{llproof}

      \proofcase{$\c \cand \Cc$}

	\begin{llproof}
	  Similar to the $\Cc \cand \c$ case.
	\end{llproof}

      \proofcase{$\cexists \tv \Cc$}

	\begin{llproof}
	  \shapePf{\Ca}{\t}{\sh}{Premise}
	  \shapePf{\Cc\where\Ca}{\t}{\sh}{By \ih}
	  \ForallPf{\semenv, \gt}{}{Definition of $\Cshape {\parens{\cexists \tv \Cc\where\Ca}} \t \sh$}
	  \vdashPf{\semenv}{\cexists \tv \cerase {\Cc\where\Ca\where{\cunif \t \gt}}}{$\implies$I}
	  \vdashPf{\semenv\where{\tv \is \gtp}}{\cerase {\Cc\where\Ca\where{\cunif \t \gt}}}{Simple inversion}
	  \eqPf{\shape \gt}{\sh}{$\implies$E on $\Cshape {\Cc\where\Ca} \t \sh$}
\Hand 	  \shapePf{\parens{\cexists \tv \Cc\where\Ca}}{\t}{\sh}{Above}
	\end{llproof}

      \proofcase{$\cfor \tv \Cc$}

	\begin{llproof}
	  Similar to $\cexists \tv \Cc$ case.
	\end{llproof}

      \proofcase{$\cexistsi \inst \x \Cc$}

	\begin{llproof}
	  Similar to $\cexists \tv \Cc$ case.
	\end{llproof}

      \proofcase{$\clet \x \tv \Cc \c$}

	\begin{llproof}
	  \shapePf{\Ca}{\t}{\sh}{Premise}
	  \shapePf{\Cc\where\Ca}{\t}{\sh}{By \ih}
	  \ForallPf{\semenv, \gt}{}{Definition of $\Cshape {\parens {\Let \x \ldots}} \t \sh$}
	  \vdashPf{\semenv}{\clet \x \tv {\cerase {\Cc\where\Ca\where{\cunif \t \gt}}} {\cerase \c}}{$\implies$I}
	  \vdashPf{\semenv}{\cexists \tv \cerase {\Cc\where\Ca\where{\cunif \t \gt}}}{Simple inversion}
	  \vdashPf{\semenv\where{\tv \is \gtp}}{\cerase {\Cc\where\Ca\where{\cunif \t \gt}}}{Simple inversion}
	  \eqPf{\shape \gt}{\sh}{$\implies$E on $\Cshape {\Cc\where\Ca} \t \sh$}
\Hand 	  \shapePf{\parens{\clet \x \tv {\Cc\where\Ca} \c}}{\t}{\sh}{Above}
	\end{llproof}

      \proofcase{$\clet \x \tv \c \Cc$}

	\begin{llproof}
	  Similar to $\clet \x \tv \Cc \c$ case.
	\end{llproof}

      \proofcase{$\cletr \x \tv \tvs \Cc \c$}

	\begin{llproof}
	  Similar to $\clet \x \tv \Cc \c$ case.
	\end{llproof}

      \proofcase{$\cletr \x \tv \tvs \c \Cc$}

	\begin{llproof}
	  Similar to $\clet \x \tv \c \Cc$ case.
	\end{llproof}
    \end{proofcases}
  \end{proof}
\end{lemma}

\begin{lemma}[Inversion of unicity]
  \label{lem:unicity-inversion}~
  \begin{enumerate}[(\roman*)]
    \item If $\Cshape {\parens{\cexists \tv \C}} \t \sh$, then $\Cshape \C \t \sh$.
    \item If $\Cshape {\parens{\cfor \tv \C}} \t \sh$, then $\Cshape \C \t \sh$.
  \end{enumerate}
  \begin{proof}
    The definition of $\Cshape \C \t \sh$ uses simple semantics on the
    erasure $\cerase \C$, so these results are easily shown by simple inversion.
  \end{proof}
\end{lemma}

\begin{lemma}[Decanonicalization]
  \label{lem:decanonicalization}
  If $\semenv \Th \c$, then $\semenv \th \c$.
  \begin{proof}
    Induction on the given derivation $\semenv \Th \c$
  \end{proof}
\end{lemma}

\begin{theorem}[Canonicalization]
  \label{thm:canonicalization}
  If $\semenv \th \c$, then $\semenv \Th \c$.
  \begin{proof}
  We proceed by induction on $\semenv \th \c$ with the measure $\cmeasure \c$.
  \begin{proofcases}
    \proofcasederivation
      {True}
      { }
      {\semenv \th \ctrue}

      \begin{llproof}
\Hand   \VdashPf{\semenv}{\ctrue}{immediate by \Rule{Can-Base}}
      \end{llproof}

    \proofcasederivation
      {Unif}
      {\semenv(\ta) = \semenv(\tb)}
      {\semenv \th \cunif \ta \tb}

      \begin{llproof}
	Similar to the \Rule{True} case.
      \end{llproof}
    \proofcasederivation
      {Conj}
      {\semenv \th \ca \\ \semenv \th \cb}
      {\semenv \th \ca \cand \cb}

      \begin{llproof}
	\vdashPf{\semenv}{\ca} {Premise}
	\vdashPf{\semenv}{\cb} {Premise}
	\VdashPf{\semenv}{\ca} {By \ih}
	\VdashPf{\semenv}{\cb} {By \ih}
	\decolumnizePf
	\casesPf{\semenv \Th \ca, \semenv \Th \cb}
      \end{llproof}

      \begin{proofcases}
	\proofcasederivationdouble
	  {Can-Base}
	  {\semenv \th \ca \\ \ca \simple}
	  {\semenv \Th \ca}
	  {Can-Base}
	  {\semenv \th \cb \\ \cb \simple}
	  {\semenv \Th \cb}

        \begin{llproof}
\Hand     \VdashPf{\semenv}{\ca \cand \cb}{immediate by \Rule{Can-Base}}
        \end{llproof}


	\proofcasederivationdouble
	  {Can-Match-Ctx}
	  {\Cshape \C \t \sh \\ \semenv \Th \C\where{\cmatched \t \sh \cbrs}}
	  {\semenv \Th \underbrace{\C\where{\cmatch \t \cbrs}}_\ca}
	  {}
	  {}
	  {\semenv \Th \cb}

	  \begin{llproof}

	    \VdashPf{\semenv}{\C\where{\cmatched \t \sh \cbrs}} {Premise}
	    \vdashPf{\semenv}{\C\where{\cmatched \t \sh \cbrs}} {\cref{lem:decanonicalization}}
	    \vdashPf{\semenv}{\C\where{\cmatched \t \sh \cbrs} \cand \cb}{By \Rule{Conj}}
 	    \VdashPf{\semenv}{\C\where{\cmatched \t \sh \cbrs} \cand \cb}{By \ih}
	    \Pf{}{}{\Cshape \C \tv \sh}{Premise}
	    \Pf{}{}{\Cshape {\parens {\C \cand \cb}} \tv \sh}{\cref{lem:compose-unicity}}
\Hand 	    \VdashPf{\semenv}{\C\where{\cmatch \t \cbrs}}{By \Rule{Can-Match-Ctx}}
	  \end{llproof}

	\proofcasederivationdouble
	  {}
	  {}
	  {\semenv \Th \ca}
	  {Can-Match-Ctx}
	  {\Cshape \C \t \sh \\ \semenv \Th \C\where{\cmatched \t \sh \cbrs}}
	  {\semenv \Th \underbrace{\C\where{\cmatch \t \cbrs}}_\cb}

	  \begin{llproof}
	    \Pf{}{}{}{Symmetric to the above case.}
	  \end{llproof}
      \end{proofcases}

      \proofcasederivation
	{Exists}
	{\semenv\where{\tv \is \gt} \th \c}
	{\semenv \th \cexists \tv \c}

	\begin{llproof}
	  \vdashPf{\semenv\where{\tv \is \gt}}{\c}{Premise}
	  \VdashPf{\semenv\where{\tv \is \gt}}{\c}{By \ih}
	  \casesPf{\semenv\where{\tv \is \gt} \Th \c}
	\end{llproof}

	\begin{proofcases}

	    \proofcasederivation
	      {Can-Base}
	      {\semenv\where{\tv \is \gt} \th \c \\ \c \simple}
	      {\semenv\where{\tv \is \gt} \Th \c}

	      \begin{llproof}
\Hand 		\VdashPf{\semenv}{\cexists \tv \c}{Immediate by \Rule{Can-Base}}
	      \end{llproof}


	      \proofcasederivation
		{Can-Match-Ctx}
		{\Cshape \C \t \sh \\ \semenv\where{\tv \is \gt} \Th \C\where{\cmatched \t \sh \cbrs}}
		{\semenv \Th \underbrace{\C\where{\cmatch \t \cbrs}}_\c}

		\begin{llproof}
		  \VdashPf{\semenv\where{\tv \is \gt}}{\C\where{\cmatched \t \sh \cbrs}}{Premise}
		  \vdashPf{\semenv\where{\tv \is \gt}}{\C\where{\cmatched \t \sh \cbrs}}{\cref{lem:decanonicalization}}
	    \decolumnizePf
		  \vdashPf{\semenv}{\cexists \tv \C\where{\cmatched \t \sh \cbrs}}{By \Rule{Exists}}
		  \VdashPf{\semenv}{\cexists \tv \C\where{\cmatched \t \sh \cbrs}}{By \ih}
		  \Pf{}{}{\Cshape \C \t \sh}{Premise}
		  \Pf{}{}{\Cshape {\parens {\cexists \tv \C}} \t \sh}{\cref{lem:compose-unicity}}
\Hand             \VdashPf{\semenv}{\cexists \tv \C\where{\cmatch \t \cbrs}}{By \Rule{Can-Match-Ctx}}
		\end{llproof}
	\end{proofcases}

	\proofcasederivation
	  {Forall}
	  {\forall \gt,~ \semenv\where{\tv \is \gt} \th \c}
	  {\semenv \th \cfor \tv \c}

	  \begin{llproof}
	    Similar to the \Rule{Exists} case.
	  \end{llproof}

	\proofcasederivation
	  {Let}
	  {\semenv \th \cexists \tv \ca \\ \semenv\where{\x \is \semenv(\cabs \tv \ca)} \th \cb}
	  {\semenv \th \clet \x \tv \ca \cb}

	  \begin{llproof}
	    \vdashPf{\semenv}{\cexists \tv \ca}{Premise}
	    \VdashPf{\semenv}{\cexists \tv \ca}{By \ih}
	    \vdashPf{\semenv\where{\x \is \semenv(\cabs \tv \ca)}}{\cb}{Premise}
	    \VdashPf{\semenv\where{\x \is \semenv(\cabs \tv \ca)}}{\cb}{By \ih}
	    \decolumnizePf
	    \casesPf{\semenv \Th \cexists \tv \ca, \semenv\where{\x \is \semenv(\cabs \tv \ca)} \Th \cb}
	  \end{llproof}

	  \begin{proofcases}
	    \proofcasederivationdouble
	      {Can-Base}
	      {\semenv \th \cexists \tv \ca \\ \cexists \tv \ca \simple}
	      {\semenv \Th \cexists \tv \ca}
	      {Can-Base}
	      {\semenv\where{\x \is \semenv(\cabs \tv \ca)} \th \cb \\ \cb \simple}
	      {\semenv\where{\x \is \semenv(\cabs \tv \ca)} \Th \cb}

	      \begin{llproof}
\Hand		\VdashPf{\semenv}{\clet \x \tv \ca \cb}{Immediate by \Rule{Can-Base}}
	      \end{llproof}

	    \proofcasederivationdouble
	      {Can-Match-Ctx}
	      {\Cshape {\parens {\cexists \tv \ca}} \t \sh \\ \semenv \Th \cexists \tv \C\where{\cmatched \t \sh \cbrs}}
	      {\semenv \Th \cexists \tv \underbrace{\C\where{\cmatch \t \cbrs}}_\ca}
	      {}
	      {}
	      {\semenv\where{\x \is \semenv(\cabs \tv \ca)} \Th \cb}

	      \begin{llproof}
		\shapePf{\parens {\cexists \tv \C}}{\t}{\sh}{Premise}
		\shapePf{\C}{\t}{\sh}{\cref{lem:unicity-inversion}}
		\VdashPf{\semenv}{\cexists \tv \C\where{\cmatched \t \sh \cbrs}}{Premise}
		\vdashPf{\semenv}{\cexists \tv \C\where{\cmatched \t \sh \cbrs}}{\cref{lem:decanonicalization}}
	    \decolumnizePf
		\eqPf{\semenv(\cabs \tv \ca)}
		  {\semenv(\cabs \tv \C\where{\cmatched \t \sh \cbrs})}
		  {\cref{corollary:matched-abstractions}}
		\vdashPf{\semenv}{\clet \x \tv {\C\where{\cmatched \t \sh \cbrs}} \cb}{By \Rule{Let}}
		\VdashPf{\semenv}{\clet \x \tv {\C\where{\cmatched \t \sh \cbrs}} \cb}{By \ih}
		\shapePf{\parens{\clet \x \tv \C \cb}}{\t}{\sh}{\cref{lem:compose-unicity}}
\Hand 		\VdashPf{\semenv}{\clet \x \tv {\C\where{\cmatch \t \cbrs}} \cb}{By \Rule{Can-Match-Ctx}}
	      \end{llproof}

	    \proofcasederivationdouble
		{}
		{}
		{\semenv \Th \cexists \tv \ca}
		{Can-Match-Ctx}
		{\Cshape \C \t \sh \\ \semenv\where{\x \is \semenv(\cabs \tv \ca)} \Th \C\where{\cmatched \t \sh \cbrs}}
		{\semenv\where{\x \is \semenv(\cabs \tv \ca)} \Th \underbrace{\C\where{\cmatch \t \cbrs}}_\cb}

		\begin{llproof}
		  \shapePf{\C}{\t}{\sh}{Premise}
		  \shapePf{\parens{\clet \x \tv \ca \C}}{\t}{\sh}{\cref{lem:compose-unicity}}
		  \VdashPf{\semenv\where{\x \is \semenv(\cabs \tv \ca)}}{\C\where{\cmatched \t \sh \cbrs}}{Premise}
		  \vdashPf{\semenv\where{\x \is \semenv(\cabs \tv \ca)}}{\C\where{\cmatched \t \sh \cbrs}}{\cref{lem:decanonicalization}}
		  \vdashPf{\semenv}{\clet \x \tv \ca {\C\where{\cmatched \t \sh \cbrs}}}{By \Rule{Let}}
		  \VdashPf{\semenv}{\clet \x \tv \ca {\C\where{\cmatched \t \sh \cbrs}}}{By \ih}
\Hand 		  \VdashPf{\semenv}{\clet \x \tv \ca {\C\where{\cmatch \t \sh}}}{By \Rule{Can-Match-Ctx}}
		\end{llproof}
	  \end{proofcases}

      \proofcasederivation
	{App}
	{\semenv(\t) \in \semenv(\x)}
	{\semenv \th \capp \x \t}

      \begin{llproof}
	Similar to the \Rule{True} case.
      \end{llproof}

      \proofcasederivation
	{LetR}
	{\semenv \th \cexists {\tv, \tvs} \ca \\
	 \semenv\where{\x \is \semenv(\cabsr \tv \tvs \ca)} \th \cb}
	{\semenv \th \cletr \x \tv \tvs \ca \cb}

      \begin{llproof}
	Similar to the \Rule{Let} case.
      \end{llproof}

      \proofcasederivation
	{AppR}
	{\greg \tv \semenvp \in \semenv(\x) \\
	 \semenv(\t) = \semenvp(\tv)}
	{\semenv \th \capp \x \t}

      \begin{llproof}
	Similar to the \Rule{App} case.
      \end{llproof}

    \proofcasederivation
      {Exists-Inst}
      {\greg \tv \semenvp \in \semenv(\x) \\ \semenv\where{\inst \is \semenvp} \th \c}
      {\semenv \th \cexistsi \inst \x \c}

      \begin{llproof}
	Similar to the \Rule{Exists} case.
      \end{llproof}


    \proofcasederivation
      {Multi-Unif}
      {\forall \t \in \ueq,~ \semenv(\t) = \gt}
      {\semenv \th \ueq}

      \begin{llproof}
	Similar to the \Rule{Unif} case.
      \end{llproof}

    \proofcasederivation
      {Incr-Inst}
      {\semenv(\inst)(\tv) = \semenv(\t)}
      {\semenv \th \cpinst \inst \tv \t}


      \begin{llproof}
	Similar to the \Rule{App} case.
      \end{llproof}


  \end{proofcases}
  \end{proof}
\end{theorem}

\begin{lemma}[Inversion of suspension]
  \label{lem:susp-inversion}
  If $\semenv \th \C\where{\cmatch \t \cbrs}$ and $\Cshape \C \t \sh$,
  then\\$\semenv \th \C\where{\cmatched \t \sh \cbrs}$.

  \begin{proof}
    We use canonicalization (\cref{thm:canonicalization}) to induct on $\semenv \Th
    \C\where{\cmatch \t \cbrs}$ instead of $\semenv \th \C\where{\cmatch \t
    \cbrs}$.

    This simplifies the proof, but introduces a circular dependency between
    \cref{thm:canonicalization} and \cref{lem:susp-inversion}.
    %
    However, this does not compromise the well-foundedness of induction, as the
    application of \cref{lem:susp-inversion} (via
    \cref{corollary:matched-abstractions}) within the proof of
    \cref{thm:canonicalization} is restricted to strictly smaller constraints.

    \begin{proofcases}
      \proofcasederivation
	{Can-Base}
	{\semenv \th \C\where{\cmatch \t \cbrs} \\ \C\where{\cmatch \t \cbrs} \simple}
	{\semenv \Th \C\where{\cmatch \t \cbrs}}

        The second premise is a contradiction.

      \proofcasederivation
	{Can-Match-Ctx}
	{\Cshape \Cp \tp \shp \\ \semenv \Th \Cp\where{\cmatched \tp \shp \cbrs'}}
	{\semenv \Th \underbrace{\Cp\where{\cmatch \tp \cbrs'}}_{\C\where{\cmatch {~\t~} {~\cbrs}}}}

	\begin{llproof}
	  \casesPf{\C = \Cp}
	\end{llproof}
	\begin{proofcases}
	  \proofcase{$\C = \Cp$}

	    \begin{llproof}
	      \eqPf{\C}{\Cp}{Premise}
	      \eqPf{\tp}{\t}{}
	      \eqPf{\shp}{\sh}{}
	      \eqPf{\cbrs'}{\cbrs}{}
\Hand         \VdashPf{\semenv}{\C\where{\cmatched \t \sh \cbrs}}{Premise}
	    \end{llproof}

	  \proofcase{$\C \neq \Cp$}

	    \newcommand{\Ctwo}{\C_2}
	    \begin{llproof}
	      \eqPf{\Ctwo\where{\cmatch \t \cbrs, \cmatch \tp \cbrs'}}{\C\where{\cmatch \t \cbrs}}{For some 2-hole context $\Ctwo$}
	      \continueeqPf{\Cp\where{\cmatch \tp \cbrs'}}{}
	      \decolumnizePf
	      \VdashPf{\semenv}{\Ctwo\where{\cmatch \t \cbrs, \cmatched \tp \shp \cbrs'}}{Premise}
	      \decolumnizePf
	      \ForallPf{\semenvp, \gtp}{}{\hspace{32.5ex}Defn. of $\Cshape {\Ctwo\where{\hole, \cmatched \tp \shp \cbrs'}} \t \sh$}
	      \decolumnizePf
	      \vdashPf{\semenvp}{\cerase {\Ctwo\where{\cunif \t \gtp, \cmatched \tp \shp \cbrs'}}}{$\implies$I}
	      \vdashPf{\semenvp}{\cerase {\Ctwo\where{\cunif \t \gtp, \ctrue}}}{\cref{lem:cong-simple}}
	    \decolumnizePf
	      \eqPf{\cerase {\Ctwo\where{\cunif \t \gtp, \ctrue}}}{\cerase {\Ctwo\where{\cunif \t \gtp, \cerase {\cmatch \tp \cbrs'}}}}{By definition}
	      \continueeqPf{\cerase {\C\where{\cunif \t \gtp}}}{By definition}
	      \vdashPf{\semenvp}{\cerase {\C\where{\cunif \t \gtp}}}{Above}
	      \eqPf{\shape \gtp}{\sh}{$\implies$E on $\Cshape \C \t \sh$}
	      \shapePf{\Ctwo\where{\hole, \cmatched \tp \shp \cbrs'}}{\t}{\sh}{Above}
	      \VdashPf{\semenv}{\Ctwo\where{\cmatched \t \sh \cbrs, \cmatched \tp \shp \cbrs'}}{By \ih}
	      \decolumnizePf
	      \ForallPf{\semenvp, \gtp}{}{\hspace{32.5ex}Defn. of $\Cshape {\Ctwo\where{\cmatched \t \sh \cbrs, \hole}} \tp \shp$}
	      \decolumnizePf
	      \vdashPf{\semenvp}{\cerase {\Ctwo\where{\cmatched \t \sh \cbrs, \cunif \tp \gtp}}}{$\implies$I}
	      \vdashPf{\semenvp}{\cerase {\Ctwo\where{\ctrue, \cunif \tp \gtp}}}{\cref{lem:cong-simple}}
	      \eqPf{\cerase {\Ctwo\where{\ctrue, \cunif \tp \gtp}}}{\cerase {\Ctwo\where{\cerase {\cmatch \t \cbrs}, \cunif \tp \gtp}}}{By definition}
	      \continueeqPf{\cerase {\Cp\where{\cunif \tp \gtp}}}{By definition}
	      \vdashPf{\semenvp}{\cerase {\C\where{\cunif \t \gtp}}}{Above}
	      \shapePf{\Cp}{\tp}{\shp}{Premise}
	      \eqPf{\shape \gtp}{\shp}{$\implies$E on $\Cshape \Cp \tp \shp$}
	      \shapePf{\Ctwo\where{\cmatched \t \sh \cbrs, \hole}}{\tp}{\shp}{Above}
\Hand 	      \VdashPf{\semenv}{\Ctwo\where{\cmatched \t \sh \cbrs, \cmatch \tp \cbrs'}}{By \Rule{Can-Match-Ctx}}
	    \end{llproof}
	\end{proofcases}
    \end{proofcases}
  \end{proof}
\end{lemma}

\begin{corollary}
  \label{corollary:matched-abstractions}
  If $\Cshape \C \t \sh$, then $\semenv(\cabs \tv \C\where{\cmatch \t \cbrs}) = \semenv(\cabs \tv \C\where{\cmatched \t \sh \cbrs})$.
  Similarly, $\semenv(\cabsr \tv \tvs \C\where{\cmatch \t \cbrs}) = \semenv(\cabsr \tv \tvs \C\where{\cmatched \t \sh \cbrs})$.
  \begin{proof}
    It is sufficient to show that $\semenv\where{\tv \is \gt} \th \C\where{\cmatch \t \cbrs}$ if and only if
    $\semenv \th \C\where{\cmatched \t \sh \cbrs}$.

    \begin{proofcases}
      \proofcase{$\implies$}

	\begin{llproof}
	  \shapePf{\C}{\t}{\sh}{Premise}
	  \vdashPf{\semenv\where{\tv \is \gt}}{\C\where{\cmatch \t \cbrs}}{Premise}
\Hand 	  \vdashPf{\semenv\where{\tv \is \gt}}{\C\where{\cmatched \t \sh \cbrs}}{\cref{lem:susp-inversion}}
	\end{llproof}
      \proofcase{$\impliedby$}

	\begin{llproof}
	  \shapePf{\C}{\t}{\sh}{Premise}
	  \vdashPf{\semenv\where{\tv \is \gt}}{\C\where{\cmatched \t \sh \cbrs}}{Premise}
\Hand 	  \vdashPf{\semenv\where{\tv \is \gt}}{\C\where{\cmatch \t \cbrs}}{By \Rule{Match-Ctx}}
	\end{llproof}
    \end{proofcases}

    For $\semenv(\cabsr \tv \tvs \C\where{\cmatch \t \cbrs}) = \semenv(\cabsr \tv \tvs \C\where{\cmatched \t \sh \cbrs})$, the proof is identical.
  \end{proof}
\end{corollary}

\section{Properties of the constraint solver}
\label{app:proofs-solving}

The primary requirement of our constraint solver is correctness:
a constraint $\c$ is satisfiable if and only if the solver terminates with a solution.

This section decomposes this requirement into three properties: preservation,
progress, and termination---and provides proofs for each. Correctness then
follows as a corollary of these results.

\subsection{Preservation}

This section details the proof of \emph{preservation} for the solver: if $\ca
\csolve \cb$, then $\ca \cequiv \cb$.
%
Since rewriting may occur under arbitrary contexts, it suffices to check for
each rule, that the equivalence $\ca \cequiv \cb$ holds under all contexts
$\C$.

However, the introduction of suspended match constraints breaks congruence of
equivalence. That is, it is no longer the case that $\ca \cequiv \cb$ implies
$\C\where\ca \cequiv \C\where\cb$.
%
For instance, we have $\cmatch \tv \cbrs \cequiv \cfalse$, yet
$\C\where{\cmatch \tv \cbrs} \cnequiv \C\where\cfalse$ for $\C \is \hole
\cand \cunif \tv \tint$.

As a result, we must prove \emph{contextual equivalence} for each rewriting
rule explicitly. This is both non-trivial and tedious. To simplify the task, we
first present a series of auxiliary lemmas that recover contextual equivalence
for many common cases.
%
Whenever possible, we prefer to work with equivalences on \emph{simple}
constraints, as these retain the desired congruence properties that do not hold
generally in our system.

\begin{definition}[Contextual eqiuvalence]
  Two constraints $\ca$ and $\cb$ are contextually equivalence, written $\ca \cequivctx \cb$,
  iff:
  \begin{mathpar}
    \ca \cequivctx \cb \uad\eqdef\uad \all \C \uad \C\where\ca \cequiv \C\where\cb
  \end{mathpar}
\end{definition}

\begin{corollary}[Simple equivalence is congruent]
  \label{corollary:cong-simple-equiv}
  Given simple constraints $\ca, \cb$ and simple context $\C$. If
  $\ca \cequiv \cb$, then $\C\where\ca \equiv \C\where\cb$.
  \begin{proof}
    Follows from \cref{lem:cong-simple}.
  \end{proof}
\end{corollary}

\begin{lemma}[Simple equivalence is contextual]
  \label{lem:ctxt-equiv-simple}
  For simple constraints $\ca, \cb$. If $\ca \cequiv \cb$, then $\ca \cequivctx \cb$.
  \begin{proof}
    We proceed by induction on the number of suspended match constraints $n$ in $\C$.

    \begin{proofcases}
      \proofcase{$n$ is 0}
	Follows from \cref{corollary:cong-simple-equiv}.

      \proofcase{$n$ is $k + 1$}

	\begin{proofcases}
	  \proofcase{$\implies$}

	  \newcommand{\Ctwo}{\Cb}
	  \begin{llproof}
	    \vdashPf{\semenv}{\C\where{\ca}}{Premise}
	    \VdashPf{\semenv}{\C\where{\ca}}{\cref{thm:canonicalization}}
	    \shapePf\Cp\t\sh{Inversion of \Rule{Can-Match-Ctx}}
	    \VdashPf{\semenv}{\Cp\where{\cmatched \t \sh \cbrs}}{\ditto}
	    \eqPf{\C\where{\ca}}{\Cp\where{\cmatched \t \sh \cbrs}}{\ditto}
	    \continueeqPf{\Ctwo\where{\cmatched \t \sh \cbrs, \ca}}{For some two-hole context $\Ctwo$}
	    \vdashPf{\semenv}{\Ctwo\where{\cmatched \t \sh \cbrs, \cb}}{By \ih}
	    \ForallPf{\semenvp, \gt}{}{Defn of $\Cshape \Cp \t \sh$}
	    \vdashPf{\semenvp}{\cerase{\Ctwo\where{\t \is \gt, \cb}}}{Premise}
	    \vdashPf{\semenvp}{\cerase{\Ctwo\where{\t \is \gt, \ca}}}{\cref{corollary:cong-simple-equiv}}
	    \vdashPf{\semenvp}{\cerase{\Cp\where{\t \is \gt}}}{Above}
	    \eqPf{\shape \gt}{\sh}{$\implies$E on $\Cshape \Cp \t \sh$}
	    \decolumnizePf
	    \shapePf{\Ctwo\where{\hole, \cb}}{\t}{\sh}{Above}
\Hand	    \vdashPf{\semenv}{\Ctwo\where{\cmatch \t \cbrs, \cb}}{By \Rule{Match-Ctx}}
	  \end{llproof}

	  \proofcase{$\impliedby$}

	  \begin{llproof}
	    Symmetric argument.
	  \end{llproof}
	\end{proofcases}
    \end{proofcases}
  \end{proof}
\end{lemma}

\begin{lemma}[Unification is simple]
  \label{lem:unif-problem-simple}
  For all unification problems $\up$, $\up \simple$.
  \begin{proof}
    By induction on the structure of $\up$.
  \end{proof}
\end{lemma}


\begin{definition}[Context equivalence]
  Two contexts $\Ca$ and $\Cb$ are equivalent with guard $P$, written $\Ca \cctxequiv^P \Cb$ iff:
  \begin{mathpar}
    \Ca \cctxequiv^P \Cb \uad\eqdef\uad \all \cs \uad P(\cs) \implies \Ca\where\cs \cequivctx \Cb\where\cs
  \end{mathpar}
\end{definition}

\begin{definition}[Match-closed]
  A predicate $P$ on constraints is \emph{match-closed} if, for all constraints $\cs, \cs'$, contexts $\C$, matches $\cmatch \t \cbrs$ and shapes $\sh$,
  \begin{mathpar}
    P(\cs, \C\where{\cmatch \t \cbrs}, \cs') \implies P(\cs, \C\where{\cmatched \t \sh \cbrs}, \cs')
  \end{mathpar}
\end{definition}

\begin{lemma}[Determines is match-closed]
  \label{lem:determines-is-match-closed}
  $\cdetermines \c \tvbs$ is match-closed. Similarly, $\th \cdetermines \c \tvbs$ is matched closed.
  \begin{proof}
    Follows from the definitions of $\cdetermines \c \tvbs$, $\th \cdetermines \c \tvbs$, and \cref{lem:cong-simple}.
  \end{proof}
\end{lemma}


\begin{lemma}[Simple context equivalence]
  \label{lem:simple-ctxt-equiv}
  For any two simple contexts $\Ca, \Cb$ and a match-closed guard $P$. If
  the two contexts $\Ca$ and $\Cb$ are equivalent under any simple constraints satisfying $P$,
  then $\Ca \cctxequiv^P \Cb$.

  \begin{proof}
    Let us assume that ($\dagger$) holds:
    \begin{mathpar}
      \all{\C, \cs \simple} P(\cs) \implies \C\where{\Ca\where\cs} \cequiv \C\where{\Cb\where\cs}
    \end{mathpar}

    We proceed by induction on the number of suspended match constraints $n$ with
    the statement $Q(n) \is  \all {\cs, \C} \cnmatches {\C} + \cnmatches \cs = n \implies P(\cs) \implies
    \C\where{\Ca\where\cs} \equiv \C\where{\Cb\where\cs}$.


    \begin{proofcases}
      \proofcase{$n$ is 0}

	\begin{llproof}
	  \simplePf{\C, \cs}{Premise ($n$ is 0)}
	  \Hand	  \equivPf{P(\cs) \implies \C\where\Ca\where\cs}{\C\where\Cb\where\cs}{$\dagger$}
	\end{llproof}

      \proofcase{$n$ is $k + 1$}

	\begin{proofcases}
	  \proofcase{$\implies$}

	  \begin{llproof}
	    \Pf{P(\cs)}{}{}{Premise}
	    \vdashPf{\semenv}{\C\where\Ca\where\cs}{Premise}
	    \VdashPf{\semenv}{\C\where\Ca\where\cs}{\cref{thm:canonicalization}}
	    \VdashPf{\semenv}{\Cp\where{\cmatched \t \sh \cbrs}}{Inversion of \Rule{Can-Match-Ctx}}
	    \shapePf{\Cp}{\t}{\sh}{\ditto}
	    \eqPf{\C\where\Ca\where\cs}{\Cp\where{\cmatch \t \cbrs}}{\ditto}
	    \commentPf{Cases on $\C, \cs$.}{}
	  \end{llproof}

	  \begin{proofcases}
	    \proofcase{$\C$ contains $\Cp$'s hole}

	      \newcommand{\Ctwo}{\Cc}
	      \begin{llproof}
		\eqPf{\C\where\Ca\where\cs}{\Ctwo\where{\cmatch \t \cbrs, \Ca\where\cs}}{For some 2-hole context $\Ctwo$}
		\VdashPf{\semenv}{\Ctwo\where{\cmatched \t \sh \cbrs, \Ca\where\cs}}{}
		\eqPf{k}{\cnmatches {\Ctwo\where{\cmatched \t \sh \cbrs, \Ca\where\cs}}}{}
		\vdashPf{\semenv}{\Ctwo\where{\cmatched \t \sh \cbrs, \Cb\where\cs}}{By \ih}
		\decolumnizePf
		\ForallPf{\semenvp, \gt}{}{}
		\vdashPf{\semenvp}{\cerase{\Ctwo\where{\cunif \t \gt, \Cb\where\cs}}}{Premise}
		\vdashPf{\semenvp}{\cerase{\Ctwo\where{\cunif \t \gt, \Ca\where\cs}}}{$\dagger$}
		\eqPf{\shape \gt}{\sh}{$\implies$E on $\Cshape \Cp \t \sh$}
		\shapePf{\Ctwo\where{\hole, \Cb\where\cs}}{\t}{\sh}{Above}
\Hand		\vdashPf{\semenv}{\Ctwo\where{\cmatch \t \cbrs, \Cb\where\cs}}{By \Rule{Match-Ctx}}
	      \end{llproof}

	    \proofcase{$\ci$ contains $\Cp$'s hole}

	      \begin{llproof}
		Similar argument to the above case, but relies on the match-closure of $P$.
	      \end{llproof}
	  \end{proofcases}


	  \proofcase{$\impliedby$}

	  \begin{llproof}
	    Symmetric argument.
	  \end{llproof}
	\end{proofcases}
    \end{proofcases}
  \end{proof}
\end{lemma}


\begin{lemma}[Simple let equivalence]
  \label{lem:simple-let-equiv}
  Given simple constraints $\ca, \cb$ and a simple context $\C$.
  Suppose that
    \begin{mathpar}
      \forall \semenv, \semenvp, \cs \simple. \uad
	\semenvp(\x) = \semenv(\cabsr \tv \tvs {\C\where\cs}) \implies
	  \semenvp \th \ca \iff \semenvp \th \cb
    \end{mathpar}
  Then, for any context $\Cp$ that does not re-bind $\x$, we have:
    \begin{mathpar}
      \cletr \x \tv \tvs {\C\where{\bar\hole}} {\Cp\where\ca}
	\cctxequiv^P \cletr \x \tv \tvs {\C\where{\bar\hole}} {\Cp\where\cb}
    \end{mathpar}
  for any match-closed guard $P$ on the holes.

  \begin{proof}
    Let us assume ($\dagger$):
    \begin{mathpar}
      \forall \semenv, \semenvp, \cs. \uad
	\semenvp(\x) = \semenv(\cabsr \tv \tvs {\C\where\cs}) \implies
	  \semenvp \th \ca \iff \semenvp \th \cb
    \end{mathpar}

    We proceed by induction on the number of suspended match constraints in
    $\Cpp, \Cp, \cs$ with the statement $P(n) \is \all {\Cpp, \Cp, \cs} \cnmatches {\Cpp, \Cp, \cs} = n \implies \Cpp\where{\cletr \x \tv \tvs {\C\where\cs} {\Cp\where{\ca}}}
    \cequiv \Cpp\where{\cletr \x \tv \tvs {\C\where\cs} {\Cp\where\cb}}$.

    \begin{proofcases}
      \proofcase{$n$ is 0}

	Thus $\Cpp, \Cp, \cs$ are simple. It suffices to show the equivalence on the \Let-constraint directly and use congruence
	of equivalence for simple constraints (\cref{lem:ctxt-equiv-simple}) to establish the result.

	We proceed by induction on the structure of $\Cp$ with the statement ($\ddagger$):
	\begin{mathpar}
	\forall \semenv, \semenvp. \uad
	  \semenvp(\x) = \semenv(\cabsr \tv \tvs {\C\where\cs}) \implies
	    \semenvp \th \Cp\where{\ca} \iff \semenvp \th \Cp\where{\cb}
	\end{mathpar}
	This holds due to the compositionality of simple equivalence using $\dagger$ as a base case.

	\begin{proofcases}
	  \proofcase{$\implies$}

	\begin{llproof}
	  \vdashPf{\semenv}{\cletr \x \tv \tvs {\C\where{\cs}} {\Cp\where{\ca}}}{Premise}
	  \vdashPf{\semenv}{\cexists {\tv, \tvs} \C\where{\cs}}{Simple inversion}
	  \vdashPf{\semenv\where{\x \is \semenv(\cabsr \tv \tvs \C\where{\cs})}}{\Cp\where{\ca}}{\ditto}
	  \vdashPf{\semenv\where{\x \is \semenv(\cabsr \tv \tvs \C\where{\cs})}}{\Cp\where{\cb}}{$\ddagger$}
	  \vdashPf{\semenv}{\cletr \x \tv \tvs {\C\where{\cs}} {\Cp\where{\cb}}}{By \Rule{LetR}}
	\end{llproof}
	  \proofcase{$\impliedby$}

	  \begin{llproof}
	    Symmetric argument.
	  \end{llproof}
	\end{proofcases}

      \proofcase{$n$ is $k + 1$}

      \begin{llproof}
	Analogous to the inductive step in \cref{lem:simple-ctxt-equiv}.
      \end{llproof}
    \end{proofcases}
  \end{proof}
\end{lemma}

\newcommand{\disjointPf}[3]{\Pf{#1}{\disjoint}{#2}{#3}}
\begin{lemma}
  \label{lem:unicity-soundness}
  If $\th \Cshape \C \t \sh$, then $\Cshape \C \t \sh$.

  \begin{proof}
    \begin{proofcases}
      \proofcasederivation
      {S-Uni-Type}
      {{\t \notin \TyVars}}
      {\th \Cshape \C \t {~\shape \t}}

      \begin{llproof}
        \notinPf{\t}{\TyVars}{Premise}
        \eqPf{\t}{\shapp[\shape \t] \tys}{For some $\tys$}
        \ForallPf{\semenv, \gt}{}{Defn. of $\Cshape \C \t {~\shape \t}$}
        \vdashPf{\semenv}{\cerase{\C\where{\cunif \t \gt}}}{Premise}
        \vdashPf{\semenva}{\cunif \t \gt}{Inversion of $\cerase \Ca$}
        \eqPf{\gt}{\semenva(\t)}{Simple inversion}
        \continueeqPf{\shapp[\shape \t] {\semenva(\tys)}}{\ditto}
\Hand   \eqPf{\shape \gt}{\shape \t}{Applying shape to both sides}
      \end{llproof}


      \proofcasederivation
        {S-Uni-Var}
        {\tv \disjoint \bvs \Cb}
        {\th \Cshape {\Ca\where{\cunif \tv {\cunif \t \ueq} \cand \Cb\where{-}}} \tv {~\shape \t}}

      \begin{llproof}
        \disjointPf{\tv}{\bvs \Cb}{Premise}
        \eqPf{\t}{\shapp[\shape \t] \tys}{For some $\tys$}
        \ForallPf{\semenv, \gt}{}{Defn. of $\Cshape \C \tv {~\shape \t}$}
        \vdashPf{\semenv}{\cerase{\Ca\where{\cunif \tv {\cunif {\shapp[\shape \t] \tys} \ueq} \cand \Cb\where{\cunif \tv \gt}}}}{Premise}
        \vdashPf{\semenva}{\cunif \tv {\cunif {\shapp[\shape \t] \tys} \ueq}}{Inversion of $\cerase \Ca$}
        \vdashPf{\semenvb}{\cunif \tv \gt}{Inversion of $\cerase \Cb$}
        \eqPf{\gt}{\semenvb(\tv)}{Simple inversion}
        \continueeqPf{\semenva(\tv)}{$\tv \disjoint \bvs \Cb$}
        \continueeqPf{\shapp[\shape \t] {\semenva(\tys)}}{Simple inversion}
  \Hand \eqPf{\shape \gt}{\shape \t}{Applying shape to both sides}
      \end{llproof}


      \proofcasederivation
        {S-Uni-BackProp}
        {\th \Cshape{\cletr \x \tv \tvs {\Ca\where{\ctrue}} {\Cb\where{\cpapp \x \tvp \tvc \inst \cand -}}} \tvc \sh \\
         \tvp \in \tv, \tvs \\
         \x \disjoint \bvs \Cb \\
         \tvp \disjoint \bvs \Ca}
        {\th \Cshape{\cletr \x \tv \tvs {\Ca\where{-}} {\Cb\where{\cpapp \x \tvp \tvc \inst}}} \tvp \sh}

        \begin{llproof}
          \inPf{\tvp}{\tv,\tvs}{Premise}
          \disjointPf{\x}{\bvs \Cb}{\ditto}
          \disjointPf{\tvp}{\bvs \Ca}{\ditto}
          \vdashPf{}{\Cshape{\cletr \x \tv \tvs {\Ca\where{\ctrue}} {\Cb\where{\cpapp \x \tvp \tvc \inst \cand -}}} \tvc \sh}{\ditto}
          \shapePf{\cletr \x \tv \tvs {\Ca\where{\ctrue}} {\Cb\where{\cpapp \x \tvp \tvc \inst \cand -}}}{\tvc}{\sh}{By \ih}
          \ForallPf{\semenv, \gt}{}{Defn. of $\Cshape \ldots \tv {~\shape \t}$}
          \vdashPf{\semenv}{\cerase{\cletr \x \tv \tvs {\Ca\where{\cunif \tvp \gt}} {\Cb\where{\cpapp \x \tvp \tvc \inst}}}}{Premise}
          \LetPf{\semenva}{\semenv[\x \is \semenv(\cabsr \tv \tvs {\cerase {\Ca\where{\cunif \tvp \gt}}})]}{}
          \eqPf{\semenvp(\tvp)}{\gt}{For any $\greg \tv \semenvp \in \semenva(\x)$}
          \vdashPf{\semenvb}{\cpapp \x \tvp \tvc \inst}{Inversion of $\cerase \Cb$}
          \eqPf{\semenvb(\inst^\x)(\tvp)}{\semenvb(\tvc)}{Simple inversion}
          \inPf{\semenvb(\inst^\x)}{\semenvb(\x)}{Since $\cexistsi \inst \x \in \Cb$, $\semenvb$ extends $\semenva$}
          \eqPf{\semenvb(\inst^\x)(\tvp)}{\gt}{Above}
          \continueeqPf{\semenvb(\tvc)}{\ditto}
          \vdashPf{\semenva}{\cerase {\Cb\where{\cpapp \x \tvp \tvc \inst \cand \cunif \tvc \gt}}}{Entailment for $\cerase \Cb$}
          \vdashPf{\semenv}{\cerase {\cletr \x \tv \tvs {\Ca\where{\cunif \tvp \gt}} {\Cb\where{\cpapp \x \tvp \tvc \inst \cand \cunif \tvc \gt}}}}{By \Rule{LetR}}
          \vdashPf{\semenv}{\cletr \x \tv \tvs {\Ca\where{\ctrue}} {\Cb\where{\cpapp \x \tvp \tvc \inst \cand \cunif \tvc \gt}}}{Simple congruence}
\Hand     \eqPf{\shape \gt}{\sh}{$\implies E$ on $\Cshape \ldots \tvc \sh$}
        \end{llproof}

    \end{proofcases}
  \end{proof}
\end{lemma}

\begin{lemma}
  \label{lem:unicity-completeness}

  If $\C$ is normalized, then $\Cshape \C \t \sh$ if and only $\th \Cshape \C \t \sh$.
  \begin{proof}~
    \begin{proofcases}

      \proofcase{$\implies$}
        \newcommand{\NC}{R}

        Let us assume $\Cshape \C \t \sh$ and $\C$ is normalized.

        Given $\C$ is normalized, every constraint in $\C$ is of the form: \\

        \begin{bnfgrammar}
          \entry[]{\NC}{
            \bar{\hat\ueq}
            \cand \overline{\cmatch \tv \cbrs}
            \cand \cexists {\overline{\inst^\x}} {\overline{\cpapp \x \tvb \tvc \inst}}
            \cand \overline{\cletr \x \tvd \tvds {\NC_1} {\NC_2}}
          }
        \end{bnfgrammar}
        \\

        By assumptions, we have $\all {\phi, \gt} {\phi \th \cerase
        \C\where{\cunif \tv \gt}} \implies \shape \gt = \sh$. Hence $\cerase
        \C$ contains $\cerase \NC$ where: \\

        \begin{bnfgrammar}
          \entry[]{\cerase \NC}{
            \bar{\hat\ueq}
            \cand \cexists {\overline{\inst^\x}} {\overline{\cpapp \x \tvb \tvc \inst}}
            \cand \overline{\cletr \x \tvd \tvds {\cerase {\NC_1}} {\cerase {\NC_2}}}
          }
        \end{bnfgrammar}
        \\

        \Wlog all constraints that may determine the shape of $\tv$ are located
        with the regional binder (following the \Rule{S-Exists-Lower} and
        \Rule{S-Let-ConjLeft} rules). There are two cases:

        \begin{proofcases}

          \proofcase{$\cunif \tv {\cunif \t \ueq} \in \bar{\hat\ueq}$} Apply \Rule{S-Uni-Var}.


          \proofcase{Otherwise}

            Since $\C$ is normalized, it must be that case that no equality
            constraint determines the shape of $\tv$. Since any such equality
            would normalize to $\cunif \tv {\cunif \t \ueq}$, contradicting our
            assumption that $\C$ is normalized.

            By elimination on the structure of $\NC$, the only constraints that
            could determine the shape of $\tv$ are incremental instantiation
            constraints that copy $\tv$. So there exists a partial
            instantiation constraint $\cpapp \x \tv \tvc \inst$ such that $\Cp
            \where{\cpapp \x \tv \tvc \inst} = \C\where{\ctrue}$ and $\Cshape
            \Cp \tvc \sh$.

            By induction, we have $\th \Cshape \Cp \tvc \sh$. From
            \Rule{S-Uni-BackProp}, we have $\th \Cshape \C \tv \sh$.


        \end{proofcases}

      \proofcase{$\impliedby$} Follows from \cref{lem:unicity-soundness}.

    \end{proofcases}

  \end{proof}
\end{lemma}

\begin{lemma}[Unification preservation]
  \label{lem:unification-preservation}
  If $\upa \unif \upb$, then $\upa \equiv \upb$
  \begin{proof}
    By induction on the given derivation $\upa \unif \upb$.
    See \citet*{Pottier-Remy/emlti} for more details.
  \end{proof}
\end{lemma}

\preservationBIS*
\newcommand{\unifPf}[3]{\Pf{#1}{\unif}{#2}{#3}}
\newcommand{\equivctxPf}[3]{\Pf{#1}{\cequivctx}{#2}{#3}}
\newcommand{\ctxequivPf}[3]{\Pf{#1}{\cctxequiv}{#2}{#3}}
\begin{proof}
  We proceed by induction on the given derivation.
  It suffices to show that for each individual rule $R$ ($\ca \csolve_R \cb$),
  that $\ca \cequivctx \cb$.

  \begin{proofcases}
    \proofcaserewrite
      {S-Unif}
      {\upa \\ \upa \unif \upb}
      {\upb}

	\begin{llproof}
	  \unifPf{\upa}{\upb}{Premise}
	  \equivPf{\upa}{\upb}{\cref{lem:unification-preservation}}
	  \simplePf{\upa, \upb}{\cref{lem:unif-problem-simple}}
\Hand 	  \equivctxPf{\upa}{\upb}{\cref{lem:ctxt-equiv-simple}}
	\end{llproof}

    \proofcaserewrite
      {S-Exists-Conj}
      {\parens {\cexists \tv \ca} \cand \cb \\ \tv \disjoint \cb}
      {\cexists \tv \ca \cand \cb}

	\begin{llproof}
	  \disjointPf{\tv}{\cb}{Premise}
	  \decolumnizePf
	  \sufficientPf{equivalence for simple constraints}{}{\cref{lem:simple-ctxt-equiv}}
	  \supposePf{\ca, \cb \simple} {Premise}
	\end{llproof}
	\begin{proofcases}
	  \proofcase{$\implies$}

	  \begin{llproof}
	    \ForallPf{\semenv}{}{}
	    \vdashPf{\semenv}{\parens {\cexists \tv \ca} \cand \cb}{Premise}
	    \vdashPf{\semenv\where{\tv \is \gt}}{\ca}{Simple inversion}
	    \vdashPf{\semenv}{\cb}{Simple inversion}
	    \vdashPf{\semenv\where{\tv \is \gt}}{\cb}{$\tv \disjoint \cb$}
	    \vdashPf{\semenv\where{\tv \is \gt}}{\ca \cand \cb}{By \Rule{Conj}}
\Hand       \vdashPf{\semenv}{\cexists \tv \ca \cand \cb}{By \Rule{Exists}}
	  \end{llproof}
	  \proofcase{$\impliedby$}

	    \begin{llproof}
	    Symmetric argument.
	    \end{llproof}
	\end{proofcases}


      \proofcase{\Rule{S-Let}, \Rule{S-True}, \Rule{S-False},
      \Rule{S-Let-ExistsLeft},
      \XDR[No longer exists]{Rule{S-Let-Exists-InstLeft},}
      \Rule{S-Let-ExistsRight},
      \XDR[No longer exists]{Rule{S-Let-Exists-InstRight},}
      \Rule{S-Let-ConjLeft},
      \Rule{S-Let-ConjRight}, \Rule{S-Inst-Name}, \Rule{S-Exists-Exists-Inst},
      \Rule{S-Exists-Inst-Conj}, \Rule{S-Exists-Inst-Let}, \Rule{S-Exists-Inst-Solve},
      \Rule{S-All-Conj}}

      \begin{llproof}
	Similar argument to the \Rule{S-Exists-Conj} case.
      \end{llproof}

    \proofcaserewrite
      {S-Match-Ctx}
      {\C\where{\cmatch \t \cbrs} \\ \th \Cshape \C \t \sh}
      {\C\where{\cmatched \t \sh \cbrs}}

	\begin{llproof}
    \vdashPf{}{\Cshape \C \t \sh}{Premise}
	  \shapePf{\C}{\t}{\sh}{\cref{lem:unicity-soundness}}
	  \sufficientPf{equivalences between constraints}{}{\cref{lem:compose-unicity}}
	\end{llproof}

	\begin{proofcases}
	  \proofcase{$\implies$}

	  \begin{llproof}
	    \ForallPf{\semenv}{}{}
	    \vdashPf{\semenv}{\C\where{\cmatch \tv \cbrs}}{Premise}
\Hand 	    \vdashPf{\semenv}{\C\where{\cmatched \tv {\shape \t} \cbrs}}{\cref{lem:susp-inversion}}
	  \end{llproof}

	  \proofcase{$\impliedby$}

	  \begin{llproof}
	    \ForallPf{\semenv}{}{}
	    \vdashPf{\semenv}{\C\where{\cmatched \tv {\shape \t} \cbrs}}{Premise}
\Hand 	    \vdashPf{\semenv}{\C\where{\cmatch \tv \cbrs}}{By \Rule{Match-Ctx}}
	  \end{llproof}
	\end{proofcases}


    \proofcaserewrite
      {S-Let-AppR}
      {\cletr \x \tv \tvs \ca {\C\where{\capp \x \t}} \\ \tvc \disjoint \t \\ \x \disjoint \bvs \C}
      {\cletr \x \tv \tvs \ca {\C\where{\cexistsi {\tvc, \inst} \x \cunif \tvc \t \cand \cpinst \inst \tv \tvc }}}


	\begin{llproof}
	  \disjointPf{\tvc}{\t}{Premise}
	  \disjointPf{\x}{\bvs \C}{Premise}
	  \decolumnizePf
	  \sufficientPf{equivalence between}{\capp \x \t \text{ and } \cexistsi {\tvc, \inst} \x \cunif \tvc \t \cand \cpinst \inst \tv \tvc}{\cref{lem:simple-let-equiv}}
	  \supposePf{\semenvp(\x) = \semenv(\cabsr \tv \tvs \ca)}{Premise}
	\end{llproof}
	\begin{proofcases}
	  \proofcase{$\implies$}

	    \begin{llproof}
	      \vdashPf{\semenvp}{\capp \x \t}{Premise}
	      \inPf{\greg \tv {\semenva}}{\semenv(\x)}{Simple inversion}
	      \eqPf{\semenva(\tv)}{\semenvp(\t)}{\ditto}
	      \vdashPf{\semenvp\where{\tvc \is \semenvp(\t), \inst \is \semenva}}{\cpinst \inst \tv \tvc}{By \Rule{Incr-Inst}}
	      \vdashPf{\semenvp\where{\tvc \is \semenvp(\t), \inst \is \semenva}}{\cunif \tvc \t}{By \Rule{Unif}}
\Hand	      \vdashPf{\semenvp}{\cexistsi {\tvc, \inst} \x {\cunif \tvc \t \cand \cpinst \inst \tv \tvc}}{By \Rule{Exists}, \Rule{Exists-Inst} and \Rule{Conj}}
	    \end{llproof}

	  \proofcase{$\impliedby$}

	    \begin{llproof}
	      Symmetric argument.
	    \end{llproof}
	\end{proofcases}

    \proofcaserewrite
      {S-Inst-Copy}
      {\cletr \x \tv \tvs {\c}
	\C\where{\cpapp \x \tvp \tvc \inst}\\
	\c = \cp \cand \cunif \tvp {\cunif {\shapp \tvbs} \ueq}\\
	\tvp \in \reg \tv \tvs \\
	\neg \cyclic {\c} \\
	\tvbs' \disjoint \tvp, \tvc, \tvbs \\
       \x \disjoint \bvs \C}
      {\cletr \x \tv \tvs {\c}
	\C\where{\cexists {\tvbs'} \cunif \tvc {\shapp \tvbs'} \cand \cpapp \x {\tvbs} {\tvbs'} \inst}}


	\begin{llproof}
	  \disjointPf{\x}{\bvs \C}{Premise}
	  \disjointPf{\tvbs'}{\tvp, \tvc, \tvbs}{Premise}
	  \decolumnizePf
	  \sufficientPf{equivalence between}{\cpapp \x \tvp \tvc \inst \text{ and } \cexists {\tvbs'} \cunif \tvc {\shapp \tvbs'} \cand \cpapp \x {\tvbs} {\tvbs'} \inst}{\cref{lem:simple-let-equiv}}
	  \supposePf{\semenvp(\x) = \semenv(\cabsr \tv {\tvs} \c)}{Premise}
	\end{llproof}
	\begin{proofcases}
	  \proofcase{$\implies$}

	    \begin{llproof}
	      \vdashPf{\semenvp} {\cpapp \x \tvp \tvc \inst}{Premise}
	      \inPf{\greg \tv \semenva}{\semenv(\x)}{$\cexistsi \inst \x \in \C$}
	      \eqPf{\semenvp(\inst)}{\semenva}{\ditto}
	      \eqPf{\semenvp(\tvc)}{\semenv(\inst)(\tvp)}{Simple inversion}
	      \continueeqPf{\semenva(\tvp)}{Above}
	      \vdashPf{\semenva}{\cp \cand \cunif \tvp {\cunif {\shapp \tvbs} \ueq}}{Above}
	      \vdashPf{\semenva}{\cunif \tvp {\cunif {\shapp \tvbs} \ueq}}{Simple inversion}
	      \eqPf{\semenva(\tvp)}{\shapp {\semenva(\tvbs)}}{\ditto}
	      \eqPf{\semenvp(\tvc)}{\shapp {\semenva(\tvbs)}}{Above}
	      \vdashPf{\semenvp\where{\tvbs' \is \semenva(\tvbs)}}{\cunif \tvc {\shapp {\tvbs'}}}{By \Rule{Unif}}
	      \vdashPf{\semenvp\where{\tvbs' \is \semenva(\tvbs)}}{\cpapp \x \tvbs {\tvbs'} \inst}{By \Rule{Incr-Inst}}
\Hand 	      \vdashPf{\semenvp}{\cexists {\tvbs'} \cunif \tvc {\shapp \tvbs'} \cand \cpapp \x {\tvbs} {\tvbs'} \inst}{By \Rule{Exists} and \Rule{Conj}}
	    \end{llproof}

	  \proofcase{$\impliedby$}

	    \begin{llproof}
	      Symmetric argument.
	    \end{llproof}
	\end{proofcases}

    \proofcaserewrite
      {S-Inst-Unif}
      {\cpinst \inst \tv \tvca \cand \cpinst \inst \tv \tvcb}
      {\cpinst \inst \tv \tvca \cand \cunif \tvca \tvcb}

      \begin{llproof}

	\sufficientPf{equivalence between}{{\cpinst \inst \tv \tvca \cand
	\cpinst \inst \tv \tvcb} \text{ and } {\cpinst \inst \tv \tvca \cand
	\cunif \tvca \tvcb} }{\cref{lem:simple-ctxt-equiv}}

      \end{llproof}

      \begin{proofcases}
	\proofcase{$\implies$}

	  \begin{llproof}
	    \vdashPf{\semenv}{\cpinst \inst \tv \tvca \cand \cpinst \inst \tv \tvcb}{Premise}
	    \vdashPf{\semenv}{\cpinst \inst \tv \tvca}{Simple inversion}
	    \vdashPf{\semenv}{\cpinst \inst \tv \tvcb}{\ditto}
	    \eqPf{\semenv(\tvca)}{\semenv(\inst)(\tv)}{\ditto}
	    \eqPf{\semenv(\tvcb)}{\semenv(\inst)(\tv)}{\ditto}
	    \eqPf{\semenv(\tvca)}{\semenv(\tvcb)}{Above}
	    \vdashPf{\semenv}{\cunif \tvca \tvcb}{By \Rule{Unif}}
\Hand	    \vdashPf{\semenv}{\cpinst \inst \tv \tvca \cand \cunif \tvca \tvcb}{By \Rule{Conj}}
	  \end{llproof}

	\proofcase{$\impliedby$}

	  \begin{llproof}
	    Symmetric argument.
	  \end{llproof}
      \end{proofcases}

    \proofcaserewrite
      {S-Inst-Poly}
      {\cletr \x \tv {\tvs} {\ueqs \cand \c} {\C\where{\cpapp \x \tvp \tvc \inst}} \\
       \cfor \tvp \cexists {\tv, \tvs} {\ueqs} \cequiv \ctrue \\
       \tvp \in \reg \tv \tvs \\
       \tvp \disjoint \c \\
       \inst.\tvp \disjoint \insts \C \\
       \x \disjoint \bvs \C}
      {\cletr \x \tv {\tvs} {\ueqs \cand \c} {\C\where\ctrue}}


	\begin{llproof}
	  \equivPf{\cfor \tvp \cexists {\tv, \tvs} \ueqs}{\ctrue}{Premise}
	  \disjointPf{\tvp}{\c}{Premise}
	  \disjointPf{\inst.\tvp}{\insts \C}{Premise}
	  \disjointPf{\x}{\bvs \C}{Premise}
	  \decolumnizePf
	  \sufficientPf{equivalence between}
	    {\cpapp \x \tvp \tvc \inst \text{ and } \ctrue}
	    {\cref{lem:simple-let-equiv}}
	  \supposePf{\semenvp(\x) = \semenv(\cabsr \tv {\tvs, \tvp} \ueqs \cand \c)}{Premise}
	\end{llproof}
	\begin{proofcases}
	  \proofcase{$\implies$}

	    \begin{llproof}
	      \vdashPf{\semenvp}{\cpapp \x \tvp \tvc \inst}{Premise}
\Hand	      \vdashPf{\semenvp}{\ctrue}{By \Rule{True}}
	    \end{llproof}

	  \proofcase{$\impliedby$}

	    \begin{llproof}
	      \vdashPf{\semenvp}{\ctrue}{Premise}
	      \inPf{\greg \tv {\semenva}}{\semenvp(\x)}{$\C = \Ca\where{\cexistsi \inst \x \Cb}$}
	      \eqPf{\semenvp(\inst)}{\semenva}{\ditto}
	      \casesPf{\semenva(\tvp)}
	    \end{llproof}
	    \begin{proofcases}
	      \proofcase{$\semenva(\tvp) = \semenvp(\tvc)$}

		\begin{llproof}
		  \eqPf{\semenva(\tvp)}{\semenvp(\tvc)}{Premise}
\Hand		  \vdashPf{\semenvp}{\cpapp \x \tvp \tvc \inst}{By \Rule{Incr-Inst}}
		\end{llproof}


	      \proofcase{$\semenva(\tvp) \neq \semenvp(\tvc)$}

		\begin{llproof}
		  \LetPf{\semenvb}{\semenva\where{\tvp \is \semenvp(\tvc)}}{}
		  \vdashPf{\semenva}{\ueqs \cand \c}{By definition}
		  \vdashPf{\semenva}{\ueqs}{Simple inversion}
		  \vdashPf{\semenvb}{\ueqs}{$\tvp$ is polymorphic}
		  \vdashPf{\semenvb}{\c}{$\tvp \disjoint \c$}
		  \vdashPf{\semenvb}{\ueqs \cand \c}{By \Rule{Conj}}
		  \inPf{\greg \tv \semenvb}{\semenv(\x)}{By definition}
		  \supposePf{\semenvc \th \Cb\where\ctrue}{Considering entailment on $\cexistsi \inst \x$}
		  \eqPf{\semenvc(\inst)}{\semenva}{\ditto}
		  \vdashPf{\semenvc\where{\inst \is \semenvb}}{\Cb\where\ctrue}{$\inst.\tvp \disjoint \insts \Cb$}
		  \vdashPf{\deriv :: \semenvc}{\Cb\where\ctrue}{By \Rule{Exists-Inst}}
		  \commentPf{$\deriv$ is a derivation that satisfies $\semenva(\tvp) = \semenvp(\tvc)$.}{}
\Hand		  \commentPf{So this case degenerates to the former case.}{}
		\end{llproof}
	    \end{proofcases}
	\end{proofcases}

    \proofcaserewrite
      {S-Inst-Mono}
      {\cletr \x \tv \tvs \c {\C\where{\cpapp \x \tvb \tvc \inst}} \\
       \tvb \notin \reg \tv \tvs \\
       \x, \tvb \disjoint \bvs \C}
      {\cletr \x \tv \tvs \c {\C\where{\cunif \tvb \tvc}}}

	\begin{llproof}
	  \disjointPf{\tvb}{\tv, \tvs}{Premise}
	  \disjointPf{\x, \tvb}{\bvs \C}{Premise}
	  \decolumnizePf
	  \sufficientPf{equivalence between}
	    {\cpapp \x \tvb \tvc \inst \text{ and } \cunif \tvb \tvc}
	    {\cref{lem:simple-let-equiv}}
	  \supposePf{\semenvp(\x) = \semenv(\cabsr \tv {\tvs} \c)}{Premise}
	\end{llproof}

	\begin{proofcases}
	  \proofcase{$\implies$}

	  \begin{llproof}
	    \vdashPf{\semenvp}{\cpapp \x \tvb \tvc \inst}{Premise}
	    \inPf{\greg \tv \semenva}{\semenv(\c)}{$\cexistsi \inst \x \in \C$}
	    \eqPf{\semenvp(\inst)}{\semenva}{\ditto}
	    \eqPf{\semenvp(\tvc)}{\semenva(\tvb)}{Simple inversion}
	    \eqPf{\semenva(\tvb)}{\semenv(\tvb)}{$\tvb \disjoint \tv, \tvs$}
	    \eqPf{\semenvp(\tvb)}{\semenv(\tvb)}{$\tvb \disjoint \bvs \C$}
	    \eqPf{\semenvp(\tvc)}{\semenvp(\tvb)}{Above}
	    \vdashPf{\semenvp}{\cunif \tvc \tvb}{By \Rule{Unif}}
	  \end{llproof}

	  \proofcase{$\impliedby$}

	  \begin{llproof}
	    Symmetric argument.
	  \end{llproof}
	\end{proofcases}


    \proofcaserewrite
      {S-Let-Solve}
      {\cletr \x \tv \tvs \ueqs \c \\ \x \disjoint \c \\
       \cexists {\tv, \tvs} \ueqs \cequiv \ctrue}
      {\c}
	\begin{llproof}
	  \disjointPf{\x}{\c}{Premise}
	  \equivPf{\cexists {\tv, \tvs} \ueqs}{\ctrue}{}
	  \decolumnizePf
	  \sufficientPf{equivalence for simple constraints}{}{\cref{lem:simple-ctxt-equiv}}
	  \supposePf{\c \simple} {Premise}
	\end{llproof}
	\begin{proofcases}
	  \proofcase{$\implies$}

	  \begin{llproof}
	    \ForallPf{\semenv}{}{}
	    \vdashPf{\semenv}{\cletr \x \tv \tvs \ueqs \c}{Premise}
	    \vdashPf{\semenv}{\cexists {\tv, \tvs} \ueqs}{Simple inversion}
	    \vdashPf{\semenv\where{\x \is \semenv(\cabsr \tv \tvs \ueqs)}}{\c}{\ditto}
\Hand	    \vdashPf{\semenv}{\c}{$\x \disjoint \c$}
	  \end{llproof}
	  \proofcase{$\impliedby$}

	    \begin{llproof}
	    \ForallPf{\semenv}{}{}
	    \vdashPf{\semenv}{\c}{Premise}
	    \vdashPf{\semenv\where{\x \is \semenv(\cabsr \tv \tvs \ueqs)}}{\c}{$\x \disjoint \c$}
	    \vdashPf{\semenv}{\cexists {\tv, \tvs} \ueqs}{}
\Hand	    \vdashPf{\semenv}{\cletr \x \tv \tvs \ueqs \c}{By \Rule{LetR}}
	    \end{llproof}
	\end{proofcases}

  \proofcaserewrite{S-Exists-Lower}
    {\cletr \x \tv {\tvas, \tvbs} \ca \cb \\
     \cdetermines {\cexists {\tv, \tvas} \ca} \tvbs \\
     }
    {\cexists \tvbs \cletr \x \tv \tvas \ca \cb}

    \begin{llproof}
      \Pf{}{}{\cdetermines {\cexists {\tv, \tvas} \ca} \tvbs}{Premise}
      \sufficientPf{equivalence for simple constraints}{}{\cref{lem:simple-ctxt-equiv} and \cref{lem:determines-is-match-closed}}
      \supposePf{\ca, \cb \simple}{Premise}
    \end{llproof}
    \begin{proofcases}
      \proofcase{$\implies$}

      \begin{llproof}
	\vdashPf{\semenv}{\cletr \x \tv {\tvas, \tvbs} \ca \cb}{Premise}
	\vdashPf{\semenv}{\cexists {\tv, \tvas, \tvbs} \ca}{Simple inversion}
	\vdashPf{\semenv\where{\x \is \semenv(\cabsr \tv {\tvs, \tvbs} \ca)}}{\cb}{\ditto}
	\vdashPf{\semenv\where{\tv \is \gt, \tvas \is \gts, \tvbs \is \bar\gtp}}{\ca}{\ditto}
	\vdashPf{\semenv\where{\tvbs \is \bar\gtp}}{\cexists {\tv, \tvas} \ca}{By \Rule{Exists}}
	\sufficientPf{}{\semenv\where{\x \is \semenv(\cabsr \tv {\tvs, \tvbs} \ca)} = \semenv\where{\tvbs \is \bar\gt'}(\cabsr \tv \tvs \ca)}{}
      \end{llproof}
      \begin{proofcases}

      \proofcase{$\implies$}

	\begin{llproof}
	  \vdashPf{\semenv\where{\tv \is \gta, \tvs \is \bar\gta, \tvbs \is \bar\gtb}}{\ca}{Premise}
	  \vdashPf{\semenv\where{\tvbs \is \bar\gtb}}{\cexists {\tv, \tvs} \ca}{By \Rule{Exists}}
	  \eqPf{\bar\gtb}{\bar\gtp}{By definition of determines}
\Hand	  \vdashPf{\semenv\where{\tvbs \is \bar\gtp, \tv \is \gta, \tvs \is \bar\gta}}{\ca}{Above}

	\end{llproof}

      \proofcase{$\impliedby$}

      \begin{llproof}
	Symmetric argument.
      \end{llproof}


      \end{proofcases}



      \proofcase{$\impliedby$}

      \begin{llproof}
	Symmetric argument.
      \end{llproof}

    \end{proofcases}

  \proofcase{\Rule{S-Compress}, \Rule{S-Gc}, \Rule{S-Exists-All}, \Rule{S-All-Escape}, \Rule{S-All-Rigid}, \Rule{S-All-Solve}}

  \begin{llproof}
    Similar argument. Use \cref{lem:simple-ctxt-equiv}.
    The simple equivalences are standard, see \citet*{Pottier-Remy/emlti}.
  \end{llproof}
  \end{proofcases}
\end{proof}

\subsection{Progress}

\begin{lemma}[Unification progress]
  If unification problem $\up$ cannot take a step $\up \unif \upp$, then either:
  \begin{enumerate}[(\roman*)]
    \item $\up$ is solved.
    \item $\up$ is $\cfalse$.
    \end{enumerate}
  \begin{proof}
    This is a standard result. See \citet*{Pottier-Remy/emlti}.
  \end{proof}
\end{lemma}


\begin{theorem}[Progress]
  \label{thm:progress}
  If constraint $\c$ cannot take a step $\c \csolve \cp$, then either:
  \begin{enumerate}%[(\roman*)]
    \item $\c$ is solved.
    \item $\c$ is stuck, it is either:
      \begin{enumerate*}
      \item
        \label {item/progress/false}
        $\cfalse$;
      \item
        \label {item/progress/scope/x}
        $\hat\C\where{\capp \x \t}$ where $\x \disjoint \hat\C$;
      \item
        \label {item/progress/scope/i}
        $\hat\C\where{\cpapp \x \tv \tvc \inst}$ where
        $\x \disjoint \hat\C$ and $\inst.\tv \disjoint \insts {\hat\C}$;
      \item
        \label {item/progress/suspended}
        for every match constraint $\hat\C\where{\cmatch \tv \cbrs}$ in
        $\c$, $\Cshape {\hat\C} \tv \sh$ does not hold for any $\sh$.
      \end{enumerate*}
     Here, $\hat\C$ is a normal context, \ie such that no other
     rewrites can be applied.
  \end{enumerate}
\end{theorem}
\begin{proof}
  We proceed by induction on the structure of $\c$. We
  focus on suspended match constraints, conjunctions, and $\Let$ rules.
  \begin{proofcases}
    \proofcase{$\cmatch \t \cbrs$}
      We have two cases:
      \begin{proofcases}
  \proofcase{$\t$ is a non-variable type} Apply \Rule{S-Match-Ctx} using \Rule{S-Uni-Type}
	\proofcase{$\t$ is a type variable $\tv$}

	  We have $\nCshape \hole \tv$. It suffices
	  that every match constraint in a context-reachable position
	  $\hat\C\where{\cmatch \tvp \cbrs}$ satisfies $\nCshape {\hat\C} \tvp$.
	  By the definition of constraint contexts, there is only one such
	  $\hat\C$, namely $\hole$, for which we already have $\nCshape \hole \tv$.
	  Hence $\cmatch \t \cbrs$ is stuck.
      \end{proofcases}

    \proofcase{$\ca \cand \cb$}
    We begin by inducting on $\ca$ and $\cb$. Then we consider cases:
    \begin{proofcases}
      \proofcase{$\ca$ (or $\cb$) take a step} Apply congruence rewriting rule.
      \proofcase{$\ca$ (or $\cb$) is $\ctrue$} Apply \Rule{S-True}.
      \proofcase{$\ca$ (or $\cb$) is $\cfalse$} Apply \Rule{S-False}.
      \proofcase{$\ca$ (or $\cb$) begins with $\exists$} Apply \Rule{S-Exists-Conj}.
      \proofcase{$\ca, \cb$ are solved}

	We either apply the above $\exists$ case, or both $\ca$ and $\cb$ are solved
	multi-equations $\ueqs_1, \ueqs_2$. We perform cases on this:
	\begin{proofcases}
	  \proofcase{$\ueqs_1$ and $\ueqs_2$ are mergable} Apply \Rule{U-Merge}.
	  \proofcase{$\cyclic {\ueqs_1, \ueqs_2}$} Apply \Rule{U-Cycle}.
	  \proofcase{Otherwise} The conjunction $\ueqs_1 \cand \ueqs_2$ is solved.
	\end{proofcases}

      \proofcase{$\ca$ and $\cb$ are stuck (and not $\cfalse$)}

	\Wlog, consider cases $\ca$.
	\begin{proofcases}
	  \proofcase{$\hat\Ca\where{\capp \x \t}$}
	    We have $\x \disjoint \bvs {\hat\Ca}$.

	    $\hat\Ca\where{\capp \x \t} \cand \cb$ is stuck as we do not bind $\x$ in $\hat\Ca \cand \cb$.
	  \proofcase{$\hat\Ca\where{\cpapp \x \tv \tvc \inst}$}
	    We have $\x \disjoint \bvs {\hat \Ca}$ and $\inst.\tv \disjoint \insts {\hat \Ca}$.

	    If $\inst.\tv \in \insts \cb$ and $\inst \disjoint \bvs {\hat\ca}$, then apply \Rule{S-Inst-Unify}.
	    It must be the case that we can apply \Rule{S-Inst-Unify}, otherwise, we could lift these instantiation
	    constraints using \Rule{S-Exists-Lower} and \Rule{S-Let-ConjLeft}, contradicting that $\hat\Ca$ is stuck.

	    Otherwise, $\x \disjoint \bvs {\hat \Ca \cand \cb}$, thus $\hat\Ca\where{\cpapp \x \tv \tvc \inst}$ is stuck.

	  \proofcase{$\hat\Ca\where{\cmatch \tvp \cbrs}$}
	    We have $\nCshape \Ca \tvp$.

	    Consider a match constraint $\cmatch \tvp \cbrs$ in $\ca$.

      If $\th \Cshape {\where{\hat\Ca\where{-} \cand \cb}} \tvp \sh$. Then we can apply \Rule{S-Match-Ctx}.

      Otherwise $\not \th \Cshape {\where{\hat\Ca\where{-} \cand \cb}} \tvp \sh$. We have \cref{lem:unicity-completeness},
      so we are stuck and $\nCshape {(\Ca \cand \cb)} \tvp$.
	\end{proofcases}

    \end{proofcases}

    \proofcase{$\cletr \x \tv \tvs \ca \cb$}
    We begin by inducting on $\ca$ and $\cb$. Then we consider cases:
    \begin{proofcases}
      \proofcase{$\ca$ (or $\cb$) take a step} Apply congruence rewriting rule.
      \proofcase{$\ca$ (or $\cb$) is $\cfalse$} Apply \Rule{S-False}.
      \proofcase{$\ca$ begins with $\exists$} Apply \Rule{S-Let-ExistsLeft}
      \proofcase{$\cb$ begins with $\exists$} Apply \Rule{S-Let-ExistsRight}
      \proofcase{$\cb$ begins with $\cand$ with $\x \disjoint$ from conjunct} Apply \Rule{S-Let-ConjRight}.
      \proofcase{$\ca$ begins with $\cand$ with $\tv, \tvs \disjoint$ from conjunct } Try apply \Rule{S-Let-ConjLeft}
      \proofcase{$\cb$ begins with $\cexistsi \inst \xp {}$, $\x \neq \xp$} Apply \Rule{S-Exists-Inst-Let}
      \proofcase{$\tvp \in \tvs$ is determined by $\ca$} Apply \Rule{S-Exists-Lower}
      \proofcase{$\cb$ is solved}

	Thus $\cb$ must be $\ctrue$ (due to above cases).
	\begin{proofcases}
	  \proofcase{$\ca$ is solved}
	    Thus $\ca$ must be $\ueqs$.

	    There are two cases:
	    \begin{itemize}
	      \proofcase{$\cexists {\tv, \tvs} \ueqs \cequiv \ctrue$} Apply \Rule{S-Let-Solve}.
	      \proofcase{$\cexists {\tv, \tvs} \ueqs \cnequiv \ctrue$} It must be the case there is some $\tvb$ that dominates a $\tvp$ in $\tv, \tvs$ in $\ueqs$.
		Hence $\cdetermines {\cexists {\tv, \tvs \setminus \tvp} \ueqs} \tvp$.
		So we can apply \Rule{S-Exists-Lower}.
	    \end{itemize}

	  \proofcase{$\ca$ is stuck}

	    The constraint $\cletr \x \tv \tvs \ca \cb$ remains stuck, since
	    no additional term variable bindings occur for the scope of $\ca$,
	    ruling out the instantiation cases. Additionally, we cannot apply
	    backpropagation since $\cb$ is $\ctrue$.
	\end{proofcases}

      \proofcase{$\cb$ is stuck}
	\begin{proofcases}
	  \proofcase{$\hat\C\where{\capp \x \t}$} We have $\x \disjoint \bvs {\hat\C}$.

	  Apply \Rule{S-Let-AppR}.

	  \proofcase{$\hat\C\where{\cpapp \x \tvp \tvc \inst}$} We have $\x \disjoint \bvs {\hat\C}$ or $\inst.\tvp \disjoint \insts {\hat\C}$.
	    \begin{itemize}
		\proofcase{$\tvp \in \reg \tv \tvs$}

		We can either apply \Rule{S-Inst-Copy} or \Rule{S-Compress}
		if a multi-equation involving $\tvp$ occurs in $\ca$.

		Otherwise, we consider cases where $\ca$ is solved or stuck.

		If $\ca$ is solved, then it must be of the form $\ueqs$.
		There are two cases:
		\begin{itemize}
		  \proofcase{$\cexists {\tv, \tvs} \ueqs \cequiv \ctrue$}
		  As $\tvp$ does not appear in the head position of any multi-equation in $\ueqs$,
		  it must be polymorphic. Thus $\cfor \tvp {\cexists {\tv, \tvs \setminus \tvp} \ueqs} \cequiv \ctrue$.
		  So we can apply \Rule{S-Inst-Poly}.

		  \proofcase{$\cexists {\tv, \tvs} \ueqs \cnequiv \ctrue$}
		  Apply \Rule{S-Exists-Lower} (using the same logic as above).

		\end{itemize}

		If $\ca$ is stuck, then neither stuck case regarding instantiations
		in $\ca$ is fixed, so in these cases the constraint remains
		stuck. If $\ca$ is stuck with $\hat\Cp\where{\cmatch \tvb
    \cbrs'}$. Then either backpropagation (via \Rule{S-Uni-BackProp} and \Rule{S-Match-Ctx})
		applies with an equation in $\hat\C$, or the entire constraint
    is stuck (by \cref{lem:unicity-completeness}).

		\proofcase{$\tvp \notin \reg \tv \tvs$} Apply \Rule{S-Inst-Mono}.




	    \end{itemize}

	  \proofcase{For any $\hat\C\where{\cmatch \tvp \cbrs}$} We have $\nCshape {\hat\C} \tvp$.

	  Either $\cletr \x \tv \tvs \ca \cb$ can progress with an instantiation constraint (in the above case) to discharge
	  the match constraint or $\cletr \x \tv \tvs \ca \cb$ is stuck.
	\end{proofcases}
    \end{proofcases}
 \end{proofcases}
\end{proof}

\subsection{Termination}

This section presents a proof of termination for our solver.
%
Most rewrite rules, in both unification and constraint solving, are
\emph{destructive}---that is, they eliminate or modify the structure of a
constraint in a way that prevents the rule from begin applied again.
%
Consequently, to establish termination, it suffices to consider only those
rules that are not inherently destructive.

\begin{lemma}[Unification termination]
  \label{lem:unification-termination}
  The unifier terminates on all inputs.
  \begin{proof}
    \newcommand{\sw}[1]{\mathprefix{sw}{(#1)}}
    \newcommand{\iw}[1]{\mathprefix{iw}{(#1)}}
    \newcommand{\tw}[1]{\mathprefix{tw}{(#1)}}
    \newcommand{\uw}[1]{\mathprefix{uw}{(#1)}}

    Let every shape $\sh$ have an integer \emph{weight}
    defined by $\sw \sh \eqdef 4 + 2 \times |\sh|$, where $|\sh|$ is the
    arity of the shape $\sh$.
    %
    The weight of a type $\tw \t$ is defined by:
    \begin{mathpar}
      \begin{tabular}{RCL}
	\tw \tv &\eqdef& 1\\
	\tw {\shapp \tys} &\eqdef& \iw {\shapp \tys} - 2\\[1ex]
	\iw \tv &\eqdef& 0\\
	\iw {\shapp \tys} &\eqdef& \sw \sh + \iw \tys\\
        \iw \tys & \eqdef & \sum\iton \iw \ti\\

      \end{tabular}
    \end{mathpar}
    The helper $\iw \t$ computes the ``internal'' weight of $\t$; in
    the common case of shallow types it is just the weight of its head
    shape.

    We define the weight of a multi-equation as the sum of the weights of its
    members. The weight of a unification problem $\uw \up$ is defined
    as the sum of the weights of its multi-equations.

    In $\up \unif \upp$, the rules \Rule{U-Decomp} and \Rule{U-Name} are not
    obviously destructive, as they may introduce new constraints that
    are structurally larger than the constraint being rewritten.

    However, we show that this is not problematic: in both cases, the unification
    weight $\uw \up$ strictly decreases. The remaining rules are obviously
    destructive and either maintain or decrease the unification weight.

    \begin{proofcases}
      \proofcaserewrite
	{U-Decomp}
	{\cunif {\pshapp \tvs} {\cunif {\pshapp \tvbs} \ueq}}
	{\cunif {\pshapp \tvs} \ueq \cand \cunif \tvs \tvbs}

	We have:
	\begin{mathpar}
	  \begin{tabular}{RRCL}
           (+)&
	    \uw {\cunif {\pshapp \tvs} {\cunif {\pshapp \tvbs} \ueq}} &=&
	      \tw {\pshapp \tvs} + \tw {\pshapp \tvbs}  + \tw \ueq \\
           (-)&
	    \uw {\cunif {\pshapp \tvs} \ueq \cand \cunif \tvs \tvbs}
	      &=&
	      \tw {\pshapp \tvs} + \tw \ueq + \tw \tvs + \tw \tvbs \Strut \\
              \hline
	       &&=&\Strut \tw {\pshapp \tvbs} - \tw \tvs - \tw \tvbs \\
               &&=&  (\sw \sh  + 0 - 2) - 2 |\sh| \\
               &&=&  (2 + 2|\sh|) - 2 |\sh| \Wide = \textbf {2}\\
	  \end{tabular}
	\end{mathpar}
	Hence $\uw {{\cunif {\pshapp \tvs} {\cunif {\pshapp \tvbs} \ueq}}} > \uw {\cunif {\pshapp \tvs} \ueq \cand \cunif \tvs \tvbs}$.


      \proofcaserewrite
	{U-Name}
      {\cunif {\pshapp \parens{\tys, \ti, \typs}} \ueq \\ \tv \disjoint \tys, \typs, \ueq \\ \ti \notin \TyVars }
      {\cexists \tv {\cunif {\pshapp \parens{\tys, \tv, \typs}} \ueq \cand \cunif \tv \ti}}

	Given $\ti \notin \TyVars$, by \cref{thm:principal-shapes},
	$\ti = \shapp[\shp] \bar\typp$ for some shape $\shp$ and types $\bar\typp$.
	So we have:
	\begin{mathpar}
	  \begin{tabular}{.R;;R;C;L.}
          (+) &
	    \uw {\cunif {\pshapp {\parens{\tys, \ti, \typs}}} \ueq}
            &=& \sw \sh + \iw \tys + \iw \ti + \iw \typs - 2 + \uw \ueq \\
          (-) &
	    \uw {\cexists \tv {\cunif \tv \ti \cand \cunif {\pshapp \parens{\tys, \tv, \typs}} \ueq}}
            &=& \sw \sh + \iw \tys + 0 + \iw \typs - 2 + \uw \ueq + 1 + \tw \ti \\
            \hline
           &&=& \iw \tyi - \iw \tv - \tw \ti - 1 \\
	    &&=& \iw \tyi - 0 - (\iw \tyi - 2) - 1 \\
	    &&=& \textbf{1}
	  \end{tabular}
	\end{mathpar}

	Hence $\uw {\cunif {\pshapp \parens{\tys, \ti, \typs}} \ueq } > \uw {\cexists \tv {\cunif {\pshapp \parens{\tys, \tv, \typs}} \ueq \cand \cunif \tv \ti}}$.
    \end{proofcases}
  \end{proof}
\end{lemma}

\terminationBIS*
\begin{proof}
  The difficulty for termination comes from the ``suspended match discharge'' rule
  \Rule{S-Match-Ctx} which can make arbitrary
  sub-constraints appear in the non-suspended part of the constraint;
  and from the instantiation rules that copy/duplicate existing
  structure in another part of the constraint, increasing its total
  size.

  As we argued before, the other rewrite rules are \emph{destructive},
  they strictly simplify the constraint towards a normal form and can
  only be applied finitely many times when taken together. The
  fragment without discharge rules and incremental instantiation is
  also extremely similar to the constraint language of
  \citet*{Pottier-Remy/emlti}, so their termination proof applies
  directly.

  \paragraph{Discharge rules} The discharge rules strictly decrease
  the number of occurrences of suspended match constraint (if we also
  count nested suspended constraints), and no rewriting rule
  introduces new suspended match constraints. So these discharge rules
  can only be applied finitely many times. To prove termination of
  constraint solving, it thus suffices to prove that rewriting
  sequences that do not contain one of the discharge rules (those that
  occur in-between two discharge rules) are always finite.

  \paragraph{Starting instantiations} By a similar argument, the number
  of non-incremental instantiations $\capp \x \t$ decreases strictly on
  \Rule{S-Let-AppR} when an incremental instantiation starts, and is
  preserved by other non-discharge rules. The rule \Rule{S-Let-AppR}
  can thus only occur finitely many times in non-discharging sequences,
  and it suffices to prove that all rewriting sequences that are
  non-discharging and do not contain \Rule{S-Let-AppR} are finite.

  \paragraph{Other instantiation rules} Among other instantiation
  rules, the rule of concern is \Rule{S-Inst-Copy}, which is not
  destructive: it introduces new instantiation constraints
  and structurally increases the size of the constraint.

  Intuitively, \Rule{S-Inst-Copy} should not endanger termination
  because the amount of copying it can perform for a given
  instantiation is bounded by the size of the types in the constraint
  $\c$ it is copying from. ($\c$ could have cyclic equations with
  infinite unfoldings, but \Rule{S-Inst-Copy} forbids copying in that
  case.) The difficulty is that rewrites to $\c$ can be interleaved
  with instantiation rules, so that the equations that are being
  copied can grow strictly during instantiation.

  To control this, we perform a structural induction: to prove that
  $(\cletr \x \tv \tvs \ca \cb)$ does not contain infinite
  non-discharging non-instance-starting rewrite rules, we can assume
  that the result holds for the strictly smaller constraint $\ca$, and
  then prove termination of the incremental instantiations of $\x$ in
  $\cb$. (The notion of structural size used here is preserved by
  non-discharging rewrite rules, as they do not affect the
  \Let-structure of the constraint.)

  Assuming that $\ca$ has no infinite rewriting sequence, it suffices
  to prove that only finitely many rewrites in the rest of the
  constraint (namely $\cb$) can occur between each rewrite of $\ca$.

  \newcommand{\sw}[1]{\mathprefix{sw}{(#1)}}
  \newcommand{\iw}[1]{\mathprefix{iw}{(#1)}}
  \newcommand{\stw}[1]{\mathprefix{tw}{(#1)}}
  \newcommand{\tw}[2]{\mathprefix{tw}{(#1 \in #2)}}
  \newcommand{\eqs}[1]{\mathprefix{eqs}({#1})}
  \newcommand{\cw}[1]{\mathprefix{cw}{(#1)}}

  We define a weight that captures the contribution of types
  within $\ca$ to the partial instances in~$\cb$:
  \begin{mathpar}
    \begin{tabular}{RCL}
      \stw {\shapp \tys} &\eqdef& 2 \times \sw \sh + \sum \iton \stw \ti \\[1ex]
      \stw \tv &\eqdef&
      \left\{
        \begin{array}{ll}
        \sup \Braces {\stw \t : \cunif \tv \t \in \ca } & \text{if $\ca$ is acyclic} \\
        0 & \text{otherwise}
        \end{array}
      \right.
    \end{tabular}
  \end{mathpar}
  The weight of a partial instantiation $\cw {\cpapp \x \tv \t \inst}$ is
  defined as  the sum of $\stw \t$ and ${\stw \tv}$.
  The weight of other constraints is given using the measure
  $\mathprefix{uw}$ defined in the the
  proof of \cref{lem:unification-termination}.

  \begin{proofcases}
    \proofcaserewrite
      {S-Inst-Copy}
      {\cletr \x \tv \tvs {\c}
	\C\where{\cpapp \x \tvp \tvc \inst}\\
	\c = \cp \cand \cunif \tvp {\cunif {\shapp \tvbs} \ueq}\\
	\tvp \in \reg \tv \tvs \\
	\neg \cyclic {\c} \\
       \tvbs' \disjoint \tvp, \tvc, \tvbs \\
       \x \disjoint \bvs \C}
      {\cletr \x \tv \tvs {\c}
	\C\where{\cexists {\tvbs'} \cunif \tvc {\shapp \tvbs'} \cand \cpapp \x {\tvbs} {\tvbs'} \inst}}

      We aim to show that the weight of the rewritten constraint
      $\cexists {\tvbs'} \cunif \tvc {\shapp \tvbs'} \cand \cpapp \x {\tvbs} {\tvbs'} \inst$
      is strictly less than the original $\cpapp \x \tvp \tvc \inst$.

      \begin{mathpar}
	\begin{tabular}{RCL}
	  \cw {\cpapp \x \tvp \tvc \inst} &=& 1 + \stw \tv \\
	  &\geq& 1 + 2 \times \sw \sh + \sum\iton \stw \tvbi  \\[1ex]
	  \cw {\cexists {\tvbs'} \cunif \tvc {\shapp \tvbs'} \cand \cpapp \x {\tvbs} {\tvbs'} \inst}
	  &=&
	  1 + \sw \sh + \sum\iton \stw \tvbi + |\tvbs'|
	\end{tabular}
      \end{mathpar}
      To ensure a strict decrease, it suffices to show that $\sw \sh > |\tvbs'|$.
      Given that $|\tvbs'| = |\sh|$, and by the definition of $\sw \sh$, this inequality holds.
      Therefore, the weight strictly decreases under \Rule{S-Inst-Copy}.

  \end{proofcases}

  Thus the constraint solver terminates.
\end{proof}

\subsection{Correctness}

\begin{lemma}
  \label{lem:unsat-match}
  Given non-simple $\c$ constraint. If every match constraint $\C\where{\cmatch \t \cbrs} = \c$
  satisfies $\nCshape \C \t$, then $\c$ is unsatisfiable.
  \begin{proof}
    By contradiction, inverting on the canonical derivation of $\c$.
  \end{proof}
\end{lemma}

\begin{lemma}[Scope preservation]
  \label{lem:scoping-preservation}
  For all $\ca, \cb$, if $\ca \csolve \cb$, then $\fvs \ca \supseteq \fvs \cb$.
  \begin{proof}
    By induction on $\ca \csolve \cb$.
  \end{proof}
\end{lemma}

\begin{corollary}
  \label{corollary:correctness}
  For the closed-term-variable constraint $\c$, $\c$ is satisfiable if and only if
  $\c \csolve^* \hat\c$ and $\hat\c$ is a solved form equivalent to $\c$.
  \begin{proof}
    We show each direction individually:
    \begin{proofcases}
      \proofcase{$\implies$}

	By transfinite induction on the well-ordering of constraints whose
	existence is shown in \cref{thm:termination}.

	We have $\c$ is satisfiable.
	By \cref{thm:progress}, we have three cases:
	\begin{proofcases}
	  \proofcase{$\c$ is solved}
	  We have $\c \csolve^* \c$ and $\c \cequiv \c$ by reflexitivity.
	  So we are done.

	  \proofcase{$\c$ is stuck}
	    Given $\c$ is a closed-term-variable constraint,
	    it must be the case that either $\c$ is $\cfalse$
      \emph{or} $\hat\C\where{\cmatch \t \cbrs}$ and $\nCshape \C \t$.

	    If $\c$ is $\cfalse$, this contradicts our assumption that $\c$ is satisfiable.
	    Similarly, by \cref{lem:unsat-match}, if $\c$ is $\hat\C\where{\cmatch \t \cbrs}$,
	    then this also contradicts the satisfiability of $\c$.

	  \proofcase{$\c \csolve \cp$}

	  By \cref{thm:preservation}, we have $\c \cequiv \cp$, thus $\cp$ is satisfiable.
    Additionally, by \cref{lem:scoping-preservation}, we have $\fvs \cp = \eset$.
	  So by induction, we have $\cp \csolve^* \hat \c$ and $\hat\c$
	  is a solved form equivalent to $\cp$.
	  By transitivity of equivalence, we therefore have $\hat\c \cequiv \c$, as
	  required.

	\end{proofcases}

      \proofcase{$\impliedby$}

      By induction on the rewriting $\c \csolve^* \hat\c$.
      \begin{proofcases}
	\proofcaserewrite
	  {Zero-Step}
	  { }
	  {\hat\c \csolve^* \hat\c}

	We have $\c = \hat\c$ by inversion. All solved forms are satisfiable, thus
	$\c$ is satisfiable.

	\proofcaserewrite
	  {One-Step}
	  {\c \csolve \cp \\ \cp \csolve^* \hat\c}
	  {\c \csolve^* \hat\c}

	  By induction, we have $\cp$ is satisfiable. By \cref{thm:preservation},
	  $\c \cequiv \cp$, hence $\c$ is satisfiable.

      \end{proofcases}
    \end{proofcases}
  \end{proof}
\end{corollary}

\section{Properties of \OML}
\label{app/oml/proofs}

% What we do

This section states and proves the two central metatheoretic properties of
\OML. The first is the \emph{soundness and completeness} of the constraint
generator $\cinfer \e \tv$ with respect to the \OML typing rules. The second
is the existence of \emph{principal types}, which follows as a consequence
of soundness and completeness: every closed well-typed term $\e$ admits a
most general type.

% Closed terms

Throughout this section, we restrict our attention to \emph{closed
terms}. This is because the typing context $\G$ can contain bindings to
terms whose type is ``guessed''. When we generate constraints for a term
$\e$ under a context $\G$, we encode the type schemes in $\G$ as part of the
constraint itself using \Let-constraints. However, these schemes are
treated as known within the constraint! As a result, we assume terms are
closed from the outside to avoid $\G$ leaking any guessed type information.

\subsection{Simple syntax-directed system}

As a first step towards proving soundness and completeness of constraint
generation, we first present a variant of the \OML type system for
\emph{simple terms}. For this system, the syntax tree completely determines
the derivation tree.

We use the standard technique of removing the \Rule{Inst} and \Rule{Gen} rules,
and always apply instantiations in \Rule{Var} (\Rule{Var-SD}) and always
generalize at let-bindings (\Rule{Let-SD}). We can show that this system is
sound and complete with respect to the declarative rules.

\begin{theorem}[Soundness of the syntax directed rules]
  \label{thm:soundness-sd}
  Given the simple term $\e$.
  If $\G \thsimplesd \e : \t$ then we also have $\G \thsimple \e : \t$
  \begin{proof}
    Induction on the given derivation.
  \end{proof}
\end{theorem}

\begin{theorem}[Completeness of the syntax directed rules]
  \label{thm:completeness-sd}
  Given the simple term $\e$.
  If $\G \thsimple \e : \ts$, then $\G \thsimple \e : \t$ for any instance $\t$ of $\ts$.
  \begin{proof}
    Induction on the given derivation.
  \end{proof}
\end{theorem}

\paragraph{Inversion} On a simple syntax-directed derivation $\G \thsimplesd \e : \t$, we have the usual inversion
principle:
\begin{lemma}[Simple inversion]
  \label{lem:simple-inversion-sd}
  ~
  \begin{enumerate}[(\roman*)]
    \item If $\G \thsimplesd \x : \t$, then $\x : \tfor \tvs \tp \in \G$ and $\t = \tp\where{\tvs \is \tys}$.
    \item If $\G \thsimplesd \efun \x \e : \t$, then $\G, \x : \ta \thsimplesd \e : \tb$ and $\t = \ta \to \tb$.
    \item If $\G \thsimplesd \eapp \ea \eb : \t$, then $\G \thsimplesd \ea : \tp \to \t$ and $\G \thsimplesd \eb : \tp$.
    \item If $\G \thsimplesd \eunit : \t$, then $\t = \tunit$.
    \item If $\G \thsimplesd \elet \x \ea \eb : \t$, then $\G \thsimplesd \ea : \tp$, $\tvs \disjoint \fvs \G$, and $\G, \x : \tfor \tvs \tp \thsimplesd \eb : \t$.
    \item If $\G \thsimplesd \eannot \e \tvs \tp : \t$, then $\G \thsimplesd \e : \tp\where{\tvs \is \tys}$ and $\t = \tp\where{\tvs \is \tys}$.
   \item If $\G \thsimplesd \etuple {\ea, \ldots, \en} : \t$, then $\G \thsimplesd \ei : \ti$ for all $1 \leq i \leq n$ and $\t = \tProd \ti$.
    \item If $\G \thsimplesd \exproj \e j n : \t$, then $\G \thsimplesd \e : \tProd \ti$ and $\t = \tj$, with $n \geq j$.
    \item If $\G \thsimplesd \expoly \e \tvs {\tfor \tvbs \tp} : \t$, then $\G \thsimplesd \e : \t\where{\tvs \is \tys}$, $\tvbs \disjoint \G$ and
      $\t = \tpoly {\tfor \tvbs \tp}\where{\tvs \is \tys}$.
    \item If $\G \thsimplesd \exinst \e \tvs \ts : \t$, then $\G \thsimplesd \e : \tpoly \ts\where{\tvs \is \tys}$ and $\ts\where{\tvs \is \tys} \leq \t$.
    \item If $\G \thsimplesd \emagic \es : \t$, then $\G \thsimplesd \ei : \tip$ for all $1 \leq i \leq n$.
    \item If $\G \thsimplesd \exrecord \T {\overline{\elab = \e}} : \t$, then $\G \thsimplesd \ei : \ti$ and ${\labfrom \elab \T} \leq \t \to \ti$ for $1 \leq i \leq n$ and $\Dom {\labfrom \labenv \T} = \elabs$.
    \item If $\G \thsimplesd \erecord {\overline{\elab = \e}} : \t$, then $\labsuni \elabs \T$ and $\G \thsimplesd \exrecord \T {\overline{\elab = \e}} : \t$.
    \item If $\G \thsimplesd \exfield \e \T \elab : \t$, then $\G \thsimplesd \e : \tp$, ${\labfrom \elab \T} \leq \tp \to \t$.
    \item If $\G \thsimplesd \efield \e \elab : \t$, then $\labuni \elab \T$ and $\G \thsimplesd \exfield \e \T \elab : \t$.
  \end{enumerate}
\end{lemma}

\subsection{Canonicalization of typability}
Our system satisfies a similar canonicalization theorem to constraint satisfiability.

\begin{lemma}[Composability of unicity]
  ~
  \label{lem:comp-unicity-typing}
  \begin{enumerate}[(\roman*)]
    \item If $\Eshape \Ea \es \sh$, then $\Eshape {\Eb\where\Ea} \es \sh$.
    \item If $\eshape \Ea \e \sh$, then $\eshape {\Eb\where\Ea} \e \sh$.
  \end{enumerate}
  \begin{proof}
    By induction on $\Eb$.
  \end{proof}
\end{lemma}

\begin{lemma}[Decanonicalization]
  \label{lem:decanonicalization-typing}
  If $\Th \e : \t$, then $\eset \th \e : \t$.
  \begin{proof}
    By induction on the given derivation $\Th \e : \t$.
  \end{proof}
\end{lemma}

\newcommand{\enimplicit}[1]{{\#\mathprefix[\mathsf]{implicit} {#1}}}
\begin{theorem}[Canonicalization]
  \label{thm:canonicalization-typing}
  If $\th \e : \ts$, then $\Th \e : \t$ for any instance $\t$ of $\ts$.
  \begin{proof}
    By induction on the following measure of $\e$:
    \begin{mathpar}
      \| \e \| \uad\eqdef\uad \angles {\enimplicit \e, |\e|}
    \end{mathpar}
    where $\angles \ldots$ denotes a lexicographically ordered pair, and
  \begin{enumerate}

    \item $\enimplicit \e$ is the number of implicit constructs in $\e$ \ie overloaded tuple projections $\eproj \e j$,
      implicit non-unique field projections $\efield \e \elab$, implicit non-unique records $\erecord {\overline{\elab = \e}}$, polytype instantiations $\einst \e$
      and polytype boxing $\epoly \e$.

    \item the last component $|\e|$ is a structural measure of terms \ie a
      application $\eapp \ea \eb$ is larger than the two terms $\ea, \eb$.
  \end{enumerate}
  This measure is analogous to the measure $\cmeasure \c$ for constraints.
  \end{proof}
\end{theorem}

\subsection{Unifiers}

A substitution $\sub$ is an idempotent function from type variables to types.
The (finite) domain of $\sub$ is the set of type variables such that $\sub(\tv)
\neq \tv$ for any $\tv \in \dom \sub$, while the codomain consists of the free
type variables of its range.
%
We use the notation $\where{\tvs \is \tys}$ for the substitution $\sub$ with
domain $\tvs$ and $\sub(\tvs) = \tys$.

The constraint induced by a substitution $\sub$, written $\exists \sub$, is
$\cexists {\tvbs} \tvs = \tys$ where $\tvbs = \rng \sub$, $\tvs = \dom \sub$
and $\sub(\tvs) = \tys$.

\begin{definition}[Unifier]
  A substitution $\sub$ is a unifier of $\c$ if $\exists \sub$ entails $\c$.
  A unifier $\sub$ of $\c$ is \emph{most general} when $\exists \sub$ is equivalent
  to $\c$.
\end{definition}

\begin{lemma}[Simple inversion of unifiers]
  \label{lem:unifier-simple-inversion}
  ~
  \begin{itemize}
    \item If $\sub$ is a unifier of $\cunif \ta \tb$, then $\sub(\ta) = \sub(\tb)$.
    \item For simple $\ca, \cb$, if $\sub$ is a unifier of $\ca \cand \cb$, then $\sub$ is a unifier of $\ca$ and $\cb$.
    \item For simple $\c$, if $\sub$ is a unifier of $\cexists \tv \c$, then $\sub\where{\tv \is \t}$ is a unifier of $\c$ for some $\t$.
    \item For simple $\c$, if $\sub$ is a unifier of $\cfor \tv \c$, then $\sub$ is a unifier of $\c$.
  \end{itemize}
  \begin{proof}
    Follows by simple inversion.
  \end{proof}
\end{lemma}

\begin{lemma}
  \label{lem:unifier-abs-equiv}
  If $\sub$ unifies $\cexists \tv \c$, then there exists a unifier $\subp$ that extends $\sub$ with $\tv$,
  where $\subp$ is most general unifier of $\exists \sub \cand \c$.


  Then $\cabs \tv \c$ is equivalent to $\cabs \tv \sigma \leq \tv$ under $\sub$, where $\ts = \tfor \tvbs \subp(\tv)$ and
  $\tvbs = \fvs {\subp(\tv)} \setminus \rng \sub$. We write this equivalent constraint abstraction as $\csem {\cabs \tv \c}_\sub$.
  \begin{proof}
    See \citet*{Pottier-Remy/emlti}.
  \end{proof}
\end{lemma}

\begin{lemma}[Let inversion of unifiers]
  For simple $\ca, \cb$.
  If $\sub$ unifies $\clet \x \tv \ca \cb$, then
  $\sub$ unifies $\cexists \tv \ca$ and
  $\sub$ unifies $\cletin \x {\csem {\cabs \tv \ca}_\sub} \cb$
  \begin{proof}
    Follows from \cref{lem:unifier-abs-equiv} and simple inversion.
  \end{proof}
\end{lemma}

\begin{lemma}
  \label{lem:mgus}
  For two substitutions $\sub$, $\subp$. If $\exists \sub \centails \exists \subp$, there exists
  $\subpp$ such that $\sub = \subpp \compose \subp$.
  \begin{proof}
    Standard result, follows from definition of $\exists \sub$.
  \end{proof}
\end{lemma}

\subsection{Soundness and completeness of constraint generation}

\begin{lemma}
  \label{lem:ctxt-gen-correctness}
  For any term context $\E$, term $\e$, $\ctxinfer \E \t \tp \where{\cinfer \e \t} = \cinfer {\E\where{\e}} \tp$.
  \begin{proof}
    By induction on the structure of $\E$.
  \end{proof}
\end{lemma}

\begin{lemma}
  \label{lem:erasure-constraint-gen}
  For any term $\e$, $\cerase {\cinfer \e \t} = \cinfer {\eerase \e} \t$.
  \begin{proof}
    By induction on $\e$.
  \end{proof}
\end{lemma}

\begin{lemma}[Simple soundness and completeness]
  \label{lem:simple-soundness-completeness}
  For simple terms $\e$.
  $\sub(\G) \thsimplesd \e : \sub(\t)$ if and only if $\sub$ is a unifier of $\csem {\G \th \e : \t}$.
  \begin{proof}
    By induction on $\e \simple$.
  \end{proof}
\end{lemma}

\begin{theorem}[Soundness and completeness]
  \label{thm:soundness-and-completeness}
  $\Th \e : \sub(\tv)$ if and only if $\sub$ is a unifier of $\csem {\e : \tv}$
  \begin{proof}
    By induction on the number $n$ of implicit terms in $\e$.
    \begin{proofcases}
      \proofcase{$n$ is $0$}

	\begin{llproof}
	  \simplePf{\e}{Premise}
	  \iffPf{\eset\thsimplesd \e : \sub(\tv)}{\sub \text{ unifies } \csem {\e : \tv}}{\cref{lem:simple-soundness-completeness}}
	  \iffPf{\eset \thsimplesd \e : \sub(\tv)}{\Th \e : \sub(\tv)}{When $\e \simple$}
\Hand	  \iffPf{\Th \e : \sub(\tv)}{\sub \text{ unifies } \csem {\e : \tv}}{Above}
	\end{llproof}

      \proofcase{$n$ is $k + 1$}

	\begin{proofcases}
	  \proofcase{$\implies$}

	  \begin{proofcases}

	    \proofcasederivation
	      {Can-Proj-I}
	      {\eshape \E \e {\any \tvcs \Pi\iton \tvcs} \\ \sub(\G) \Th \E\where{\exproj \e j n} : \sub(\tv)}
	      {\Th \E\where{\eproj \e j} : \sub(\tv)}

	      \begin{llproof}
		\ThTypPf{\sub(\G)}{\E\where{\exproj \e j n}}{\sub(\tv)}{Premise}
		\UnifierPf{\sub}{\csem {\G \th \E\where{\exproj \e j n} : \tv}}{By \ih}
		\eqPf{\csem {\G \th \E\where{\exproj \e j n } : \tv}}{\cletG {\cinfer {\E\where{\exproj \e j n}} \tv}}{By definition}
		\continueeqPf{\cletG {\ctxinfer \E \tvb \tv}\where{\cinfer {\exproj \e j n} \tvb}}{\cref{lem:ctxt-gen-correctness}}
		\equivPf{\cinfer {\exproj \e j n} \tvb}{\cexists {\tvaa \tvcs} {\cinfer \e \tvaa \cand \cunif \tvaa {\Pi\iton \tvcs} \cand \cunif \tvb \tvc_j}}{By definition}
		\continueequivPf{\cexists {\tvaa} \cinfer \e \tvaa \cand \cmatched \tvaa {\any \tvcs \Pi\iton \tvcs}{\cbranch {\cpatprod \tvc j} {\cunif \tvb \tvc}}}{\ditto}
		\UnifierPf{\sub}{\cletG \ctxinfer \E \tvb \tv \where{\cexists \tvaa \cinfer \e \tvaa \cand \ldots}}{Above}
		\eshapePf{\E}{\e}{\any \tvcs \Pi\iton \tvcs}{Premise}
		\LetPf{\C}{\cletG \ctxinfer\E \tvb \tv \where{\cexists \tvaa \cinfer \e \tvaa \cand \hole}}{}
		\vdashPf{\semenv}{\cerase{\C\where{\cunif \tvaa \gt}}}{Premise}
		\eqPf{\cexists \tvaa \cinfer \e \tvaa \cand \cunif \tvaa \gt}{\cexists \tvaa \cinfer {\eannot \e {} \gt} \tvaa}{By definition}
		\decolumnizePf
		\continueeqPf{\cinfer {\emagic {\eannot \e {} \gt}} \tvb}{\ditto}
		\eqPf{\cerase {\C\where{\cunif \tvaa \gt}}}{\cerase {\cletG \ctxinfer \E \tvb \tv \where{\cinfer {\emagic {\eannot \e {} \gt}} \tvb}}}{\ditto}
		\continueeqPf{\cerase {\cletG \cinfer {\E\where{\emagic {\eannot \e {} \gt}}} \tv}}{\cref{lem:ctxt-gen-correctness}}
		\continueeqPf{\cletG \cerase {\cinfer {\E\where{\emagic {\eannot \e {} \gt}}} \tv}}{By definition}
		\continueeqPf{\cletG \cinfer {\eerase {\E\where{\emagic {\eannot \e {} \gt}}}} \tv}{\cref{lem:erasure-constraint-gen}}
		\UnifierPf{\semenv}{\cletG \cinfer {\eerase {\E\where{\emagic {\eannot \e {} \gt}}}} \tv}{Above}
		\ThTypPf{}{\eerase {\E\where{\emagic {\eannot \e {} \gt}}}}{\semenv(\tv)}{By \ih}
		\thTypPf{\eset}{\eerase {\E\where{\emagic {\eannot \e {} \gt}}}}{\semenv(\tv)}{\cref{lem:decanonicalization-typing}}
		\eqPf{\shape \gt}{\any \tvcs \Pi\iton \tvcs}{$\implies$E}
		\shapePf \C \tvaa {\any \tvcs \Pi\iton \tvcs}{Above}
		\UnifierPf{\sub}{\C\where{\cmatch \tvaa {\cbranch {\cpatprod \tvc j} {\cunif \tvb \tvc}}}}{By \Rule{Match-Ctx}}
		\eqPf{\cinfer {\eproj \e j} \tvb}{\cexists \tvaa \cinfer \e \tvaa \cand \cmatchdots \tvaa}{By definition}
		\eqPf{\C\where{\cmatchdots \tvaa}}{\cletG \ctxinfer \E \tvb \tv\where{\cexists \tvaa \cinfer \e \tvaa \cand \ldots}}{\ditto}
		\continueeqPf{\cletG \ctxinfer \E \tvb \tv \where{\cinfer {\eproj \e j} \tvb}}{Above}
		\continueeqPf{\cletG \cinfer {\E\where{\eproj \e j}} \tv}{\cref{lem:ctxt-gen-correctness}}
		\continueeqPf{\csem {\E\where{\eproj \e j} : \tv}}{}
\Hand  		\UnifierPf{\sub}{\csem {\E\where{\eproj \e j} : \tv}}{}
	      \end{llproof}

      \proofcase{\Rule{Can-Poly-I}, \Rule{Can-Use-I}, \Rule{Can-Rcd-I}, \Rule{Can-Rcd-Proj-I}}

	    \begin{llproof}
	      Similar arguments.
	    \end{llproof}
	  \end{proofcases}

	  \proofcase{$\impliedby$}

	  \begin{proofcases}
	    \proofcasederivation
	      {Can-Match-Ctx}
	      {\Cshape \C \tvaa {\any \tvcs \Pi\iton \tvcs} \\
	       \sub \text{ unifies } \C\where{\cmatched \tvaa {\any \tvcs \Pi\iton \tvcs} {\ldots}}}
	      {\sub \text{ unifies } \underbrace{\C\where{\cmatch \tvaa {\cbranch {\cpatprod \tvc j} {\cunif \tvb \tvc}}}}_{\cinfer \e \tv}}

	      \begin{llproof}
		\eqPf{\csem {\e : \t}}{\cletG \cinfer {\E\where{\eproj \e j}} \tv}{Premise}
		\eqPf{\C}{\cletG \ctxinfer \E \tvb \tv\where{\cexists \tv \cinfer \e \tv \cand \hole}}{Premise}
		\UnifierPf{\sub}{\C\where{\cmatched \tvaa {\any \tvcs \Pi\iton \tvcs} {\ldots}}}{Premise}
		\UnifierPf{\sub}{\csem {\E\where{\exproj \e j n} : \tv}}{Above (See $\implies$ direction)}
		\ThTypPf{}{\E\where{\exproj \e j n}}{\sub(\tv)}{By \ih}
		\thTypPf{\Gp}{\E\where{\emagic {\eannot \e {} \gt}}}{\tp}{Premise}
		\eqPf{\Gp}{\eset}{$\E\where{\emagic {\eannot \e {} \gt}}$ is closed}
		\ThTypPf{}{\E\where{\emagic {\eannot \e {} \gt}}}{\tp}{\cref{lem:decanonicalization-typing}}
		\UnifierPf{\where{\tv \is \tp}}{\csem {\E\where{\emagic {\eannot \e {} \gt}} : \tv}}{By \ih}
		\vdashPf{\semenv\where{\tv \is \semenv(\tp)}}{\cinfer {\E\where{\emagic {\eannot \e {} \gt}}} \tv}{By definition}
		\shapePf{\C}{\tvaa}{\any \tvcs \Pi\iton \tvcs}{Premise}
		\eqPf{\shape \gt}{\any \tvcs \Pi\iton \tvcs}{$\implies$E}
		\eshapePf{\E}{\e}{\any \tvcs \Pi\iton \tvcs}{Above}
		\ThTypPf{}{\E\where{\eproj \e j}}{\sub(\tv)}{By \Rule{Can-Proj-I}}
	      \end{llproof}

      \proofcase{$\epoly \e$, $\einst \e$, $\erecord {\overline{\elab = \e}}$, $\efield \e \elab$}

	    \begin{llproof}
	      Similar arguments.
	    \end{llproof}
	  \end{proofcases}
	\end{proofcases}
    \end{proofcases}
  \end{proof}
\end{theorem}

\subsection{Principal types}

\principalTypesBIS*
\begin{proof}

Let $\e$ be an arbitrary closed well-typed term; that is, there exists a type
  $\t$ such that $\th \e : \t$.
  By \cref{thm:soundness-and-completeness}, the constraint $\cinfer \e \tv$ is satisfiable
  (specifically under the unifier $\cunif \tv \t$). By \cref{corollary:correctness}, there
  exists a solved constraint $\hat\c$ such that $\hat\c \cequiv \cinfer \e \tv$.
  From $\hat\c$, we extract a unifier $\sub$. Since $\hat\c \cequiv \exists \sub$,
  it follows that $\sub$ is \emph{most general}.

  We claim that $\sub(\tv)$ is the principal type of $\e$. This amounts to showing:
  \begin{enumerate}[(\roman*)]
    \item
      \label{proof:principal-types:1}
      $\th \e : \sub(\tv)$
    \item
      \label{proof:principal-types:2}
      For any other typing $\th \e : \tp$, then $\tp = \theta(\sub(\tv))$ for some $\theta$.
  \end{enumerate}
  Since $\sub$ is a unifier of $\cinfer \e \tv$, it follows immediately from
  \cref{thm:soundness-and-completeness} that $\th \e : \sub(\tv)$, proving \ref{proof:principal-types:1}.
  For \ref{proof:principal-types:2}, suppose $\th \e : \tp$ for some $\tp$. Then by
  \cref{thm:soundness-and-completeness} again, there exists a unifier $\subp$ of $\cinfer \e \tv$
  such that $\subp(\tv) = \tp$. Since $\sub$ is most general, we have $\exists \subp \centails \exists \sub$,
  and by \cref{lem:mgus}, this implies the existence of a substitution $\subpp$ such that
  $\subp = \subpp \compose \sub$.
  Hence, $\tp = \subp(\tv) = \subpp(\sub(\tv))$, witnessing that $\tp$ is an instance of $\sub(\tv)$, as
  required \ref{proof:principal-types:2}.
\end{proof}


\Draft{\section {Further study}

\subsection {Defaulting}
\label {app/default-rules}

Default rules, which does not fit well with \geninst-inference are still
often used in practice, and therefore would deserve further investigation.

\paragraph {Default shapes}

In this section we study a particular form of defaulting where, rather than
general default rules that could fire any constraint, we restrict to default
shapes. That is, we may attach a default shape $\sh$ to a match constraints
$\cmatch \t \cbrs$, which is then written $\cmatch \t \cbrs \cdefault
\sh$. The default shape $\sh$ can then be used to force the shape of
$\t$ when it could not be determined from context.

%% We show that under some conditions for well-behaved defaulting there is
%% actually an optimal strategy.

Restricting to default shapes has several benefits.  First, a strategy $\S$
can be reduced to the choice of a mapping from constrains $\c$ to of a
subset $\Sfire \S \c$ of suspended constraints of $\suspc$ that should be
defaulted, simultaneously.  The behavior is then entirely determined,
reusing the same logic that runs when the shape is determined by the context
instead of been forced by the default clause.  In particular, this ensures
that the same behavior could have been obtained by an explicit shape
constraint in the source.

\paragraph {Strategies}

We write $\Sempty$ for the empty strategy that never defaults (and thus fails
on all constraints with leftover suspensions) and $\Sfull$ the full strategy
that defaults all suspended constraints, simultaneously.
%% We can formulate some good properties that defaulting strategies should
%% satisfy.
A strategy $\S$ is \emph{reasonable} if for all constraints
$\c$, $\S (\c)$ succeeds more often than the empty strategy on C.
%
This criterion rules out weird strategies that would default a suspension
before solving the other constraints that could discharge this suspension,
possibly with another shape, hence a different output.

A strategy $\S$ applied to a constraint $\c$, either allows to solve $\c$,
hence with a principal solution written $\Ssol \S$ or ends in error ($\Ssol
\S$ is equal to $\bot$).  Let use write $\Sols$ for the union of
all $\Ssol \S$ for all successful reasonable strategies $\S$.
%
We say that $\S$ is non-ambiguous if, for any $\c$, $\Ssol \S$ is $\bot$
whenever $\Sols$ has more than two elements.  This condition forces a
non-ambiguous strategy to fail instead of picking an arbitrary solution when
different defaulting strategies would give incompatible solutions.

A \emph{good} strategy as one that is both \emph{reasonable} and
\emph{non-ambiguous}. A good strategy should not fail more often that
$\Sempty$ nor succeed when there are more than one possible solution.
%
We claim that there is an optimal \emph{good} strategy $\Sopt$ that explores all
possible default subsets, succeeds when there is exactly one principal
solution, then following a successful strategy, and fails otherwise.

Unfortunately, $\Sopt$ is inefficient: as described , it runs in time
exponential in the number of remaining suspended constraints.  Therefore, we
should seek for sub-optimal, but more efficient good strategies.

\paragraph {Dependencies}

Given a stuck constraint $\c$, we may look at the dependency order between
its suspended constraints.
\begin{version}{}

Intuitively, a suspended constraint  $\suspcb$
%% of the form $(\cmatch \tvab \cbrbs \cdefault \shb)$
depends on another suspended constraint $\suspca$
%% of the form $(\cmatch \tvaa \cbras \cdefault \sha)$,
%% and we write $\suspca \cprec \suspcb$,
if defaulting  $\suspca$ allows to solve $\suspcb$---without further
defaulting.  Unfortunately, this is too strong as a suspended constraint may
help resolve another one without solving it. For instance, defaulting two
suspended constraints could be needed to allow solving a third one.

\end{version}
A suspended constraint $\suspcb$ depends on another suspended
constraint~$\suspca$ if there exists a strategy $\S$ that can solve
$\suspcb$ only after defaulting $\suspca$.  That is, if successively
defaulting $\suspca, \bar \suspcs$ solves $\suspcb$ while $\bar \suspcs$
does not.
%
This defines a partial ordering~$\cprec^s$ on suspended constraints of~$\c$,
which is however expensive to compute.

Therefore, we define a syntactical over approximation of $\cprec$ that is
easier to compute. We assume that $\c$ is partially solved. That is, it is
stuck, but excluding cases (\ref{item/progress/scope/x}) and
(\ref{item/progress/scope/i}) which cannot occur with well-formed
constraints.  The solved part of $\cerase {\c}$ defines a partial
(structural domination) ordering $\cprec$ on variables.

$\cerase {\c}$ may also contains incremental instantiations
\emath{\cpapp \x \tv \tvc  \inst}, which implies that
knowing shape of $\tv$ will also determine the shape of $\tvc$.  We thus
extend the $\cprec$ ordering with $\tv \cprec \tvc$ for each partial
instantiation \emath{\cpapp \x \tv \tvc \inst}.

Notice that \emath{\cpapp \x \tv \tvc \inst} has also the potential to force
the unification of $\tv$ and $\tvc$, which will happen whenever the scope of
$\tv$ is lowered, \eg by instantiation of $\tv$ without necessarily
determining the shape of $\tv$.  Hence, one may be tempted to add a reverse
edge from $\tvc$ to $\tv$. However, this may only happen if $\tv$ has been
reached, hence $\tvc$ is already considered as reached, hence determined,
even if its shape is not yet known.

Finally,   for each suspended constraints $\cmatch \tvi {\cbrs_i} \cdefault
\shi$ of $\c$, let $\bar\tvbs_i$ be the set of free variables of $\cbrs_i$
and add $\tvi \cprec \tvci$ for each $\tvci$ in $\tvbs_i$.  Indeed, whenever
the shape of $\tvi$ is determined, the constraint $\suspci$ will be
released, and since it is being treated opaquely, it could do anything with
its free variables, hence fully determined they shapes.  This may release
unification constraints between variables of $\tvbs_i$. Hence, we may be
tempted to claim equivalences between them. However, this may only happened
if $\tvi$ has been reached, and thus all variables of $\tvbs_i$ are
considered as reached as well, hence potentially determined.

The partial ordering between variables induces a partial ordering between
suspended constrains: $\suspca \cprec \suspcb$ whenever $\tvaa \cprec \tvab$
and $\suspci$'s are of the form $(\cmatch \tvai {\bar\cbri} \cdefault
\shi)$.

The syntactic dependency ordering $\cprec$ is an over approximation of the
semantic dependency ordering $\cprec^s$.  Hence, a suspended constraint that
is minimal for the syntactic ordering is also minimal for the semantics
ordering---it depends on no other suspended constraints.

A strategy that always defaults a minimal constraint is
\begin{enumerate*}
\item reasonable;
\item non-ambiguous; and\goodbreak
\item \emph{complete}, \ie it preserves the set of solutions $\Sols$.
\end{enumerate*}
Indeed, either a strategy does not discharge this minimal constraint, and it
fails on $\c$, or it discharges this constraint and this must be via
defaulting, given minimality, and for minimal constraints the defaulting
order does not affect the result.  In fact, all constraints that are minimal
can also be defaulted simultaneously.

However, such a strategy may be stuck if there is no minimal element, but
only a minimal cycle of constraints (there is always one). \XDR{Can we still
say that the strategy is complete? Is then the empty strategy complete? We
should distinguish being stuck from returning a failure.}

If there is a cycle of constraints that are minimal, then defaulting them
altogether, which we call the \emph{whole-cycle} strategy, is reasonable and
non-ambiguous, but not complete.  For example, consider the constraint:
\begin{mathpar}
\Wedges {
\cmatch \tv {
    \cbranch \wild {\cunif \tvb \tint}
    } \cdefault \tint
\\
\cmatch \tvb {
    \cbranch\wild {\cunif \tva \tbool}
    } \cdefault \tbool
}
\end{mathpar}
that are clearly interdependent.  Defaulting $\set \tva$ succeeds while
defaulting $\set \tvb$ or $\set {\tva, \tvb}$ fails.  The whole-cycle
strategy $\set{\tva, \tvb}$ fails here, so it is not complete.

\paragraph {Opaque suspended constraints}

Notice however, that if we treat the suspended constraints as opaque, we
cannot distinguish this example from the following where defaulting only
$\set
\tva$ would incorrectly succeed in an ambiguous situation. Hence, it would
not be a non-ambiguous strategy:
\begin{mathpar}
\Wedges{
\cmatch \tv {
    \cbranch \tint  {\cunif \tvb \tint} \mid
    \cbranch \tbool {\cunif \tvb \tint}
    } \cdefault \tint
\\
\cmatch \tvb {
    \cbranch \tint  {\cunif \tva \tint} \mid
    \cbranch \tbool {\cunif \tva \tbool}
    } \cdefault \tbool
}
\end{mathpar}
Indeed, defaulting $\set{\tva}$ gives
\relax $\wedges {\cunif \tva \tint \\ \cunif \tvb \tint}$,
whereas defaulting $\set \tvb$ gives
\relax $\wedges {\cunif \tva \tint \\ \cunif \tvb \tbool}$;
and defaulting $\set {\tva,\tvb}$ fails. This constraint is ambiguous.
Hence, the whole-cycle strategy \emph{correctly} fails.

Intuitively, treating suspended constraints as opaque is a way to compensate
for the over-approxi\-mation of syntactic dependencies by allowing allow to
choose for opaque constraints concrete constraints that actually create
all the dependencies used in the approximation.

We call a strategy \emph{opaque} when it only depends on the syntactic
dependency relation and not on the actual branches of the suspended
constraints.
%
For example, the optimal strategy is \emph{not} opaque, as it behaves
differently on the two examples above whereas they have the same syntactic
dependencies.

\begin{property}
The whole-cycle strategy is optimal among opaque strategies.
\end{property}
\begin{proof}[Proof hint]{}
We may reduce the general case to a specific cycle of size $n$
(one minimal element of the topological sort of the dependency order).
On this cycle, all opaque strategies are characterized by which subset of
the cycle they default.

To show that defaulting all together is optimal:
\begin{enumerate*}

\item
  We show that it is not worse than defaulting less.
\item
  Conversely, we may build a specific example with an $n$-cycle of
  suspended constraints where defaulting all at once succeeds, and
  defaulting strictly less fails.

\end{enumerate*}
The opacity of suspended constraints also leaves us enough freedom to
enforce all the worse case dependencies in the definition of $\cprec$.
\end{proof}

\begin{version}{\Draft}
Computing the syntactic dependencies $\cprec$ is in $O(n \log n)$ where $n$
is the size of the constraint (not just of suspended constraints).  However,
it is not stable by defaulting. Indeed, this may remove (and add) some
syntactic dependencies, which requires the re-computation of $\prec$.
Incremental computation of dependencies with both edge removal and insertion
may be needed. In any case, experimentation will be needed to
understand whether this is a critical issue.
\end{version}
}{}

\clearpage
\setcounter{tocdepth}{1}
\tableofcontents


\end{document}

% LocalWords:  omnidirectional typecheck polymorphism Hindley Milner kinded
% LocalWords:  GADTs typechecked codomain typechecking subexpressions Bodin
% LocalWords:  monomorphic subexpression Dunfield Riboulet jfla subtyping
% LocalWords:  greek Chargueraud typable monotype polytype Garrigue Remy th
% LocalWords:  impredicative polytypes minimality RCL ary Proj toplevel mlf
% LocalWords:  typability backpropagation arity Compositionality equi Damas
% LocalWords:  equitypable compositionality inlined equitypability nullary
% LocalWords:  metatheoretical finiteness nonvariable mydesc Inlining Yi na
% LocalWords:  unicity inlining typedness pincipality scrutinee equalities
% LocalWords:  declaratively LeBotlan directionality polymorphically mleth
% LocalWords:  directionally typecheckers typechecker disambiguates unshare
% LocalWords:  typechecks acyclic emlti Disambiguating disambiguated GADT
% LocalWords:  prototyped instantiation Monotypes postfixed unsatisfiable
% LocalWords:  satisfiable matchee datatypes arbritraty generalizable Unif
% LocalWords:  monomorphization desugars Forall satisfiability formers iff
% LocalWords:  equirecursive principalShapes subterms Susp Ctx foundedness
% LocalWords:  unsatisfiability getx olymorphic nstances ive typings LetR
% LocalWords:  metavariable subterm Metatheory  monotypes Rcd Henglein phd
% LocalWords:  ExistsLeft ExistsRight ConjLeft ConjRight equational AppR Wf
% LocalWords:  existentials instantiations desugared susp ively renamings
% LocalWords:  quantifications subcomponents BackProp Tarjan's invariants
% LocalWords:  rossberg wasm WebAssembly dunfield krishnaswami jones Huet's
% LocalWords:  huet unif conf Vytiniotis Peyton Schrijvers Sulzmann Leijen
% LocalWords:  disambiguating namespace implicits Bour Yallop Didier Annot
% LocalWords:  acyclicity canonicalization aaa formedness unannotated inX
% LocalWords:  casted inHole inAnnot unsubstituted oml RCLL Scm LCL Decomp
% LocalWords:  cheatsheet inI ProjX PolyX UseX unifier Gc Defn sw iw
% LocalWords:  InstLeft InstRight mergable tw uw RRCL eqs priori multiset
% LocalWords:  reflexitivity metatheoretic lexicographically Unifiers TODO
% LocalWords:  rcl env Pcd illtyped inHole Ui unfoldings welltyped mprset
% LocalWords:  lessskip Xbla Ybla explicitating Composability composability
% LocalWords:  determinacy subcases unifications annotatability approxi Rcx
% LocalWords:  mation macroified cpoint concretize Dom omnidirectionality
% LocalWords:  hindley popl garrigue monomorphically polyparams polyml Chk
% LocalWords:  impredicativity PolyML knownness mathpar nihilo fn
% LocalWords:  Bidirectionalized destructors unparsable metalevel Erwig CPP
% LocalWords:  Variational ifdef variational benevs goldilocks
% LocalWords:  Pottier's unsuspended sep Det Esc macroify ConRight cie Incr
% LocalWords:  explicits mathescape annotability QuickLook Zp
% LocalWords:  inferrable destructor undecidability
